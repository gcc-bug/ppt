\documentclass[aspectratio=1610]{beamer}
%Information to be included in the title page:

\usetheme{Madrid}
\input{smile_styles.tex}

\title{Sample title}
\author[Gcc]{Dingchao Gao}
\institute[ISCAS]{Institute of Software Chinese Academy of Sciences}

\begin{document}

\begin{frame}[plain]
  \titlepage
  %  Title page
\end{frame}
\section{induction}
\begin{frame}
  SIS, 
\end{frame}

\section{example}
\begin{frame}{problem \cite{example}}
  \begin{figure}[htbq]
    \centering
    \includegraphics[width=0.5\textwidth]{figure/problem.png}
    \caption{Zed city as an undirected graph} 
    \label{fig-zed}
  \end{figure}
\end{frame}
\begin{frame}[fragile]
  \frametitle{boolean function}
  \begin{columns}
    \begin{column}{0.45\linewidth}
      \begin{block}{python implementation of f }
        \begin{lstlisting}[language=Python]
def f(v0, ..., v6 : BitVec(2)) -> BitVec(1):
  c0 = (v0 != ’00’)
  c1 = (v1 != ’01’) and (v1 != v0)
  c2 = (v2 != ’00’) and (v2 != ’10’) and (v2 != v0)
  c3 = (v3 != ’00’) and (v3 != v0) and (v3 != v1) and (v3 != v2)
  c4 = (v4 != ’01’) and (v4 != v1) and (v4 != v3)
  c5 = (v5 != ’11’) and (v5 != v2) and (v5 != v3)
  c6 = (v6 != ’11’) and (v6 != v2) and (v6 != v3) and (v6 != v4) and (v6 != v5)
  return c0 and c1 and c2 and c3 and c4 and c5 and c6
          \end{lstlisting}
      \end{block}
    \end{column}
    \begin{column}{0.45\linewidth}
      \begin{block}{hand-optimized python implementation of f }
        \begin{lstlisting}[language=Python]
def f(v0, ..., v6 : BitVec(2)) -> BitVec(1):
  c1 = (v1[0] == v1[1]) and (v3 != v1)
  c023 = ((v0 ˆ v2 ˆ v3) == ’00’)
  c4 = (v4 != v1) and (v4 != v3)
  c5 = (v5 != v2) and (v5 != v3)
  c6 = ((v2 ˆ v3 ˆ v5 ˆ v6) == ’00’) and (v6 != v4)
  return c1 and c023 and c4 and c5 and c6
          \end{lstlisting}
      \end{block}
    \end{column}
  \end{columns}
\end{frame}

\begin{frame}{flow}
  \begin{itemize}
    \item Angel:prepare a uniform quantum state
    given as input a Boolean function
    \item Tweedledum:synthesizing,
    manipulating, and optimizing quantum circuits
    \item Caterpillar:automatically translate the combinational parts of a quantum
    algorithm into quantum gates
  \end{itemize}
\end{frame}
\begin{frame}{initial state \footfullcite{initial}}
  \begin{itemize}
    \item \begin{align}
      \left|\varphi_{j}\right\rangle & = \frac{f}{\sqrt{|f|}} & = \frac{1}{\sqrt{|f|}} \sum_{x \in \operatorname{on}(J)}|x\rangle
      \end{align}
    \item 
    \begin{align}
      \mathrm{QSP}_{f}|0\rangle^{\otimes n} & = \left(\mathrm{QSP}_{f_{\bar{x}_{i}}} \oplus \mathrm{QSP}_{f_{x_{i}}}\right)\left(G\left(p_{f}\left(\bar{x}_{i}\right)\right) \otimes I_{2^{n}-1}\right)|0\rangle
    \end{align}
    \begin{align}
      G(p)|0\rangle & = \sqrt{p}|0\rangle+\sqrt{1-p}|1\rangle
    \end{align}
    \begin{align}
      G\left(p\left(x_{i}\right)\right) & = R_{y}\left(2 \cos ^{-1}\left(\sqrt{p\left(x_{i}\right)}\right)\right)
    \end{align}
  \end{itemize}
\end{frame}
\begin{frame}{initial state}
  \begin{figure}[htbq]
    \centering
    \includegraphics[width=0.5\textwidth]{figure/QSP.png}
    \caption{the general idea of QSP in the quantum  circuit  model for $i=n−1$.} 
    \label{fig-qsp}
  \end{figure}
\end{frame}
\begin{frame}{initial state example}
  \begin{itemize}
    \item for $f(x)=x_0x_1\vee x_1x_2\vee x_2x_0$
    \item 
    \begin{figure}[htbq]
      \centering
      \includegraphics[width=0.5\textwidth]{figure/qsp_circuit.png}
      \caption{the abstract quantum gates of $QSP_{f}$} 
      \label{fig-qsp-example}
    \end{figure}
  \end{itemize}
\end{frame}

\begin{frame}{compiling oracle}
  \begin{itemize}
    \item 
  \end{itemize}
\end{frame}
\begin{frame}{XAG\footfullcite{multiplicative}}
  \begin{itemize}
    \item change a for b
    \item figure
  \end{itemize}
\end{frame}
\begin{frame}{XAG example}
  \begin{itemize}
    \item for
    \item we 
  \end{itemize}
\end{frame}
\begin{frame}{result}
  table   
\end{frame}

\section{tweedledum}
\begin{frame}{induction}  
  ABC 
\end{frame}
\begin{frame}{compilation flow}
  \begin{figure}[htbq]
    \centering
    \includegraphics[width=0.8\textwidth]{figure/work_flow.png}
    \caption{compilation flow overview} 
    \label{fig-compilation}
  \end{figure}
\end{frame}
\begin{frame}{flexibility}
  \begin{figure}[htbq]
    \centering
    \includegraphics[width=0.9\textwidth]{figure/flex.png}
    \caption{tweedledum's IR flexibility} 
    \label{fig-flex}
  \end{figure}
\end{frame}
\begin{frame}{synthesis}
  \begin{itemize}
    \item 
  \end{itemize}  
\end{frame}
\begin{frame}{synthesis}
  \begin{figure}[htbq]
    \centering
    \includegraphics[width=0.7\textwidth]{figure/boolean.png}
    \caption{overview of possible Boolean function synthesis flows} 
    \label{fig-boolean}
  \end{figure}
\end{frame}
\begin{frame}{compilation passes}
  \begin{itemize}
    \item 
  \end{itemize}
\end{frame}
\section*{}
\begin{frame}[noframenumbering,allowframebreaks,t]
	\frametitle{references}
	\printbibliography
\end{frame}
\begin{frame}
\centering
\Huge{END\\Thank you}
\end{frame}
\end{document}