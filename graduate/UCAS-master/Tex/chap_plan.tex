\chapter{研究工作计划与进度安排}
\section{研究计划中完成的部分}
目前已经是十二月下旬,开题半年后,目前已经完成的研究内容如下:
\begin{itemize}
    \item 收集相关资料,完成初期调研,并撰写开题报告。深入研究了关于基于TDD的量子模型检测方法、量子线路的可达性、持续可达性和重复可达性的相关文献和资料,并撰写开题报告,明确研究的目标和方法。
    \item 设计并实现可达空间的计算。根据所掌握的理论知识和之前的调研成果,设计并实现了python 代码,完成了基于TDD的量子模型检测方法中的image computaion部分。通过编程和模拟实验,验证了该方法的有效性,并进行实验结果的分析和讨论。
    \item 优化代码执行效率与性能。为了进一步加快计算过程,将原本用Python编写的代码用C++重构,显著提升了程序的运行效率。此外,我还为这些C++模块提供了Python接口,便于与现有的研究工具和流程集成。这一改进不仅加快了实验的执行速度,而且为后续实验提供了坚实的技术基础,使得更复杂的量子模型检测方法成为可能。
\end{itemize}
\section{未来的进度安排}
中期报告之后,目前的进度安排如下: 
\begin{itemize}
    \item 2024.01至2024.02:进行工具性能测试、优化和评价。在这个阶段,我将对已实现的基于TDD的量子模型检测工具进行性能测试和评价。特别的,考虑在实际场景中应用已完成的工具。根据测试结果,对工具进行必要的优化和改进,以提高其计算效率和准确性。同时,结合实验结果和评估数据,对工具的性能进行客观评价,并提出可能的改进方案。
    \item 2024.03至2024.04:总结研究工作并撰写硕士论文。在这个阶段,我将对整个研究过程进行总结和归纳,梳理研究中的重要发现和创新点。然后,撰写硕士论文,包括引言、相关工作、方法设计、实验结果、分析和讨论等部分,以完整而准确地呈现研究成果。此外,还将对未来可能的研究方向和改进方案进行探讨和展望。
\end{itemize}

