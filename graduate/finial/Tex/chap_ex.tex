\chapter{实验设计与评估}
在上一章节中,详细介绍了TDD表示,如何在量子模型检测中应用TDD,并在最后简单介绍整体软件设计。
本次研究的目的是应用TDD的表示能力,降低量子模型检测的资源消耗,为大规模的量子系统验证提供更实用的工具。
其中image computation,即量子迁移系统的一步映射是模型检测中,反复用的算法。
可以说利用更少的资源进行image computation,就可以降低计算量子系统的可达空间时的资源消耗。
因此本章节的实验主要围绕image computation,并进一步讨论优化方法的参数影响。

在这实验过程中,时间资源可以用计算过程中的用时度量。空间资源方面,由于不同编译语言,不同计算平台之间存在差异,所以用计算过程中TDD节点个数进行度量空间资源。
同时由于最大节点数可以表现实验中内存资源占用最多的时刻。因此实验中通过比较最大节点数讨论空间资源占用。
具体实验方面,\ref{sec-optimize}节中具体讨论了五种进一步的优化方案。分别是addition 和 contraction 两种通过电路拆分进行优化的方案,以及调整TDD索引顺序,以及TDD拆分和用子空间近似表示两种对TDD优化方法。其中调整TDD索引顺序是在实验前进行的,因此这里不再展示。最后本章节中的实验是在Intel Xeon-Gold-5215 CPU,512GB RAM的硬件平台进行的。

\section{电路划分技术}
为了方便比较,本节将首先利用数据绘制图像,然后分析具体趋势。其中蓝色圆点折线始终代表没有电路拆分技术的Basic方法,橙色方块折线始终代表addition 拆分技术方法,绿色三角块折线始终代表contraction 电路拆分技术方法。
图中时间单位均为秒,参数选择上addition优化方法中的参数为$k=1$,addition优化方法中的参数为$k1=k2=4$。
\begin{figure}[htbp]
    \centering
    \begin{subfigure}[b]{.45\textwidth}
        \centering
        \includegraphics[height=4.5cm]{Img/alg_Grover_time.pdf}
        \caption{对Grover 算法应用不同电路拆分技术的时间对比}
        \label{fig:grover-time}
    \end{subfigure}
    \qquad
    \begin{subfigure}[b]{.45\textwidth}
        \centering
        \includegraphics[height=4.5cm]{Img/alg_Grover_node.pdf}
        \caption{对Grover 算法应用不同电路拆分技术的最大节点对比}
        \label{fig:grover-node}
    \end{subfigure}
    
    \caption{对Grover算法运行image computation时不同电路拆分技术的资源对比}
    \label{fig:grover-compare}
\end{figure}



图\ref{fig:grover-compare}表示了对Grover搜索算法运行不同image computation算法的资源对比。可以看到不论是基础算法(basic)和addition 方案在超过20量子比特时都无法在规定时间内完成。而contraction方案,在40个比特以后才不能在超时前完成计算。


图\ref{fig:QFT-compare}表示了对quantum Fourier transform(QFT)算法运行不同image computation算法的资源对比。可以看到不论是基础算法(basic)和addition 方案在超过20量子比特后无法进行计算。而contraction 方法在100比特前始终没有超时,特别是最大节点数相比其他两种方法显著地小。
\begin{figure}[!htbp]
    \centering
    \begin{subfigure}[b]{.45\textwidth}
        \centering
        \includegraphics[height=4.5cm]{Img/alg_QFT_time.pdf}
        \caption{对QFT 算法应用不同电路拆分技术的时间对比}
        \label{fig:QFT-time}
    \end{subfigure}
    \qquad
    \begin{subfigure}[b]{.45\textwidth}
        \centering
        \includegraphics[height=4.5cm]{Img/alg_QFT_node.pdf}
        \caption{对QFT 算法应用不同电路拆分技术的最大节点对比}
        \label{fig:QFT-node}
    \end{subfigure}
    \caption{对QFT算法运行image computation时不同电路拆分技术的资源对比}
    \label{fig:QFT-compare}
\end{figure}

图\ref{fig:BV-compare}表示了对Bernstein–Vazirani(BV)算法运行不同image computation算法的资源对比。尽管三种方法都能在5分钟内计算500量子比特的Bernstein-Vazirani算法的图像。但在该方法中绿色的contraction 方法依然比其他两种方法有优势。
\begin{figure}[!htbp]
    \centering
    \begin{subfigure}[b]{.45\textwidth}
        \centering
        \includegraphics[height=4.5cm]{Img/alg_BV_time.pdf}
        \caption{对BV 算法应用不同电路拆分技术的时间对比}
        \label{fig:BV-time}
    \end{subfigure}
    \qquad
    \begin{subfigure}[b]{.45\textwidth}
        \centering
        \includegraphics[height=4.5cm]{Img/alg_BV_node.pdf}
        \caption{对BV 算法应用不同电路拆分技术的最大节点对比}
        \label{fig:BV-node}
    \end{subfigure}
    \caption{对BV算法运行image computation时不同电路拆分技术的资源对比}
    \label{fig:BV-compare}
\end{figure}

图\ref{fig:GHZ-compare}表示了对Greenberger–Horne–Zeilinger (GHZ)状态制备电路运行不同image computation算法的资源对比。三种方法也都能在4秒钟内完成500量子比特的GHZ算法的图像计算。但是时间方面addition 方法比较好,但和contraction差距之间不明显。最大节点数方面,contraction则同样比addition好,但差距不明显。
\begin{figure}[!htbp]
    \centering
    \begin{subfigure}[b]{.45\textwidth}
        \centering
        \includegraphics[height=4.5cm]{Img/alg_GHZ_time.pdf}
        \caption{对GHZ 算法应用不同电路拆分技术的时间对比}
        \label{fig:GHZ-time}
    \end{subfigure}
    \qquad
    \begin{subfigure}[b]{.45\textwidth}
        \centering
        \includegraphics[height=4.5cm]{Img/alg_GHZ_node.pdf}
        \caption{对GHZ 算法应用不同电路拆分技术的最大节点对比}
        \label{fig:GHZ-node}
    \end{subfigure}
    \caption{对GHZ算法运行image computation时不同电路拆分技术的资源对比}
    \label{fig:GHZ-compare}
\end{figure}

图\ref{fig:QRW-compare}表示了对在$2^n$ 环上的quantum random walk (QRW)算法运行不同image computation算法的资源对比。可以看到基础算法(basic)和addition 方案再次在超过20量子比特后不能计算。而contraction 方法在100比特前始终没有超时,空间资源方面的最大节点数在20比特之后趋于稳定。

\begin{figure}[!htbp]
    \centering
    \begin{subfigure}[b]{.45\textwidth}
        \centering
        \includegraphics[height=4.5cm]{Img/alg_QRW_time.pdf}
        \caption{对QRW 算法应用不同电路拆分技术的时间对比}
        \label{fig:QRW-time}
    \end{subfigure}
    \qquad
    \begin{subfigure}[b]{.45\textwidth}
        \centering
        \includegraphics[height=4.5cm]{Img/alg_QRW_node.pdf}
        \caption{对QRW 算法应用不同电路拆分技术的最大节点对比}
        \label{fig:QRW-node}
    \end{subfigure}
    \caption{对QRW算法运行image computation时不同电路拆分技术的资源对比}
    \label{fig:QRW-compare}
\end{figure}

总的来说,三种不同的电路方案中,addition 电路划分方案法优于基础算法,而contraction 电路划分方案则是最佳选择,其性能大大超过其他两种方法。比如,对于QRW\_20,它仅需要14秒,而addition 电路划分方案法和contraction 电路划分方案分别只需要218秒和341秒。此外,它能够处理远超20量子比特的Grover、QFT和QRW电路。更重要的是,对于QFT、BV、GHZ和QRW,contraction 电路划分方案的TDD最大节点数呈线性增长。

对于contraction 电路划分方案,可以看到,无论在时间还是空间消耗方面,它都比基础算法具有指数级的效率提升。可能的原因是这种算法避免了对整个功能的计算。

而尽管contraction 电路划分方案总是优于addition 电路划分方案,但保留addition 电路划分方案是必要的。首先,注意到addition 电路划分方案对不同电路的性能表现各不相同,这意味着addition 电路划分方案具有其独特效果。
比如GHZ电路中,addition方案的时间资源比较小。特别是对于中心化电路,即在无向图中某些索引的度远大于其他的电路。因此有些电路天然需要使用addition 电路划分方案进行分割。另一方面,addition 电路划分方案可以与contraction 电路划分方案结合使用。
同时对于addition 电路划分方案,由于电路被划分成许多更简单的部分,因此可以使用更小的空间消耗计算每个部分的图像。也就是当空间资源有限时,可以增加参数$k$,将电路切割成更多部分。然后可以逐部分image computation,再将它们相加。还可以使用二级缓存来帮助完成这一方案的最终计算。

表\ref{table:time}给出了三种方法的资源消耗的详细数据。time表示计算TDD收缩的总时间,单位为秒,最大节点数表示计算过程中TDD的节点最大个数。
basic表示没有使用优化技术,addition表示使用\ref{addition}节中的addition优化技术,contraction表示使用\ref{contraction}节中的contraction优化技术。“-”表示超过一小时的运行上限。

\begin{table}[!htbp]
    \centering
    \scalebox{0.8}{
        \begin{tabular}{llllllllll}
            \hline
            \multirow{2}{*}{Benchmark} &  & \multicolumn{2}{c}{basic} &  & \multicolumn{2}{c}{addition优化方案} &  & \multicolumn{2}{c}{contraction优化方案} \\ \cline{3-4} \cline{6-7} \cline{9-10} 
                                       &  & 时间        & 最大节点数       &  & 时间          & 最大节点数        &  & 时间           & 最大节点数          \\ \hline
            Grover\_15 &   & 19.33  & 15785     &   & 17.35      & 15099  & & 1.61 & 597  \\
            Grover\_18 &   & 76.47  & 61694     &   & 66.02      & 60332  & & 2.41 & 516  \\
            Grover\_20 &   & 294.65 & 243946    &   & 259.87     & 241240 & & 4.39  & 1036 \\ 
            Grover\_40 &   & -      &           &   & -          &        & & 2953.57 & 851973 \\
            \hline
            QFT\_15     &  & 34.64   & 65536   &  & 18.88  & 32770   &  & 0.08 & 63  \\
            QFT\_18     &  & 282.12  & 524288  &  & 148.13 & 262146 &   & 0.10  & 31  \\
            QFT\_20     &  & 1199.21 & 2097152 &  & 655.19 & 1048578 &  & 0.12 & 63  \\
            QFT\_30     &  & -       &         &  & -      &        &  & 0.29 & 31  \\
            QFT\_50     &  & -       &         &  & -      &        &  & 1.02 & 51  \\
            QFT\_100    &  & -       &         &  & -      &        &  & 7.14 & 101 \\
            \hline
            BV\_100     &  & 7.36    & 596     &  & 7.43      & 596     &  & 0.41           & 102 \\
            BV\_200     &  & 31.57   & 1196    &  & 30.03     & 1196    &  & 1.70           & 202 \\
            BV\_300     &  & 75.66   & 1796    &  & 75.56     & 1796    &  & 4.28           & 302 \\
            BV\_400     &  & 146.47  & 2396    &  & 145.40    & 2396    &  & 9.18           & 402 \\
            BV\_500     &  & 244.15  & 2996    &  & 223.90    & 2996    &  & 16.31          & 502 \\
            \hline
            GHZ\_100    &  & 0.38    & 595     &  & 0.13      & 301    &  & 0.18           & 200 \\%& 0.03    
            GHZ\_200    &  & 0.72    & 1195    &  & 0.37      & 601    &  & 0.48           & 400 \\%& 0.12     
            GHZ\_300    &  & 1.29    & 1795    &  & 0.62      & 901    &  & 0.80           & 600 \\%& 0.24     
            GHZ\_400    &  & 2.03    & 2395    &  & 1.00      & 1201    &  & 1.26           & 800 \\%& 0.42     
            GHZ\_500    &  & 2.96    & 2995    &  & 1.45      & 1501    &  & 1.72           & 1000\\%& 0.62     
            \hline
            QRW\_15     &  & 36.86   & 13122     &  & 24.59     & 10882     & & 7.16  & 222 \\
            QRW\_18     &  & 139.76  & 90538     &  & 84.69     & 37064     & & 11.23 & 226 \\
            QRW\_20     &  & 341.05  & 265614    &  & 218.29    & 107714    & & 14.31 & 404 \\
            QRW\_30     &   &-       &          &  &-          &          & & 36.82 & 404 \\
            QRW\_50     &   &-       &          &  &-          &          & & 118.08 & 404 \\
            QRW\_100    &   &-       &          &  &-          &          & & 692.08 & 436 \\
            \hline
        \end{tabular}
    }
    \caption{对不同测试实验应用image computation}
    \label{table:time}
\end{table}
\section{电路划分技术的参数选择}

本节中同样以Grover\_15电路为例,研究不同参数电路划分方案性能的影响。
首先讨论参数$k$对addition 电路划分方案性能的影响。使用参数$k$从1到13的addition 电路划分方案image computation。表\ref{table:addition}为具体数据。根据表\ref{table:addition}绘制折线图\ref{fig:addition-ex}。从折线图中,可以看出,当参数$k$小于五时,总时间不会发生显著变化。但当参数$k$增大时,时间将会指数级增长。这是因为随着$k$的增长,将电路划分成$2^k$部分。这也说明,通常不应该将电路划分成太多部分,只有在存在明显中心时才进行划分。
\begin{figure}
    \centering
    \includegraphics[width=.8\textwidth]{Img/addition_para.pdf}
    \caption{不同参数$k$对Grover\_15电路的additon划分方案的时间影响}
    \label{fig:addition-ex}
\end{figure}
\begin{table}[htbp]
    \centering
    \scalebox{0.8}{
        \begin{tabular}{cccccccccccccc} % Corrected to 14 cs for 14 columns
            \toprule
            k  & 1 & 2 & 3 & 4 & 5 & 6 & 7 & 8 & 9 & 10 & 11 & 12 & 13 \\
            \midrule
            时间 & 219 & 225 & 220 & 218 & 363 & 570 & 621 & 728 & 1298 & 1982 & 2868 & 4680 & 8615 \\
            \bottomrule
            \end{tabular}
    }
    \caption{对grover\_15应用不同的addition参数的时间对比。}
    \label{table:addition}
\end{table}
以Grover\_15电路为例,继续研究参数$k_1$和$k_2$对contraction 电路划分方案性能的影响。表\ref{table-contraction}展示了对Grover\_15应用不同的contraction参数的时间比较,单位为秒。其中表\ref{table-contraction}中蓝色表示用时为用时超过10秒的参数,紫色表示用时小于等于2秒。深紫色表示用时超过100秒,即特别长的参数对。深紫色表示用时小于等于1.5秒,表示用时特别少的参数对。

从表\ref{table-contraction}中的颜色可以看出,时间比较长的蓝色部分主要集中在图下半部分。这说明只要不把参数$k_1$设置得太大,contraction 电路划分方案方法就会很有效。这意味着对于参数有很宽的选择范围。

同时从表格中还可以看到,当将contraction 电路划分方案的参数$k_1$设置在2到9之间,$k_2$设置在2到6之间时,时间基本都是比较短的紫色。这意味着通常最好将参数设置为适中的值。


\begin{table}[!htbp]
    \centering
    \scalebox{0.8}{
        \begin{tabular}{c|ccccccccccccccc}
            \rowcolor[HTML]{FFFFFF} 
            \DiagonalCell{k1}{k2}                         & 1                           & 2                           & 3                           & 4                           & 5                           & 6                           & 7                          & 8                           & 9                           & 10                          & 11                          & 12                          & 13                          & 14                          & 15                          \\\hline
                \rowcolor[HTML]{FFFFFF} 
        1                          & 2.8                                              & 2.2                         & 2.1                         & \cellcolor[HTML]{CCC0DA}2.0 & \cellcolor[HTML]{CCC0DA}1.9 & \cellcolor[HTML]{CCC0DA}2.0                      & 2.1                        & \cellcolor[HTML]{CCC0DA}2.0 & 2.1                         & \cellcolor[HTML]{CCC0DA}2.0 & \cellcolor[HTML]{CCC0DA}2.0 & 2.1                         & 2.2                         & 2.1                         & 2.1                         \\ \cline{3-7}
        \rowcolor[HTML]{CCC0DA} 
        
        \rowcolor[HTML]{FFFFFF} 
        3                          & \multicolumn{1}{l|}{\cellcolor[HTML]{FFFFFF}2.2} & \cellcolor[HTML]{CCC0DA}1.9 & \cellcolor[HTML]{CCC0DA}1.8 & \cellcolor[HTML]{CCC0DA}1.6 & \cellcolor[HTML]{CCC0DA}2.0 & \multicolumn{1}{l|}{\cellcolor[HTML]{CCC0DA}1.9} & 2.1                        & 2.1                         & 2.5                         & 2.3                         & 2.7                         & 2.3                         & 3.1                         & 2.8                         & 3.3                         \\
        \rowcolor[HTML]{FFFFFF} 
        4                          & \multicolumn{1}{l|}{\cellcolor[HTML]{FFFFFF}2.3} & \cellcolor[HTML]{CCC0DA}1.8 & \cellcolor[HTML]{CCC0DA}2.0 & \cellcolor[HTML]{CCC0DA}1.7 & \cellcolor[HTML]{CCC0DA}2.0 & \multicolumn{1}{l|}{\cellcolor[HTML]{FFFFFF}2.1} & 2.2                        & 2.1                         & 2.6                         & 2.3                         & 2.8                         & 2.7                         & 3.3                         & 3.0                         & 3.3                         \\
        \rowcolor[HTML]{FFFFFF} 
        5                          & \multicolumn{1}{l|}{\cellcolor[HTML]{FFFFFF}2.2} & \cellcolor[HTML]{CCC0DA}1.7 & \cellcolor[HTML]{CCC0DA}1.9 & \cellcolor[HTML]{CCC0DA}1.6 & \cellcolor[HTML]{CCC0DA}1.9 & \multicolumn{1}{l|}{\cellcolor[HTML]{CCC0DA}2.0} & 2.3                        & \cellcolor[HTML]{CCC0DA}1.9 & 2.5                         & 2.3                         & 2.8                         & 2.7                         & 3.4                         & 3.0                         & 3.6                         \\
        \rowcolor[HTML]{FFFFFF} 
        6                          & \multicolumn{1}{l|}{\cellcolor[HTML]{FFFFFF}2.1} & \cellcolor[HTML]{B1A0C7}1.5 & \cellcolor[HTML]{CCC0DA}1.8 & \cellcolor[HTML]{CCC0DA}1.7 & 2.2                         & \multicolumn{1}{l|}{\cellcolor[HTML]{CCC0DA}1.9} & 2.5                        & 2.2                         & 2.9                         & 2.8                         & 3.1                         & 2.9                         & 3.7                         & 3.7                         & 4.2                         \\
        \rowcolor[HTML]{FFFFFF} 
        7                          & \multicolumn{1}{l|}{\cellcolor[HTML]{FFFFFF}2.1} & \cellcolor[HTML]{CCC0DA}1.5 & \cellcolor[HTML]{CCC0DA}1.9 & \cellcolor[HTML]{CCC0DA}1.6 & 2.2                         & \multicolumn{1}{l|}{\cellcolor[HTML]{CCC0DA}1.9} & 2.5                        & 2.2                         & 2.8                         & 3.0                         & 3.6                         & 3.3                       & 4.2                         & 5.7                         & 5.0                         \\
        \rowcolor[HTML]{FFFFFF} 
        8                          & \multicolumn{1}{l|}{\cellcolor[HTML]{CCC0DA}2.0} & \cellcolor[HTML]{CCC0DA}1.7 & \cellcolor[HTML]{CCC0DA}1.8 & \cellcolor[HTML]{CCC0DA}1.7 & 2.1                         & \multicolumn{1}{l|}{\cellcolor[HTML]{CCC0DA}2.0} & 2.4                        & 2.2                         & 2.8                         & 2.8                         & 3.7                         & 3.4                         & 4.3                         & 4.8                         & 5.2                         \\
        \rowcolor[HTML]{FFFFFF} 
        9                          & \multicolumn{1}{l|}{\cellcolor[HTML]{FFFFFF}2.1} & \cellcolor[HTML]{B1A0C7}1.5 & \cellcolor[HTML]{CCC0DA}2.0 & \cellcolor[HTML]{B1A0C7}1.4 & 2.2                         & \multicolumn{1}{l|}{\cellcolor[HTML]{CCC0DA}2.0} & 2.5                        & \cellcolor[HTML]{CCC0DA}2.0 & 3.3                         & 2.9                         & 3.7                         & 3.5                         & 4.9                         & 4.7                         & 5.8                         \\ \cline{3-7}
        \rowcolor[HTML]{FFFFFF} 
        10                         & 2.3                                              & \cellcolor[HTML]{CCC0DA}1.9 & 2.3                         & \cellcolor[HTML]{CCC0DA}1.6 & 2.6                         & 2.7                                              & 3.1                        & 2.2                         & 4.0                         & 3.6                         & 4.6                         & 3.9                         & 5.6                         & 5.2                         & 7.5                         \\
        \rowcolor[HTML]{FFFFFF} 
        11                         & 3.2                                              & 3.2                         & 3.5                         & 3.1                         & 4.7                         & 4.2                                              & 5.6                        & 4.2                         & 6.8                         & 7.2                         & 7.6                         & 6.3                         & 9.0                         & 8.1                         & \cellcolor[HTML]{8DB4E2}11  \\
        \rowcolor[HTML]{FFFFFF} 
        12                         & 5.6                                              & 6.0                         & 7.2                         & 6.0                         & 8.3                         & 9.0                                              & 8.9                        & 7.8                         & \cellcolor[HTML]{8DB4E2}11  & \cellcolor[HTML]{8DB4E2}11  & \cellcolor[HTML]{8DB4E2}12  & \cellcolor[HTML]{8DB4E2}11  & \cellcolor[HTML]{8DB4E2}12  & \cellcolor[HTML]{8DB4E2}15  & \cellcolor[HTML]{8DB4E2}16  \\
        \rowcolor[HTML]{8DB4E2} 
        \cellcolor[HTML]{FFFFFF}13 & 11                                               & 12                          & 14                          & 12                          & 15                          & 18                                               & 18                         & 15                          & 18                          & 20                          & 18                          & 32                          & 32                          & 30                          & 25                          \\
        \rowcolor[HTML]{8DB4E2} 
        \cellcolor[HTML]{FFFFFF}14 & 20                                               & 21                          & 24                          & 32                          & 31                          & 44                                               & 77                         & 50                          & 86                          & \cellcolor[HTML]{538DD5}109 & 68                          & \cellcolor[HTML]{538DD5}133 & 70                          & \cellcolor[HTML]{538DD5}119 & \cellcolor[HTML]{538DD5}142 \\
        \rowcolor[HTML]{538DD5} 
        \cellcolor[HTML]{FFFFFF}15 & \cellcolor[HTML]{8DB4E2}28                       & \cellcolor[HTML]{8DB4E2}30  & \cellcolor[HTML]{8DB4E2}31  & \cellcolor[HTML]{8DB4E2}53  & \cellcolor[HTML]{8DB4E2}69  & 111                                              & \cellcolor[HTML]{8DB4E2}85 & \cellcolor[HTML]{8DB4E2}81  & 102                         & 153                         & 114                         & 130                         & 166                         & 162                         & 235                        
                         
        \end{tabular}
    }
    \caption{对grover\_15应用不同的contration参数的时间对比。蓝色越深,时间越长;紫色越深,时间越短。}%calculating the 
    \label{table-contraction}
\end{table}

\section{对TDD结构的优化}





    最后为了验证基于TDD优化方法,即对TDD的分割和近似方法在降低image computation过程中内存占用以及缓解内存溢出问题上的效果。实验中,选取了Grover\_40和QFT\_100作为测试案例,分别采用了TDD的分割和子空间近似的方法进行image computation,并记录了TDD的处理时间及其最大节点数量。实验成果汇编于表\ref{table:tdd-based}。表中,变量 $k$ 用于指示采用的分割次数。当 $k=1$时,将原TDD一分为二,形成两个较为均衡的部分,从而显著减少了节点数量。$k=2$时,进一步将一个分支细分为两个小部分。而到了 $k=3$,在另一个分支上继续分割,节点数因此再次大幅度降低。

对于Grover\_40电路,初始使用子空间 $S=span{\ket{0\cdots0}}$ 以简化过程。从表\ref{table:tdd-based}可见,在不进行image computation优化的情况下,最大的TDD节点数达到了589,865。通过在达到最大TDD之前实施基于TDD的分割,可以将最大节点数减至393,423。若对两个分支均分割TDD,则最大节点数可进一步降至245,814,不足原数的一半。此外,这一过程几乎不会增加时间消耗。
    
但有时,计算末尾才会出现最大的TDD。这意味着,即便在整个计算过程中已将TDD分割为多个小片段,最大的TDD仍难以避免,除非采取近似方法。这里以初始子空间为 $S=span\{\ket{+\cdots+ -\cdots -}\}$(最后20个量子比特全为 $\ket{-}$)的QFT\_100为例,计算结束时出现了含有524,369节点的最大TDD。因此,虽然会在计算过程中通过TDD分割获得多个小TDD,但在最后不会将它们合并。相反,用这些TDD张成的子空间来近似原始TDD。进而,通过将原来的一维子空间近似为二维子空间,可以将最大TDD的节点数降至262,226;将原空间近似为四维时,节点数可降至131,155。在整个近似过程中,时间消耗同样不会显著增加。值得注意的是,实际验证时无需将这些小TDD同时存储于内存中,而可以逐一进行计算与验证。

\begin{table}[]
    \centering
    \scalebox{0.8}{
    \begin{tabular}{c|c|ccccc}
                        电路  & 优化方法 & & $k=0$ & $k=1$ & $k=2$ & $k=3$ \\\hline
    % \multirow{2}{*}{Grover\_20} & 时间   & 3.10 &3.36 &3.39 &3.37   \\
    %                         &max\_node &1178  &793  &780  &513   \\\hline
    \multirow{2}{*}{Grover\_40} & \multirow{2}{*}{TDD分割}    &时间       & 1,510.42   &1,519.24 & 1,459.02 & 1,495.20  \\
                           & &最大节点个数     &589,865     & 393,423 & 393,239 & 245,814  \\\hline
    % \multirow{2}{*}{QFT}    &时间       & 0.45   &0.30 &0.37 & 0.40  \\
    %                         &approx     &515     & 259 & 259 & 132  \\\hline
    \multirow{2}{*}{QFT\_100}  & \multirow{2}{*}{子空间近似}    &时间       &121.28    & 118.78 & 116.69 & 128.31\\
                              &   & 最大节点个数     &524,369     & 262,226 & 262,226 & 131,155\\
    % \multirow{2}{*}{GHZ\_500}    &时间       & 1.72   & & &  \\
    %                         &approx     &1000     & 501 & 501 & 251  \\\hline
    
    \end{tabular}
    }
    \caption{TDD拆分与近似的优化方案}

    \label{table:tdd-based}
\end{table}

\section{本章小结}
本章主要介绍了几种优化image computation算法的方法,并进行了实验对比。主要优化方法包括电路拆分(addition和contraction)和对TDD结构的优化(拆分和子空间近似)。

实验结果表明,contraction电路拆分方法在降低时间和空间资源消耗方面效果最佳,能够显著提高算法效率,尤其对线路层数较深的量子算法很有优势。addition电路拆分方法次之,对某些特殊电路也有一定优化效果。

在电路拆分方法中,参数的选择很关键。对addition方法,参数k不宜过大;对contraction方法,参数k1、k2适中值范围较宽,通常取2-9和2-6较为合适。

另一方面,针对TDD结构进行拆分和子空间近似也能有效降低内存占用,避免内存溢出,且时间开销较小。尤其是当出现大的最终TDD时,子空间近似方法很有帮助。

总的来说,上述优化方法能够有效降低image computation过程中的时间和空间资源消耗,提高量子模型检测的可扩展性,为大规模量子系统验证提供实用的工具。