

\section{本章小结}
本章主要介绍了几种优化image computation算法的方法,并进行了实验对比。主要优化方法包括电路拆分(addition和contraction)和对TDD结构的优化(拆分和子空间近似)。

实验结果表明,contraction电路拆分方法在降低时间和空间资源消耗方面效果最佳,能够显著提高算法效率,尤其对线路层数较深的量子算法很有优势。addition电路拆分方法次之,对某些特殊电路也有一定优化效果。

在电路拆分方法中,参数的选择很关键。对addition方法,参数k不宜过大;对contraction方法,参数k1、k2适中值范围较宽,通常取2-9和2-6较为合适。

另一方面,针对TDD结构进行拆分和子空间近似也能有效降低内存占用,避免内存溢出,且时间开销较小。尤其是当出现大的最终TDD时,子空间近似方法很有帮助。

总的来说,上述优化方法能够有效降低image computation过程中的时间和空间资源消耗,提高量子模型检测的可扩展性,为大规模量子系统验证提供实用的工具。