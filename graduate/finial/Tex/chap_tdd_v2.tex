\chapter{在量子模型检测中应用TDD表示}

张量决策图(TDD)是一种基于张量网络,结合了二元决策图的表示优势的数据结构。
TDD针对于量子计算,可以有很好的空间优势,同时也可以很方便的结合过去在二元决策图的模型检测算法的优化技术。
本章将主要介绍在本文工作中使用的方法,因此将主要介绍如何应用TDD进行模型检测并作出针对性改进,最后介绍本文工作中的软件实现。

\section{将量子系统建模为量子迁移系统}

在\ref{sec-transition}节中介绍了跃迁系统,其中定义\ref{def-model-q}给出了量子跃迁系统的简单定义。这里简单回顾一下。对于一个希尔伯特空间$\mathcal{H}$ 。基于 $\mathcal{H}$ 的量子转移系统 $\mathcal{M}$ 可以表述为四元组 $(\mathcal{H}, S, \Sigma, \mathcal{T})$,这里 $S$ 作为 $\mathcal{H}$ 的一个封闭子空间,被定义为初始空间。$\Sigma=\{\sigma_1,\ldots,\sigma_m\}$ 是一系列符号集合,$\mathcal{T}=\{\mathcal{T}_\sigma, \sigma \in \Sigma\}$ 则代表对 $\mathcal{H}$ 执行的一组量子操作。


\begin{example}
    一个单量子比特的系统可能遭受两种潜在的噪声影响:比特位翻转和相位翻转。如果不能准确判断会发生哪一种噪声,这样的系统可以被表达为一个量子转移系统 $\mathcal{M}=\big(\mathcal{H}_2,S,\{1,2\},\{\mathcal{T}_1,\mathcal{T}_2\} \big)$,这里 $S$ 代表 $\mathcal{H}$ 中的一个子空间,$\mathcal{T}_1=\{\sqrt{p}I, \sqrt{1-p}X\}$ 和 $\mathcal{T}_2=\{\sqrt{p}I, \sqrt{1-p}Z\}$。 $I,X,Z$ 分别代表恒等矩阵、Pauli $X$ 矩阵和Pauli $Z$ 矩阵。
\end{example}
\begin{example}
    \label{ex-image-grover}
    量子线路也可以表达为一个量子迁移系统。图\ref{fig:grover}展示了实现两量子比特 Grover 迭代的线路 \citep{Grover_1996},这是 Grover 算法的一个基本过程,用于找到布尔函数 $f(x)=1$ 的解。该类算法需要验证的属性是能否始终进入布尔函数解所在的子空间。
    对于该线路,第一个 CCX 门表示搜索用的oracle,实现了$ O\ket{x}\ket{y}=\ket{x}\ket{f(x)\oplus y}$,其中 $f(x)=x_1 \wedge x_2$。因此,该线路oracle的解是\(11\) 。其他门实现了一个补$2\ket{\psi}\bra{\psi}-I$,其中$\ket{\psi}=\frac{1}{\sqrt{2}^{n}}\sum_{i=0}^{2^n-1}\ket{i}$。给定输入状态 $\ket{++-}=\frac{1}{2}\sum_{i=0}^3\ket{i}\ket{-}$,线路首先将状态变为 $\frac{1}{2}\sum_{i=0}^2 \ket{i}\ket{-}-\frac{1}{2}\ket{11}\ket{-}$,然后变为 $\ket{11}\ket{-}$。对于状态空间 $S=span\{\ket{++-},\ket{11-}\}$。当目前系统状态 $\ket{\varphi} \in S$,下一步系统状态总会在 $S$ 中。

    因此该系统可以表示为量子迁移系统 $(\mathcal{H}_8, S, {1}, \mathcal{T})$ 进行建模,其中 $\mathcal{T}_1 = {(2\ket{\psi}\bra{\psi}-I) O}$,需要检查的属性是$\mathcal{T}_1(S)=S$。
    \begin{figure}[!htbp]
        \centering
        \includegraphics[height=4cm]{Img/cir_grover.pdf}
        \caption{Grover\_3的量子线路图}
        \label{fig:grover}
    \end{figure} 
\end{example}
\section{基于TDD表示的子空间算法}
\subsection{子空间的分解}
在\ref{sec-logic}节中介绍了量子逻辑中的原子命题用希尔伯特空间中的子空间表示。
因此在量子模型检测中,需要用到子空间的表示。而一种比较方便的表示子空间的办法是将子空间表示为一组基态。
给定子空间投影算子的矩阵形式,如果有一个非零的列向量,那么在该向量的方向上就可以找到一个基向量。然后,通过正交化过程,可以递归地找到原始子空间的一组正交基向量。


但是使用矩阵表示进行所有列的遍历会遇到很高的复杂度。而通过使用子空间投影算子的TDD表示,就可以可以容易地找到第一个非零列。
具体方法是寻找该TDD表示的最左侧非零路径。
通过这种图的方法避免了在计算过程中需要显式表示相应的向量,也减少了0复杂度​​。
综上所述,算法\ref{alg-basis_dec}给出了一个计算子空间的一组计算基的方法。在此算法基础上,可以给出根据当前系统状态计算下一步系统状态的算法\ref{alg-image}.
\begin{algorithm}
\caption{给出投影算子$P$的一组正交基}
\label{alg-basis_dec} 
\begin{algorithmic}[1]
    \Require 子空间$S$的TDD形式投影算子$P=P_S$ 
    \Ensure $P$的一组正交基分解$B$
    \State $B\gets \{\}$
    \If{\(P\) 是空的} 
        \Return \(B\)
    \Else
        \State \(\ket{i} \gets\) TDD表示\(P\)中最左侧非零路径所表示的列号\(i\)
        \State \(\ket{u_i}\gets\) 由在TDD表示\(P\) 中具有列号 \(\ket{i}\) 的路径表示的状态
        \State \(\ket{v_i} \gets \frac{\ket{u_i}}{\|\ket{u_i}\|}\)
        \State \(P \gets P - \ket{v_i}\bra{v_i}\)
        \State \(B' \gets \) 递归调用本算法,得到新的$P$的计算基分解
        \State \(B \gets B \cup \{\ket{v_i}\} \cup B'\)
    \EndIf
    \State \Return \(B\)
\end{algorithmic}
\end{algorithm}

\begin{example}
    \label{ex-image-sub}
    例子\ref{ex-image-grover}将Grover算法的线路建模为量子迁移系统。其中用到了子空间$S=span\{\ket{++-},\ket{11-}\}$。相应投影算子 $P$ 的矩阵和 TDD 表示如图 \ref{fig:P} 所示。
    根据该TDD表示求解原子空间的基的过程如下:
    \begin{enumerate}
        \item 应用图算法,得到最左侧路径对应于 $(x_1,x_2,x_3,y_1,y_2,y_3)=(0,0,0,0,0,0)$,这意味着投影算子的第一列非零。
        \item 遍历所有 $(x_0,x_1,x_2)=(0,0,0)$ 的路径,得到向量 $\ket{v_1}=\frac{1}{6}[1,-1,1,-1,1,-1,0,0]$。
        \item $\ket{v_1}$标准化为 $\ket{v_1}=\frac{1}{\sqrt{3}}(\ket{00}+\ket{01}+\ket{10})\ket{-}$。
        \item 设 $P'=P-\ket{v_1}\bra{v_1}$。那么 $P'$ 等于 $\ket{11-}\bra{11-}$。
        \item 对 $P'$ 重复上述的过程,获得 另一组基$\ket{v_2}=\ket{11}\ket{-}$,此时TDD为空。因此 ${\ket{v_1},\ket{v_2}}$ 是子空间 $S$ 的一个基。
    \end{enumerate}
\end{example}
\subsection{子空间的并}
在\ref{sec-connect}节中介绍了量子逻辑中的连接词。
其中子空间之间的包含理解为量子逻辑的蕴含;子空间 的正交补理解为量子逻辑的否定;子空间的交理解为量子逻辑的合取;些子空间的并理解为量子逻辑的析取。其中最复杂的是子空间的并。本小节主要讨论在TDD表示下子空间的并。

设 $S=S_1\vee S_2$ 为两个子空间 $S_1,S_2$ 的并集。假设 $B_1=\{\ket{\psi_{11}},\cdots,\ket{\psi_{1k}}\}$ 和 $B_2=\{\ket{\psi_{21}},\cdots,\ket{\psi_{2l}}\}$ 分别是 $S_1$ 和 $S_2$ 的一组正交基。子空间$S$的一组正交基$B$可以通过格拉姆-施密特正交化方法(Gram–Schmidt process)计算而来,具体方法如下。
\begin{enumerate}
    \item 令 $B=B_1$,并定义 $P=\sum_{j=1}^{k}{\ket{\psi_{1j}}\bra{\psi_{1j}}}$。
    \item 遍历$B_2$ 中的基向量。假设当前向量为 $\ket{\psi_{2j}}$。计算 $\ket{u_j}=\ket{\psi_{2j}}-P\ket{\psi_{2j}}$ 并将其标准化为 $\ket{v_j}=\frac{\ket{u_j}}{|\ket{u_j}|}$。
    \item 如果 $\ket{v_j}$ 为 0,则考虑下一个向量;否则,此时它与 $P$ 正交,将其添加到 $B$ 中。同时,还将 $P$ 更新为 $P+\ket{v_j}\bra{v_j}$。
    \item 重复上述过程直到遍历完 $B_2$ 中的所有元素。此时$B$ 将成为 $S$ 的一个基,$P$ 将成为到 $S$ 的投影算子。
\end{enumerate}

这个过程展示了如何从一组生成向量开始,通过计算和正交化过程,构建出一个复合空间的正交归一化基。在量子计算和量子信息中,这种方法特别有用,因为它允许准确地描述和操控量子态的子空间。

\begin{example}
    例子\ref{ex-image-grover}中对Grover 3线路建模,例子\ref{ex-image-sub}对涉及的子空间进行了分解。考虑分解的逆过程,设 $P_1,P_2$ 为两个一维子空间 $S_1,S_2$ 的投影算子,分别由 $B_1=\{\ket{++-}\}$ 和 $B_2=\{\ket{11-}\}$ 生成。显然,$P_1=\ket{++-}\bra{++-}$。将 $B_1$ 完善为 $S=S_1\vee S_2$ 的一个基。计算得到$\ket{u}=\ket{11-}-P_1\ket{11-}=\ket{11-}-\frac{1}{4}\ket{++-}=[-\frac{1}{4},\frac{1}{4},-\frac{1}{4},\frac{1}{4},-\frac{1}{4},\frac{1}{4},\frac{3}{4},-\frac{3}{4}]^T$,可以被标准化为 $\ket{v}=-\frac{1}{2\sqrt{3}}(\ket{00}+\ket{01}+\ket{10}-3\ket{11})\ket{-}$。那么 $B=\{\ket{++-},\ket{v}\}$ 是一个正交归一化基,且 $P=P_1+\ket{v}\bra{v}$ 是相应的投影算子。
\end{example}
\section{计算一步迁移}
结合子空间的分解算法,可以给出对于量子系统的一步映射算法,具体如算法\ref{alg-image}所示。
\begin{algorithm}
\caption{基于迁移系统的一步映射算法}
\label{alg-image}
\begin{algorithmic}[1] % The optional argument [1] enables line numbering
\Require 一个量子迁移系统 $(\mathcal{H},S,\Sigma,\mathcal{T})$, 其中转移关系$\mathcal{T}=\{\mathcal{T}_\sigma\mid \sigma\in \Sigma\}$,  $\mathcal{T}_\sigma=\{E_{\sigma,j_\sigma}\}$
\Ensure 系统下一步状态$\mathcal{T}(S)$的投影算子$P$
\State $P \gets 0$ 
\State $B \gets $调用算法\ref{alg-basis_dec},得到$S$的计算基分解
\State $K \gets \cup_{\sigma,j_\sigma}\{E_{\sigma,j_\sigma}\}$
\For{$\ket{\psi}$ \textbf{在} $B$, $E$ \textbf{在} $K$}
    \State $\ket{\phi} \gets $收缩\(\ket{\psi}\)和\(E\)所有相同索引
    \State $P=P \vee \text{span}\{\ket{\phi}\}$
\EndFor
\State \Return $P$ 
\end{algorithmic}
\end{algorithm}

在此基础上,可以计算该量子迁移系统$(\mathcal{H},S,\Sigma,\mathcal{T})$的可达空间。具体方法如下:
\begin{enumerate}
    \item 初始化可达空间为\(R = S\),并计算可达空间的投影算子\(P_{R}\)。
    \item 调用算法\ref{alg-image},计算系统$(\mathcal{H},S,\Sigma,\mathcal{T})$下一步状态\(S'=\mathcal{T}(S)\)。
    \item \label{it-be}调用算法\ref{alg-basis_dec},分解子空间得到\(S'\)的一组基\(B = \{\ket{\psi_{1}},\cdots,\ket{\psi_{n}}\}\)。同时初始化一个空的状态空间$S''$。
    \item 遍历\(B\)中的基,假设当前向量为\(\ket{\psi_{i}}\),计算得到\(\ket{u_i}=P\ket{\psi_{i}}\),并标准化为$\ket{v_j}=\frac{\ket{u_j}}{|\ket{u_j}|}$。
    \item \label{it-end}当$\ket{v_j}$为0时,考虑下一个向量;否则,此时$\ket{v_j}$与P正交,将该向量张开的空间并到子空间$S''$中,即$S''=S''\vee span\{\ket{v_j}\}$。同时更新可达空间$R=R\vee span\{\ket{v_j}\}$,即将 $P$ 更新为 $P+\ket{v_j}\bra{v_j}$。
    \item 遍历结束后,如果$S''$为空,说明该量子系统的可达空间已经收敛,即此时的$R$就是系统的可达空间。否则用$S''$作为量子系统的初态,计算迁移系统$(\mathcal{H},S'',\Sigma,\mathcal{T})$的下一步状态,并重复上述\ref{it-be}到\ref{it-end}的过程。
\end{enumerate}

\section{针对模型检测的改进}
\label{sec-optimize}
本文工作的主要目的是借助TDD数据结构,构建能快速计算量子模型检测中可达问题的方案。本文工作的主要挑战在于尽可能减少程序的运行时间以及空间资源。为此,需要采用一系列方法来开发更有效的算法,以优化TDD操作和收缩张量网络。其中包括开发新技术来分割线路和优化TDD结构。下面简单介绍一下具体改进方法。
\subsection*{索引顺序的调整}
\label{contraction}在BDD中,索引的顺序很重要。因为索引顺序会直接影响BDD的大小。一个良好的变量顺序可以使得BDD比一个糟糕的变量顺序小得多。图\ref{fig:bdd-compare}的了两张图都表示了布尔函数$f (x1,x2,x3,x4)=x1x2+x3x4$。但图\ref{fig:bdd-good}的结构更简单,图中边和节点的数目更少。其中图\ref{fig:bdd-bad}的索引顺序为\{x1,x3,x2,x4\},图\ref{fig:bdd-good}的索引顺序为\{x1,x2,x3,x4\}。找到一个好的索引顺序是一个NP问题。在工程实现中,目前只能通过小规模线路上寻求规律,然后在更大规模线路中应用较优顺序。
% \textcolor{red}{目前也有一些研究借助xxx机器学习算法,寻找收缩的最佳序列\citep{zhang2024quantum}}
\begin{figure}[!htbp]
	\centering
	\begin{subfigure}[b]{.4\textwidth}
        \centering
        \includegraphics[height=8cm]{Img/bdd_bad.pdf}
		\caption{索引顺序为\{x1,x3,x2,x4\}}
		\label{fig:bdd-bad}
	\end{subfigure}
	\begin{subfigure}[b]{.4\textwidth}
        \centering
        \includegraphics[height=8cm]{Img/bdd_good.pdf}
		\caption{索引顺序为\{x1,x2,x3,x4\}}
		\label{fig:bdd-good}
	\end{subfigure}
	\caption{布尔函数$f (x1,x2,x3,x4)=x1x2+x3x4$在不同索引顺序下的BDD}
	\label{fig:bdd-compare}
\end{figure}

\subsection*{addition的线路拆分方案}
\label{addition}关于常用的量子线路划分方法,
第一种被称为addition\citep{chen2018classical}。将量子线路视为张量网络,首先将一个量子线路C转换成无向图G。G中的每个节点表示量子线路的一个索引,并且如果它们是相同门的输入或输出索引,则在G中连接两个节点。并且当满足以下两个条件之一时输入和输出索引不变:
\begin{itemize}
	\item 是对角线量子门的输入和输出索引;
	\item 是受控门的控制比特位的输入和输出索引。
\end{itemize}
在得到无向图后,就可以通过对连通度高索引进行张量切片,从而对量子线路进行划分。 
\begin{example}
    图\ref{fig:addition}展示了图\ref{fig:grover}中Grover\_3线路图的索引链接图。该图描述了量子线路的连通性,通过选择图中连通度最大的索引可以对线路进行分割。因此选择图中连通度较大的$x_1^1,x_1^3x_2^1$可以对线路进行较好的划分。
 
\begin{figure}[!htbp]
	\centering
	\includegraphics[height=6cm]{Img/cir_index_graph.pdf}
	\caption{Grover\_3的索引连接图}
	\label{fig:addition}
\end{figure} 
\end{example}

\subsection*{contraction的线路拆分方案}
另一种常用的线路划分方法成为contraction。该方法是\ref{sec-compare}小节中的TDD part I的延申。在这一方法中,将量子线路划分为若干个较小的部分,其收缩等于原始线路。应用两个预设整数参数$k1$和$k2$,将线路划分为若干小线路。其中每个小线路涉及最多$k1$个量子比特,并且与至多跨越不同部件的$k2$个多比特门相连。
\begin{example}
    图\ref{fig:contraction}展示了对Bit flip线路进行k1=3,k2=2的拆分结果。
\begin{figure}[!htbp]
	\centering
	\includegraphics[height=7cm]{Img/cir_contraction.pdf}
	\caption{对Bit flip线路进行contraction的拆分}
	\label{fig:contraction}
\end{figure} 
\end{example}



\subsection*{基于窗函数对TDD分割}
按照 \citep{narayan1996partitioned} 中关于经典模型检测方法的讨论,可以设计基于TDD的划分法。即利用布尔函数将一个大张量细分成若干个小张量的优化方法。设 $\varphi$ 是一个含有 $n$ 个索引的张量。从 $\set{0,1}^n$ 到 $\set{0,1}$ 的集合中选取一系列窗口布尔函数 (window boolean function)$\{w_1,\cdots, w_k\}$,这些函数对于同一输入,始终满足以下条件: 
\begin{itemize}
    \item $w_1+\cdots +w_k=1$
    \item 对任意 $i \neq j$,$w_i \cdot w_j = 0$
\end{itemize} 
因此可以看到对于一个索引$a$,有且只有一个窗口函数$w_i (a) = 1$,其他窗口函数都为$0$。
对一个张量$\varphi$,令$\varphi_i=\varphi \cdot w_i$。此时如果 $w_i(a)=1$,则$\varphi(a)=\varphi_i(a)$ ;而当$w_i(a)=0$,则$\varphi_i(a)=0$,此时存在另一个窗口函数使得$w_j(a)=1$,$\varphi(a)=\varphi_j(a)$ 。
进而,可以得出 $\varphi = \varphi_1+ \cdots +\varphi_k$。基于这样的张量划分方法,也可以对TDD进行划分。

\begin{example}
    以例子\ref{ex-image-sub}中的
    量子态 $\ket{v_1}=\frac{1}{\sqrt{3}}(\ket{00}+\ket{01}+\ket{10})\ket{-}=\frac{\sqrt{2}}{\sqrt{3}}\ket{0}\ket{+}\ket{-}+\frac{1}{\sqrt{3}}\ket{1}\ket{0}\ket{-}$ 作为示例。将 $q_0$ 量子比特看作是一个布尔变量,设$w_1=\bar{q_0}$ 和 $w_2=q_0$。窗口函数$w_1+w_2=1$,并且互斥。 $\ket{v_1}\cdot w_1$ 与 $\ket{v_1} \cdot w_2$ 分别代表(非标准化的)状态 $\frac{\sqrt{2}}{\sqrt{3}}\ket{0}\ket{+}\ket{-}$ 和 $\frac{1}{\sqrt{3}}\ket{1}\ket{0}\ket{-}$。$\ket{v_1}$ 的TDD表示及其窗口函数分割的TDD可在图\ref{fig:tdd-split} 中查看。
    \begin{figure}
    \centering
	\begin{subfigure}[b]{.3\textwidth}
        \centering
        \includegraphics[height=6cm]{Img/tdd_proj.pdf}
		\caption{$\ket{v_1}$的TDD表示}
        \label{fig:tdd-split-a}
	\end{subfigure}
	\begin{subfigure}[b]{.3\textwidth}
        \centering
        \includegraphics[height=6cm]{Img/tdd_proj_a0.pdf}
		\caption{$\ket{v_1}$在$w_1=\bar{q_0}$下的TDD表示}
	\end{subfigure}
    \quad
    \begin{subfigure}[b]{.3\textwidth}
        \centering
        \includegraphics[height=6cm]{Img/tdd_proj_a1.pdf}
		\caption{$\ket{v_1}$在$w_2=q_0$下的TDD表示}
	\end{subfigure}
    
    \caption{$\ket{v_1}=\frac{\sqrt{2}}{\sqrt{3}}\ket{0}\ket{+}\ket{-}+\frac{1}{\sqrt{3}}\ket{1}\ket{0}\ket{-}$的TDD表示与窗口函数分解}
    \label{fig:tdd-split}
    \end{figure}
\end{example}
\subsection*{用子空间近似TDD表示$\ket{\psi}$}
通过上述方法的优化,最终得到的TDD可能依然偏大。在很多实际应用中,对子空间进行一个合理的过度估计已足够,这与经典案例中的做法相似 \citep{Cho_1996,lin2014parallel,Wang+farside_iccad03}。
针对特定量子态 $\ket{\psi}$,能够通过包含它的适当子空间来近似 $\ket{\psi}$。
例如,假设已通过张量加法或窗口函数分割将量子态分为 $\ket{\psi} = \ket{\psi_1}+\ket{\psi_2}$。于是,可以通过 $span\{\ket{\psi_1},\ket{\psi_2}\}$ 子空间近似地表示 $\ket{\psi}$。

进一步地,也可以对量子态 $\ket{\psi}$ 进行张量积估计。即找到一系列张量积态 $\ket{\psi_1},\cdots, \ket{\psi_k}$,使得 $\ket{\psi}$ 位于它们张成的空间之内。这样,便能够计算出一个更大系统的映射。但这种方法的成本在于过估计子空间的维度可能会过大。给定量子态 $\ket{\psi}$,若 $\ket{\psi}=\sum_{j=0}^{2^{k-1}-1}{\ket{j}\ket{\phi}\ket{\gamma_j}}$,认为它的第 $k$ 个量子比特是可分割的。对于每个态 $\ket{\psi}$,首先探查是否存在可分割的量子比特。若存在,提取出相应的态并移除该量子比特;否则,将采用那些振幅非零的计算基态进行估计。
\begin{example}
    再次以例子\ref{ex-image-sub}中的
    量子态$\ket{v_1}=\frac{1}{\sqrt{3}}(\ket{00}+\ket{01}+\ket{10})\ket{-}$ 为例。图 \ref{fig:tdd-split-a}给出了该量子态的TDD表示。
可以发现其第三个量子比特$q_2$是可分割的,即所有的$q_0$,$q_1$都会经过同样的$q_2$结点。$q_2$对应的量子态为 $\ket{-}$。在移除这个量子比特后,得到 $\ket{\psi}=\frac{1}{\sqrt{3}}(\ket{00}+\ket{01}+\ket{10})$,该态属于子空间 $span\{\ket{00},\ket{01},\ket{10}\}$。从而为态 $\ket{v_1}$ 找到了一种张量积的近似表示 $\{\ket{00-},\ket{01-},\ket{10-}\}$。
\end{example}

\section{软件系统实现}
为了实现软件的高效运行,模块化设计至关重要。每个模块在软件系统中扮演着关键角色,并且具有特定的功能和目的。以下是本次毕业设计中软件必须包含的模块及其重要性的说明:
\begin{itemize}
    \item \textbf{输入处理模块}:该模块的主要职责是处理输入数据,例如接收用OpenQASM格式编写的量子算法代码。其核心功能是将这些代码转换为TDD表示形式。鉴于当前存在多种量子编程语言,此模块的模块化处理能够显著提升系统的灵活性和兼容性。
    \item \textbf{内存管理模块}:本模块负责管理TDD节点的存储和维护。当创建新的TDD节点时,它会运用哈希算法与现有节点进行对比,以避免重复创建相同节点。这种方法不仅减少了内存占用,还提高了处理效率。
    \item \textbf{TDD基础模块}:该模块主要执行TDD节点的压缩操作,或者导出TDD的树状结构图。节点收缩是TDD核心的运算过程,而树状结构图的导出功能则有助于用户更好地理解和分析TDD的结构。
    \item \textbf{TDD算法模块}:此模块为TDD提供更复杂的算法支持。例如,它能够调整节点收缩的顺序,以优化系统运行效率。此外,它还能执行其他高级功能,如检验TDD是否存在于特定子空间中,对TDD进行分割。
\end{itemize}
通过上述模块的协同工作,
保证了最终的软件系统能够高效、灵活地处理本文工作中对量子模型检测的各种需求。
\section{本章小结}
本章主要介绍了在本文工作中采用的基于张量决策图(TDD)的量子模型检测方法。首先简单回顾了如何将量子系统建模为量子迁移系统。然后给出了计算子空间基、子空间并运算、以及系统一步迁移的算法。为加速模型检测,本章还介绍了一些针对性优化方法,包括索引顺序调整、addition线路拆分、contraction 线路拆分、基于TDD的划分,以及近似TDD表示等技术。最后,概述了软件系统的模块化设计,包括输入处理、内存管理、TDD基础运算和TDD算法等模块。

通过上述方法,本文以TDD为数据结构,构建了一种能高效计算量子模型检测中可达性分析问题的解决方案。TDD结合了BDD的紧凑表示和张量网络的计算优势,为解决量子计算中的模型检测问题提供了新的可能性。软件系统的模块化设计也为高效实现这一方案奠定了基础。总的来说,本章介绍的技术为量子模型检测提供了一种有前景的新方法。