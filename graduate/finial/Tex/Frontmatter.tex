%---------------------------------------------------------------------------%
%->> Frontmatter
%---------------------------------------------------------------------------%
%-
%-> 生成封面
%-

\maketitle% 生成中文封面
\MAKETITLE% 生成英文封面
%-
%-> 作者声明
%-
\makedeclaration% 生成声明页
%-
%-> 中文摘要
%-
\intobmk\chapter*{摘\quad 要}% 显示在书签但不显示在目录
\setcounter{page}{1}% 开始页码
\pagenumbering{Roman}% 页码符号

% 随着量子计算领域的快速发展,模型检测作为一种验证技术,在确保量子系统可靠性方面显得尤为关键。本次研究着重于解决量子模型检测中的一个主要挑战:当面对大规模和复杂的量子线路时,计算复杂性会随着系统规模的增长而指数级增加。为应对这一挑战,本研究提出利用张量决策图(Tensor Decision Diagram,简称TDD)来设计一种更高效的量子模型检测方法。

% 研究的核心在于开发能够有效利用TDD进行操作和收缩张量网络的算法,以此降低计算的复杂性和提高处理效率。在具体实现上,包括了开发新技术以分割张量网络、优化TDD结构,以及改进现有算法以适应复杂的量子系统。此外,研究还涉及设计实现一种基于TDD的量子线路状态转移方法,通过逐步收缩表示量子线路的TDD和当前系统状态的TDD,推导出系统下一状态的TDD。这种方法能够有效地验证系统所处的子空间,从而评估量子系统是否达到了需要验证的属性。

% 在本次中期报告前,本研究已在Python和C++两种编程语言环境下,开发出不同版本的TDD,并成功实现了系统的一步状态计算,即image computation。此外,对这些实现进行了一系列的实验测试,以验证其有效性和性能。

% 本研究的最终目标是通过详细的分析和实验,撰写一篇硕士学位论文。该论文将总结本研究的主要成果,为量子系统的设计、验证和优化提供有力的理论和实践支持。通过这项工作,希望为量子计算技术的进一步发展奠定坚实的理论和应用基础。
随着近年来量子计算机在规模和可靠性方面的快速发展,针对量子系统进行自动化验证已成为亟待解决的重要问题。传统的模型检测技术在面对量子系统时往往遇到资源消耗过多的瓶颈。为此,本文提出了一种基于张量网络和张量决策图(TDD)的量子模型检测新方法,旨在降低资源需求,扩展算法的适用范围。

首先,对量子计算、模型检测及其相关数学基础进行了概述,为读者了解研究内容做好铺垫。然后介绍了TDD数据结构,它本质上是一种带权重的决策树,用于紧凑高效地表示张量网络。接着阐述了如何将量子线路转化为TDD表示,并介绍了TDD的规范化、化简和基本运算等操作。

在此基础上,提出了基于TDD的量子模型检测算法流程。首先将量子系统建模为量子迁移系统,然后给出了计算TDD表示的子空间的基、子空间并运算以及系统一步迁移的具体算法。为加速计算,设计了多种优化策略,包括电路拆分、基于TDD的划分,以及近似TDD表示等。

通过在多种量子算法实例上的实验对比,结果表明这些优化方法能够有效降低模型检测中一步迁移过程的时间和空间资源消耗,从而提高算法的可扩展性。其中,基于收缩(contraction)的电路拆分技术优化效果最为显著,能显著提升处理线路层数较深的量子算法的能力。

最后,总结了该研究工作对量子模型检测的贡献,并讨论了未来的工作方向,如结合更复杂量子系统建模、改进TDD的表示能力等。该项研究为量子计算机软硬件可靠性验证提供了有价值的新工具支持。
%//TODO:修改摘要

\keywords{模型检测,量子计算,张量决策图}% 中文关键词
% //TODO:中文关键词
%-
%-> 英文摘要
%-
\intobmk\chapter*{Abstract}% 显示在书签但不显示在目录

% With the rapid development of the quantum computing field, model checking, as a verification technology, becomes particularly critical in ensuring the reliability of quantum systems. This research focuses on addressing a major challenge in quantum model checking: the computational complexity exponentially increases with the system size when facing large-scale and complex quantum circuits. To tackle this challenge, the study proposes using Tensor Decision Diagrams (TDD) to design a more efficient quantum model checking method.

% The core of the research lies in developing algorithms that can effectively utilize TDD for operations and contraction of tensor networks, thereby reducing computational complexity and improving processing efficiency. Specifically, it includes developing new techniques for partitioning tensor networks, optimizing TDD structures, and improving existing algorithms to adapt to complex quantum systems. Moreover, the study involves designing and implementing a TDD-based quantum circuit state transition method. This method deduces the TDD of the system's next state by progressively contracting the TDD representing the quantum circuit and the TDD of the current system state. This method can effectively verify the subspaces in which the system resides, thereby assessing whether the quantum system has achieved the properties that need to be verified.

% By the mid-term report, this research had developed different versions of TDD in both Python and C++ programming environments and successfully implemented the system's one-step state computation, i.e., image computation. In addition, a series of experimental tests were conducted to verify their effectiveness and performance.

% The ultimate goal of this research is to write a master's thesis through detailed analysis and experiments. This thesis will summarize the main findings of the research, providing strong theoretical and practical support for the design, verification, and optimization of quantum systems. Through this work, it aims to lay a solid theoretical and application foundation for the further development of quantum computing technology.
With the rapid development of quantum computers in terms of scale and reliability in recent years, automated verification for quantum systems has become an important problem that urgently needs to be addressed. Traditional model checking techniques often encounter the bottleneck of excessive resource consumption when facing quantum systems. To this end, this paper proposes a new method for quantum model checking based on tensor networks and Tensor Decision Diagrams (TDD), aimed at reducing resource requirements and expanding the applicability of the algorithm.

Firstly, an overview of quantum computing, model checking, and their related mathematical foundations is provided, laying the groundwork for readers to understand the research content. Then, the TDD data structure is introduced, which is essentially a weighted decision tree used for compact and efficient representation of tensor networks. The paper then explains how to transform quantum circuits into TDD representations and introduces operations such as normalization, simplification, and basic operations of TDDs.

On this basis, a TDD-based quantum model checking algorithm process is proposed. It starts with modeling the quantum system as a quantum transition system, then presents the algorithms for computing the basis of the subspace represented by TDD, subspace union operations, and the specific algorithm for system one-step transitions. To accelerate computation, several optimization strategies are designed, including circuit splitting, TDD-based partitioning, and approximate TDD representation, among others.

Experimental comparisons on various quantum algorithm instances show that these optimization methods can effectively reduce the time and space resource consumption in the image computation process of model checking, thereby improving the scalability of the algorithm. Among them, the circuit splitting technique based on contraction optimization has the most significant effect, significantly enhancing the capacity to handle quantum algorithms with deeper circuit layers.

Finally, the paper summarizes the contributions of this research work to quantum model checking and discusses future work directions, such as integrating more complex quantum system modeling and improving the representation capability of TDDs. This research provides valuable new tool support for the reliability verification of quantum computer hardware and software.

\KEYWORDS{Model Checking, Quantum Computation, Tensor Decision Diagrams}% 英文关键词

\pagestyle{enfrontmatterstyle}%
\cleardoublepage\pagestyle{frontmatterstyle}%

%---------------------------------------------------------------------------%
