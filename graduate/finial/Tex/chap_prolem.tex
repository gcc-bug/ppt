\chapter{问题介绍}

\section{量子模型检测中的关键问题}
\begin{itemize}
  \item 可达性分析
  \item 模型检验的挑战
\end{itemize}
% //TODO: add key problem here

\section{TDD在量子模型检测中的应用}

\section{研究问题的现实意义}
\begin{itemize}
  \item QDA方面的验证
\end{itemize}
% //TODO: add usage here
在模型检测中,有三类比较重要的可达性问题,分别是可达性、持续可达性以及重复可达性。本研究主要应用基于TDD的量子模型检测,探究可达性问题,从而解决实际量子线路问题。

比如RUS电路等价性问题。图\ref{fig:rus-equal}展示了两个RUS量子电路。当电路中测量门结果为$00$时,第三个比特的输出态均为$\left(I+2iZ\right)/ \sqrt 5 |\psi\rangle$。这样的等价性检验任务,可以转换为模型检测中的可达空间是否一致。
\begin{figure}[!htbp]
	\centering
	\begin{subfigure}[b]{0.4\textwidth}
        \centering
        \includegraphics[height=3cm]{Img/rus_s.pdf}
	\end{subfigure}
	\qquad
	\begin{subfigure}[b]{0.4\textwidth}
        \centering
        \includegraphics[height=3cm]{Img/rus_T.pdf}
	\end{subfigure}
	\caption{两个等价的RUS电路\citep{Bocharov_2015}}
	\label{fig:rus-equal}
\end{figure}
