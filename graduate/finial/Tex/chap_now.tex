\chapter{已有科研基础与所需的科研条件}
在模型检测中,从目前已知状态出发,计算下一状态的image computation起到了关键的作用。这一步目前已经完成。而得益于导师和研究组师兄的帮助,后续过程也会比较顺利。
\section{研究基础}
在模型检测中,image computation指的是在给定当前状态$s_i\in S$和行为$\alpha\in Act$的情况下计算接下来的状态。

目前,关于使用TDD对量子的image computation的计算已经完成。表\ref{table:time}给出了在不同电路拆分技术下Grover算法的计算时间,单位为秒。其中basic表示没有使用优化技术,addition表示使用\ref{addition}中的addition优化技术,contraction表示使用\ref{contraction}中的contraction优化技术。“-”表示超过一小时的运行上限。
\begin{table}[!htbp]
    \centering
    \begin{tabular}{llllllllll}
        \hline
        \multirow{2}{*}{Benchmark} &  & \multicolumn{2}{c}{basic} &  & \multicolumn{2}{c}{addition} &  & \multicolumn{2}{c}{contraction} \\ \cline{3-4} \cline{6-7} \cline{9-10} 
                                   &  & time        & max \#node       &  & time          & max \#node        &  & time           & max \#node          \\ \hline
        Grover\_15 &   & 19.33  & 15785     &   & 17.35      & 15099  & & 1.61 & 597  \\
        Grover\_18 &   & 76.47  & 61694     &   & 66.02      & 60332  & & 2.41 & 516  \\
        Grover\_20 &   & 294.65 & 243946    &   & 259.87     & 241240 & & 4.39  & 1036 \\ 
        Grover\_40 &   & -      &           &   & -          &        & & 2953.57 & 851973 \\
        \hline
        QFT\_15     &  & 34.64   & 65536   &  & 18.88  & 32770   &  & 0.08 & 63  \\
        QFT\_18     &  & 282.12  & 524288  &  & 148.13 & 262146 &   & 0.10  & 31  \\
        QFT\_20     &  & 1199.21 & 2097152 &  & 655.19 & 1048578 &  & 0.12 & 63  \\
        QFT\_30     &  & -       &         &  & -      &        &  & 0.29 & 31  \\
        QFT\_50     &  & -       &         &  & -      &        &  & 1.02 & 51  \\
        QFT\_100    &  & -       &         &  & -      &        &  & 7.14 & 101 \\
        \hline
        BV\_100     &  & 7.36    & 596     &  & 7.43      & 596     &  & 0.41           & 102 \\
        BV\_200     &  & 31.57   & 1196    &  & 30.03     & 1196    &  & 1.70           & 202 \\
        BV\_300     &  & 75.66   & 1796    &  & 75.56     & 1796    &  & 4.28           & 302 \\
        BV\_400     &  & 146.47  & 2396    &  & 145.40    & 2396    &  & 9.18           & 402 \\
        BV\_500     &  & 244.15  & 2996    &  & 223.90    & 2996    &  & 16.31          & 502 \\
        \hline
        GHZ\_100    &  & 0.38    & 595     &  & 0.13      & 301    &  & 0.18           & 200 \\%& 0.03    
        GHZ\_200    &  & 0.72    & 1195    &  & 0.37      & 601    &  & 0.48           & 400 \\%& 0.12     
        GHZ\_300    &  & 1.29    & 1795    &  & 0.62      & 901    &  & 0.80           & 600 \\%& 0.24     
        GHZ\_400    &  & 2.03    & 2395    &  & 1.00      & 1201    &  & 1.26           & 800 \\%& 0.42     
        GHZ\_500    &  & 2.96    & 2995    &  & 1.45      & 1501    &  & 1.72           & 1000\\%& 0.62     
        \hline
        QRW\_15     &  & 36.86   & 13122     &  & 24.59     & 10882     & & 7.16  & 222 \\
        QRW\_18     &  & 139.76  & 90538     &  & 84.69     & 37064     & & 11.23 & 226 \\
        QRW\_20     &  & 341.05  & 265614    &  & 218.29    & 107714    & & 14.31 & 404 \\
        QRW\_30     &   &-       &          &  &-          &          & & 36.82 & 404 \\
        QRW\_50     &   &-       &          &  &-          &          & & 118.08 & 404 \\
        QRW\_100    &   &-       &          &  &-          &          & & 692.08 & 436 \\
        \hline
    \end{tabular}
    \caption{对不同测试实验应用image computation}
    \label{table:time}
\end{table}

通过对比不同优化技术下的计算时间,可以看到使用优化技术能够显著降低计算时间。例如,在Grover-20的例子中,使用"contraction"优化技术的情况下,计算时间从294秒降低到了4秒。这表明优化技术在提高计算效率方面起到了积极的作用。同时,表\ref{table:addition}展示了对同一线路,即Grover\_15应用不同的addition参数的时间,可以看到合适的参数
选择也是非常重要的。
\begin{table}[!htbp]
    \centering
    \begin{tabular}{c|ccccccccccccccc}
        \rowcolor[HTML]{FFFFFF} 
        \diagbox{k1}{k2}                         & 1                           & 2                           & 3                           & 4                           & 5                           & 6                           & 7                          & 8                           & 9                           & 10                          & 11                          & 12                          & 13                          & 14                          & 15                          \\\hline
            \rowcolor[HTML]{FFFFFF} 
    1                          & 2.8                                              & 2.2                         & 2.1                         & \cellcolor[HTML]{CCC0DA}2.0 & \cellcolor[HTML]{CCC0DA}1.9 & \cellcolor[HTML]{CCC0DA}2.0                      & 2.1                        & \cellcolor[HTML]{CCC0DA}2.0 & 2.1                         & \cellcolor[HTML]{CCC0DA}2.0 & \cellcolor[HTML]{CCC0DA}2.0 & 2.1                         & 2.2                         & 2.1                         & 2.1                         \\ \cline{3-7}
    \rowcolor[HTML]{CCC0DA} 
    
    \rowcolor[HTML]{FFFFFF} 
    3                          & \multicolumn{1}{l|}{\cellcolor[HTML]{FFFFFF}2.2} & \cellcolor[HTML]{CCC0DA}1.9 & \cellcolor[HTML]{CCC0DA}1.8 & \cellcolor[HTML]{CCC0DA}1.6 & \cellcolor[HTML]{CCC0DA}2.0 & \multicolumn{1}{l|}{\cellcolor[HTML]{CCC0DA}1.9} & 2.1                        & 2.1                         & 2.5                         & 2.3                         & 2.7                         & 2.3                         & 3.1                         & 2.8                         & 3.3                         \\
    \rowcolor[HTML]{FFFFFF} 
    4                          & \multicolumn{1}{l|}{\cellcolor[HTML]{FFFFFF}2.3} & \cellcolor[HTML]{CCC0DA}1.8 & \cellcolor[HTML]{CCC0DA}2.0 & \cellcolor[HTML]{CCC0DA}1.7 & \cellcolor[HTML]{CCC0DA}2.0 & \multicolumn{1}{l|}{\cellcolor[HTML]{FFFFFF}2.1} & 2.2                        & 2.1                         & 2.6                         & 2.3                         & 2.8                         & 2.7                         & 3.3                         & 3.0                         & 3.3                         \\
    \rowcolor[HTML]{FFFFFF} 
    5                          & \multicolumn{1}{l|}{\cellcolor[HTML]{FFFFFF}2.2} & \cellcolor[HTML]{CCC0DA}1.7 & \cellcolor[HTML]{CCC0DA}1.9 & \cellcolor[HTML]{CCC0DA}1.6 & \cellcolor[HTML]{CCC0DA}1.9 & \multicolumn{1}{l|}{\cellcolor[HTML]{CCC0DA}2.0} & 2.3                        & \cellcolor[HTML]{CCC0DA}1.9 & 2.5                         & 2.3                         & 2.8                         & 2.7                         & 3.4                         & 3.0                         & 3.6                         \\
    \rowcolor[HTML]{FFFFFF} 
    6                          & \multicolumn{1}{l|}{\cellcolor[HTML]{FFFFFF}2.1} & \cellcolor[HTML]{B1A0C7}1.5 & \cellcolor[HTML]{CCC0DA}1.8 & \cellcolor[HTML]{CCC0DA}1.7 & 2.2                         & \multicolumn{1}{l|}{\cellcolor[HTML]{CCC0DA}1.9} & 2.5                        & 2.2                         & 2.9                         & 2.8                         & 3.1                         & 2.9                         & 3.7                         & 3.7                         & 4.2                         \\
    \rowcolor[HTML]{FFFFFF} 
    7                          & \multicolumn{1}{l|}{\cellcolor[HTML]{FFFFFF}2.1} & \cellcolor[HTML]{CCC0DA}1.5 & \cellcolor[HTML]{CCC0DA}1.9 & \cellcolor[HTML]{CCC0DA}1.6 & 2.2                         & \multicolumn{1}{l|}{\cellcolor[HTML]{CCC0DA}1.9} & 2.5                        & 2.2                         & 2.8                         & 3.0                         & 3.6                         & 3.3                       & 4.2                         & 5.7                         & 5.0                         \\
    \rowcolor[HTML]{FFFFFF} 
    8                          & \multicolumn{1}{l|}{\cellcolor[HTML]{CCC0DA}2.0} & \cellcolor[HTML]{CCC0DA}1.7 & \cellcolor[HTML]{CCC0DA}1.8 & \cellcolor[HTML]{CCC0DA}1.7 & 2.1                         & \multicolumn{1}{l|}{\cellcolor[HTML]{CCC0DA}2.0} & 2.4                        & 2.2                         & 2.8                         & 2.8                         & 3.7                         & 3.4                         & 4.3                         & 4.8                         & 5.2                         \\
    \rowcolor[HTML]{FFFFFF} 
    9                          & \multicolumn{1}{l|}{\cellcolor[HTML]{FFFFFF}2.1} & \cellcolor[HTML]{B1A0C7}1.5 & \cellcolor[HTML]{CCC0DA}2.0 & \cellcolor[HTML]{B1A0C7}1.4 & 2.2                         & \multicolumn{1}{l|}{\cellcolor[HTML]{CCC0DA}2.0} & 2.5                        & \cellcolor[HTML]{CCC0DA}2.0 & 3.3                         & 2.9                         & 3.7                         & 3.5                         & 4.9                         & 4.7                         & 5.8                         \\ \cline{3-7}
    \rowcolor[HTML]{FFFFFF} 
    10                         & 2.3                                              & \cellcolor[HTML]{CCC0DA}1.9 & 2.3                         & \cellcolor[HTML]{CCC0DA}1.6 & 2.6                         & 2.7                                              & 3.1                        & 2.2                         & 4.0                         & 3.6                         & 4.6                         & 3.9                         & 5.6                         & 5.2                         & 7.5                         \\
    \rowcolor[HTML]{FFFFFF} 
    11                         & 3.2                                              & 3.2                         & 3.5                         & 3.1                         & 4.7                         & 4.2                                              & 5.6                        & 4.2                         & 6.8                         & 7.2                         & 7.6                         & 6.3                         & 9.0                         & 8.1                         & \cellcolor[HTML]{8DB4E2}11  \\
    \rowcolor[HTML]{FFFFFF} 
    12                         & 5.6                                              & 6.0                         & 7.2                         & 6.0                         & 8.3                         & 9.0                                              & 8.9                        & 7.8                         & \cellcolor[HTML]{8DB4E2}11  & \cellcolor[HTML]{8DB4E2}11  & \cellcolor[HTML]{8DB4E2}12  & \cellcolor[HTML]{8DB4E2}11  & \cellcolor[HTML]{8DB4E2}12  & \cellcolor[HTML]{8DB4E2}15  & \cellcolor[HTML]{8DB4E2}16  \\
    \rowcolor[HTML]{8DB4E2} 
    \cellcolor[HTML]{FFFFFF}13 & 11                                               & 12                          & 14                          & 12                          & 15                          & 18                                               & 18                         & 15                          & 18                          & 20                          & 18                          & 32                          & 32                          & 30                          & 25                          \\
    \rowcolor[HTML]{8DB4E2} 
    \cellcolor[HTML]{FFFFFF}14 & 20                                               & 21                          & 24                          & 32                          & 31                          & 44                                               & 77                         & 50                          & 86                          & \cellcolor[HTML]{538DD5}109 & 68                          & \cellcolor[HTML]{538DD5}133 & 70                          & \cellcolor[HTML]{538DD5}119 & \cellcolor[HTML]{538DD5}142 \\
    \rowcolor[HTML]{538DD5} 
    \cellcolor[HTML]{FFFFFF}15 & \cellcolor[HTML]{8DB4E2}28                       & \cellcolor[HTML]{8DB4E2}30  & \cellcolor[HTML]{8DB4E2}31  & \cellcolor[HTML]{8DB4E2}53  & \cellcolor[HTML]{8DB4E2}69  & 111                                              & \cellcolor[HTML]{8DB4E2}85 & \cellcolor[HTML]{8DB4E2}81  & 102                         & 153                         & 114                         & 130                         & 166                         & 162                         & 235                        
                     
    \end{tabular}
    \caption{对grover\_15应用不同的addition 参数}%calculating the 
    \label{table:addition}
\end{table}
\section{必要研究条件}
在硬件方面,实验室提供的服务器为运行程序提供了便利。对于经典计算机模拟量子计算而言,内存容量一直是主要的瓶颈之一。然而,实验室服务器提供了大量的内存空间,为程序的运行提供了必要的条件。这意味着我可以更轻松地处理复杂的计算任务,探索量子线路的可达性、持续可达性以及重复可达性。

在软件层面,实验室内的老师和师兄们拥有丰富的研究经验,他们的知识和经验将为选题的完成提供必要的支持。举例来说,我的导师是应圣钢老师,他在博士期间主要致力于量子马尔可夫链的可达性分析。他的研究背景与我的选题非常契合,而且他在学术指导方面也表现出耐心和专业性。有他的指导,我可以充分利用他的专业知识和研究经验,解决在研究过程中可能遇到的问题。

实验室提供的硬件和导师的支持为我的研究工作奠定了坚实的基础。我相信在这样的环境下,我将能够深入探索基于TDD的量子模型检测方法,以及对量子线路的可达性、持续可达性和重复可达性进行计算与验证。
