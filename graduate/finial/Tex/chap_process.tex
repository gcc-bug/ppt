\section{拟采用的研究过程、技术路线、实验评估方案及其可行性分析}
本次研究的主要目的是借助TDD数据结构,构建能快速计算量子模型检测中可达问题的方案。因此主要采用的方法是模拟量子计算。本次研究的主要挑战在于尽可能减少程序的运行时间。为此,需要采用一系列方法来开发更有效的算法,以优化TDD操作和收缩张量网络。其中包括开发新技术来分割张量网络和优化TDD结构。在各种方法中,分割张量网络和拆分决策图被认为是主要的优化策略,并且已经取得了一定的研究成果。
\subsection{Tensor Decision Diagrams(TDD)介绍}
\subsubsection{张量网络表示量子线路}
张量是与一组与索引\(I=\{x_1,\ldots,x_n\}\)相关联的多维线性映射。在量子计算中,可以假设只从\(\{0,1\}\)中取值。因此,张量定义为:
\begin{align}
    \phi :{\{0,1\}}^I\rightarrow\mathbb{C}
\end{align}
其中\(\mathbb{C}\)为实数。张量网络是一个无向图\(G=\left(V,E\right)\)。其中顶点集V中的每个顶点v表示一个张量。边集\(E\)中每条边\(e\)代表与相邻两个张量相关联的公共索引。通过以任意顺序收缩连接的张量,可以得到一个秩为\(m\)的张量,其中\(m\)是$G$中开放边数。这个独立于收缩顺序的张量也称为该张量网络的张量表示\citep{biamonte2019lectures}。
张量网络提供了一种新的量子线路表示方法\citep{pednault2017breaking}。一个向量表示为$[\alpha_0,\alpha_1]$的量子比特x可以描述为秩为1的张量$\phi_x$,其中$\phi_x\left(0\right)=\alpha_0, \phi_x\left(1\right)=\alpha_1$。具有输入比特x和输出比特y的单比特门可以表示为秩2张量$\phi_{xy}$。例如,以x作为输入比特和y作为输出比特的单比特门Z-gate的张量表示:$\phi_{xy}\left(00\right)=1,\phi_{xy}\left(01\right)=\phi_{xy}\left(10\right)=0,\phi_{xy}\left(11\right)=-1$。类似的$n$比特量子门可以表示成一个秩为$2n$的张量。在张量表示中,一般不区分输入和输出索引。只有当将张量解释为门或电路时会规定关于其信息。

给定量子线路中所有量子门的输入和输出状态的索引值,就可以得到量子电路的张量表示。图\ref{fig:example_cir_map}给定了图\ref{fig:example_cir}中量子线路中各个状态的索引值,从而可得的张量表示:
\begin{align}
\phi_{x_0x_3y_0y_3}\left(a_0a_3b_0b_3\right)=\sum_{a_1,a_2,b_1,b_2=0}^{1}T\left(a_0a_1\right)H\left(b_0b_1\right)CX\left(a_1b_1a_2b_2\right)T\left(a_2a_3\right)H\left(b_2b_3\right)
\end{align}

\begin{figure}[!htbp]
    \centering
    \includegraphics[width=.6\textwidth]{Img/tensor_example.pdf}
    \caption{用张量表示图\ref{fig:example_cir}中量子线路}   
    \label{fig:example_cir_map}
\end{figure}

\subsubsection{张量决策图}
张量决策图(Tensor Decision Diagrams,或TDD)是一种具有决策图和张量网络特征的数据结构\citep{Hong_2022}。它可用于表示张量和量子电路。与BDD(Boolean Decision Diagrams)类似,TDD是一种建立在索引顺序$I=\{x_1,\ldots,x_n\}$上的决策树模型。具体定义如下:
\begin{align}
    \mathcal{F}=\left(V,E,index,value,low,high,w\right)
\end{align}
其中$V$是一个有限节点集,被划分为非终端节点$V_n$和终端节点$V_T$。用$r_{\mathcal{F}}$表示$\mathcal{F}$的唯一根节点。$E=\left\{\left(v,low\left(v\right)\right),\left(v,high\left(v\right)\right):v\in V_N\right\}$是树中所有边集合,其中$\left(v,low\left(v\right)\right)$和$\left(v,high\left(v\right)\right)$分别称为v的低边和高边。根结点$r_{\mathcal{F}}$具有唯一的入射边$e_{\mathcal{F}}$,该入射边没有源结点。$index:V_r\rightarrow I$将每个非终端节点分配给I中的索引。$value:V_T\rightarrow\mathbb{C}$将每个终端节点赋予一个复数值。$low$和$high$都是$V_N\rightarrow V$中的映射,它们分别为每个非终端节点指定其低边和高边后继。$w:E\rightarrow\mathbb{C}$将每条边赋予一个复数权重。特别地,$w\left(e_r\right)$称为$\mathcal{F}$的权重,并记作$w_{\mathcal{F}}$。 



图\ref{fig:tdd_ex} 展示了一个TDD的例子。其中索引顺序$I=\{x_0,y_0,y_3\}$。非终端节点$V_n$用圆表示,终端节点$V_T$用方形表示。每个节点的低边用虚线表示,高边用实线表示。该示例中各边的权重均为1,即$w:E\rightarrow 1$。根节点$r_{\mathcal{F}}=x_0$。决策树的权重$w_{\mathcal{F}}=1$。
\begin{figure}[!htbp]
    \centering
    \includegraphics[width=.6\textwidth]{Img/tree_tdd.pdf}
    \caption{一个TDD示例}   
    \label{fig:tdd_ex}
\end{figure}

对于TDD中的一个节点$v$,如果$v$是终端节点,则$\phi\left(v\right):= valuev$是一个秩为$0$的张量,即常数。如果$v$是非终端节点,则:
\begin{align}
    \phi(v):=w_{0} \cdot \overline{x_{v}} \cdot \phi(\operatorname{low}(v))+w_{1} \cdot x_{v} \cdot \phi(h i g h(v))
\end{align}
其中$x_v=index\left(v\right),\bar{x_v}=1-index\left(v\right)$。如果$v$是非终端节点,也可以表示为:
\begin{align}
    \left.\phi\left(v\right)\right|_{x_v=c}≔w_c \phi (v_c)
\end{align}
其中$c\in\{0,1\}$。因此整个TDD也可以表示为:
\begin{align}
    \phi\left(\mathcal{F}\right)=w_{\mathcal{F}}\cdot\phi\left(r_{\mathcal{F}}\right)
\end{align}
\subsubsection{TDD的规范与化简}
对于图表 6中的TDD,在结构上显然比较冗余。因此可以先进行规范化(normalized),让后进行化简(reduced)\citep{Hong_2022}。
规范化的目的是使得TDD的终端节点只包含0和1,同时将终端节点的张量值沿路径逐步上移。具体规范步骤如下:
\begin{myen}
    \item 如果$v$是一个非零值$value\left(v\right)\neq 1$的终端节点,则将其值设置为1,并将每个入边的权重w更改为$value\left(v\right)\cdot w$。\label{norm1}
    \item 假设v是一个非终端节点,且$\phi\left(v\right)\neq 0$。首先规范化$\phi\left.\left(low(v\right)\right)$和$\phi\left.\left(high(v\right)\right)$。完成$\phi\left.\left(low(v\right)\right)$和$\phi\left.\left(high(v\right)\right)$的规范化后。如果$\phi\left.\left(low(v\right)\right)\neq 0$,并且此时$\phi\left.\left(high(v\right)\right)=0$或$\left|w_0\right|\geq\left|w_1\right|$,则将$w$设置为$w_0$。否则,将$w$设置为$w_1$。然后更新$w_0≔w_0/w,w_1≔w_1/w$。此时完成$v$的规范化。\label{norm2}
\end{myen}

\begin{figure}[!htbp]
    \centering
    \begin{subfigure}[b]{0.4\textwidth}
        \centering
        \includegraphics[height=5cm]{Img/tree_norm1.pdf}
        \caption{对图\ref{fig:tdd_ex}的终端节点规范化}
        \label{fig:tdd-norma}
    \end{subfigure}
    \begin{subfigure}[b]{0.4\textwidth}
        \centering
        \includegraphics[height=5cm]{Img/tree_norm2.pdf}
        \caption{对图\ref{fig:tdd-norma}的$y_3$节点规范化}
        \label{fig:tdd-normb}
    \end{subfigure}
    \\
    \begin{subfigure}[b]{0.8\textwidth}
        \centering
        \includegraphics[width=.8\textwidth]{Img/tree_norm3.pdf}
        \caption{对图\ref{fig:tdd_ex}的规范化结果}
        \label{fig:tdd-normc}
    \end{subfigure}
    % \caption{对图\ref{fig:tdd_ex}的规范化过程}
    \label{fig:tdd-norm}
\end{figure}
对于图\ref{fig:tdd_ex}中的TDD。具体规范化过程为:应用第\ref{norm1}步于$i$和$-i$两个终端节点得到图表 \ref{fig:tdd-norma};然后,应用第\ref{norm2}步于右侧两个$y_3$节点,得到图表 \ref{fig:tdd-normb};最后,应用第二条规则于于右侧$y_0$节点。最后可以获得图表 \ref{fig:tdd-normc}。

在完成规范化后,可以进一步简化TDD,使得TDD只包含一个终端节点,同时尽量使用重复出现的张量。具体简化步骤如下:
\begin{myen}
    \item 	合并所有值为1的终端节点。删除所有终端$0$节点,并将它们的入边重定向到唯一的终端节点,并将它们的权重重置为$0$。\label{sympl1}
	\item 将所有权重为$0$的边重定向到终端节点。如果根节点的入边权重为$0$,则终端节点成为新的根,该TDD为空。删除所有从根节点不可达的节点,以及涉及它们的所有边。\label{sympl2}
	\item 如果一个节点v的低边和高边后继相同,并且其低边和高边具有相同的权重w,则删除该节点。如果入边权重$w=0$,则将其传入边重定向到终端节点。否则,将其传入边重定向到其后继节点。\label{sympl3}
	\item 合并两个具有相同索引、相同$0$和$1$后继以及对应边上相同权重的节点。\label{sympl4}
\end{myen}

\begin{figure}[!htbp]
    \centering
    \begin{subfigure}[b]{0.4\textwidth}
        \centering
        \includegraphics[height = 6cm]{Img/tree_redu1.pdf}
        \caption{对\ref{fig:tdd-normc}中规范TDD的终端节点化简}
        \label{fig:tdd-redu1}
    \end{subfigure}
    \begin{subfigure}[b]{0.4\textwidth}
        \centering
        \includegraphics[height = 6cm]{Img/tree_redu2.pdf}
        \caption{对图\ref{fig:tdd-normc}中规范TDD的简化结果}
        \label{fig:tdd-redu2}
    \end{subfigure}
    \label{fig:tdd-redu}
\end{figure}

对于图\ref{fig:tdd-normc}中规范TDD。具体化简过程为:首先重复第\ref{sympl1} 步,合并所有终端节点,得到图\ref{fig:tdd-redu1};由于没有符合第\ref{sympl2}和第\ref{sympl3}步的节点,因此直接进行第\ref{sympl4}步应用重复第四条规则合并第一个和第三个$y_3$节点,以及第二个和第四个$y_3$节点。最终可以得到\ref{fig:tdd-redu2}所示的简化TDD。

下文中所有的TDD均指化简后的TDD,即reduced tensor decision diagrams。

\subsection{研究过程}
量子模型的跃迁系统定义为:$\mathcal{M}=\{\mathcal{H},Act,\{U_\alpha,\alpha\in Act\},\mathcal{H}_0\}$。以可达空间的研究过程为例,主要为以下几步:
\begin{myen}
	\item \label{item:process_inital}将可达空间初始化为$\mathcal{H}_0$。
	\item \label{item:process_state}取当前可达空间的最大纠缠态作为当前状态,并得到对应的TDD表示。
	\item \label{item:process_cir}获得量子电路,即转换关系$U_\alpha,\alpha\in Act$的TDD表示。
	\item \label{item:process_cont} 将前两步的TDD进行收缩,即通过基于TDD模拟计算得到下一个量子状态的TDD。
	\item \label{item:process_end}将下一个TDD加入到可达空间中,如果可达空间发生变化,回到\ref{item:process_state}步,否则结束。
\end{myen}
	
依据系统的可达空间,就可以对其他可达性问题进行分析。整个研究过程也基本按实现过程逐步推进。
\subsection{技术路线}
在应用TDD计算模型检测中,存在两个关键步骤。第一步是获取量子线路的TDD表示。如何利用更少的资源表示一个量子线路,即更简化的TDD表示是优化计算流程的一个重要的思路。
比如和经典类似的,索引值的顺序会影响TDD的资源。还有一些量子特有的思路,比如利用局部表示的等价性化简决策图。具体内容,下文简要介绍。

而第二步是逐步收缩表示当前状态的TDD与表示量子线路的TDD,最终得到下一步状态的TDD。在这两个步骤中加入好的优化方法是能够减少运行时间的关键。在不同的分割量子线路方案中,有两类主要技术路线,下文也将简要介绍。
\subsubsection{索引对决策图的影响}

在BDD中,索引的顺序很重要。因为索引顺序会直接影响BDD的大小。一个良好的变量顺序可以使得BDD比一个糟糕的变量顺序小得多。图\ref{fig:bdd-compare}的了两张图都表示了布尔函数ƒ(x1,...,x8)=x1x2+x3x4+x5x6+x7x8,但图\ref{fig:bdd-good}的结构更简单。其中图\ref{fig:bdd-bad}的索引顺序为\{x1,x3,x5,x7,x2,x4,x6,x8\},图\ref{fig:bdd-good}的索引顺序为\{x1,x2,x3,x4,x5,x6,x7,x8\}。找到一个好的索引顺序是一个NP问题,在具体使用中,只能避免最差的情况,并尽量使用使结构最简的索引顺序。

\begin{figure}[!htbp]
	\centering
	\begin{subfigure}[b]{.4\textwidth}
        \centering
        \includegraphics[height=5cm]{Img/BDD_Variable_Ordering_Bad.svg.pdf}
		\caption{索引顺序为\{x1,x3,x5,x7,x2,x4,x6,x8\}}
		\label{fig:bdd-bad}
	\end{subfigure}
	\begin{subfigure}[b]{.4\textwidth}
        \centering
        \includegraphics[height=5cm]{Img/BDD_Variable_Ordering_Good.svg.pdf}
		\caption{索引顺序为\{x1,x2,x3,x4,x5,x6,x7,x8\}}
		\label{fig:bdd-good}
	\end{subfigure}
	\caption{同一布尔函数在不同索引顺序下的结构图\citep{wiki:bdd}}
	\label{fig:bdd-compare}
\end{figure}

TDD与BDD类似,也有索引顺序问题。在工程实现中,尽量使用能简化TDD结构的索引顺序也是能降低计算时间的一种技术路线。
由于寻找最优索引顺序是一个NP问题,因此在工程实现中,只能通过小规模电路上寻求规律,然后在更大规模电路
中应用较优顺寻。
\subsubsection{应用可逆化简决策图}
由于量子状态都在同一希尔伯特空间中。因此作用某些算子后,不同的量子状态可能等价。
当存储算子的资源少于存储状态的资源时,就有可能存储算子表示不同的状态\citep{vinkhuijzen2023limdd}。图\ref{fig:qmdd-example}表示了一个QMDD的例子,应用等价性,可以化简为图\ref{fig:limdd-example}。
TDD也可以应用类似技术,进行进一步化简,从而降低资源要求。
\begin{figure}[!htbp]
    \centering
    \includegraphics[height=8cm]{Img/limdd.pdf}
    \caption{一个QMDD示例}
    \label{fig:qmdd-example}
\end{figure}
\begin{figure}[!htbp]
    \centering
    \includegraphics[height=8cm]{Img/limdd_reduce.pdf}
    \caption{应用等价性化简图\ref{fig:qmdd-example}}
    \label{fig:limdd-example}
\end{figure}
\subsubsection{addition的拆分方案}
\label{addition}
关于常用的量子线路划分方法,
第一种被称为addition\citep{chen2018classical}。将量子电路视为张量网络,首先将一个量子电路C转换成无向图G。G中的每个节点表示量子电路的一个索引,并且如果它们是相同门的输入或输出索引,则在G中连接两个节点。并且当满足以下两个条件之一时输入和输出索引不变:
\begin{itemize}
	\item 是对角线量子门的输入和输出索引;
	\item 是受控门的控制比特位的输入和输出索引。
\end{itemize}
	
该图描述了量子电路的连通性,通过选择图中连通度最大的索引可以对电路进行分割。图\ref{fig:addition}展示了Grover\_3电路图的索引链接图。因此选择图中连通度较大的$x_1^1,x_1^3x_2^1$可以对电路进行较好的划分。
 
\begin{figure}[!htbp]
	\centering
	\includegraphics[height=5cm]{Img/cir_index_graph.pdf}
	\caption{Grover\_3的索引连接图}
	\label{fig:addition}
\end{figure} 
\subsubsection{contraction的拆分方案}\label{contraction}
另一种常用的电路划分方法成为contraction。在这一方法中,将量子电路划分为若干个较小的部分,其收缩等于原始电路。对于两个预设整数参数k1和k2,将电路划分为若干小电路。其中每个小电路涉及最多k1个量子比特,并且与至多跨越不同部件的k2个多比特门相连。图\ref{fig:contraction}展示了对Bit flip电路进行k1=3,k2=2的拆分结果。
\begin{figure}[!htbp]
	\centering
	\includegraphics[height=5cm]{Img/cir_contraction.pdf}
	\caption{对量子电路进行contraction的拆分}
	\label{fig:contraction}
\end{figure} 

\subsection{实验评估方案}
本次研究的主要关注点是时间效率。为了确保研究工作的实用性,在完成研究后需要对计算过程的时间消耗进行评估。为此,将借助不同的量子算法来进行综合评估。在评估过程中,需要选择目前已经被广泛应用和验证的量子算法,如Grover算法、BV算法和QRW算法等。从而保证研究项目的实用性。
通过评估时间效率,可以对研究成果进行客观的评估和比较。此外,为了更好地了解研究成果在当前领域中的地位,还需要与其他相关研究成果进行对比。例如,与基于QMDD的研究和使用TensorFlow等工具的研究进行比较。通过与这些类似研究成果的对比,可以更全面地评估研究成果在时间效率方面的优势和差距。
这样的评估和对比分析将,可以为研究工作提供更多的参考和支持。通过与其他方法的对比,可以更好地理解我们的方法在量子模型检测中的优势和局限性,并为进一步的研究和实际应用提供指导。
\subsection{可行性分析}
首先,本项目在前期已进行了较为充分的文献资料搜集和调研工作,对目前量子模型检测技术现状、相关工作、现有项目的优缺点、验证需求有较为充分的认识,为后期的研究和实验工作打下了坚实的基础。
其次电路拆分方案是最直接降低运行时间的一类方案。而前期已经小范围尝试该方案的时间效率。表\ref{table:split}展示了计算不同测试image computation,应用不同拆分方案的用时。其中basic表示没有使用优化技术,addition表示使用addition优化技术,contraction表示使用contraction优化技术,时间单位为秒。

\begin{table}[!htbp]
	\centering
	\begin{tabular}{@{}llll@{}}
	\toprule
	Benchmark  & basic   & addition & contraction \\ \midrule
	Grover\_20 & 294.65  & 259.87   & 4.39        \\
	QFT\_20    & 1199.21 & 655.19   & 0.12        \\
	QRW\_20    & 341.05  & 218.29   & 14.31       \\
	BV\_100    & 7.36    & 7.43     & 0.41        \\ \bottomrule
	\end{tabular}
	\caption{不同拆分方案在不同类型测试中用时}
	\label{table:split}
\end{table}

