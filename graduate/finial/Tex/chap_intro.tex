
\chapter{引言}

\section{研究背景}
\begin{itemize}
  \item 量子计算的兴起与发展:简要介绍量子计算的历史和发展趋势。
  \item 量子模型检测的重要性:介绍量子模型检测在量子计算领域中的重要性。(todo: 重要性)
  \item TDD技术的引入:描述Tensor Decision Diagrams(TDD)技术的引入背景及其在量子计算中的潜在应用。
\end{itemize}
% //TODO:add some

\subsection{量子计算简介}
% //TODO:add more intro
量子计算机(quantum computer)是一种利用量子比特特性进行计算的一种设备。在量子计算中,量子比特的特殊性质允许其同时处于多种状态,这与经典比特的二进制状态不同。量子计算机的状态空间可以用希尔伯特空间(Hilbert space)\(\mathcal{H}\)表示\citep{nielsen2010quantum},即可以进行内积运算(inner product)的复向量空间。比特状态可以用\(\mathcal{H}\)的向量表示,量子门由\(\mathcal{H}\)上的酉算子(unitary operator)表示。

量子线路(quantum circuit)是一种描述量子计算的模型。在量子线路中,通过量子比特的初始化、应用量子门、测量以及其他可能的操作的序列来构建和执行量子计算任务。量子线路通常从左向右阅读,每个量子门的作用是将输入的量子比特状态转变为输出状态,该过程可以认为是量子门的酉矩阵与输入的量子状态的乘积。
\begin{figure}[!htbp]
    \centering
    \includegraphics[width=.6\textwidth]{Img/example_cir.pdf}
    \caption{一个量子线路的例子}
    \label{fig:example_cir}
\end{figure}

图\ref{fig:example_cir} 所示的量子线路展示了一个具体的量子线路示例。其中有单比特门\(H=\frac{1}{\sqrt2}\left[\begin{matrix}1&1\\1&-1\\\end{matrix}\right],T=\left[\begin{matrix}1&0\\0&e^{-i\pi/4}\\\end{matrix}\right]\),以及双比特门\(CX=\left[\begin{matrix}\begin{matrix}1&0\\0&1\\\end{matrix}&\begin{matrix}0&0\\0&0\\\end{matrix}\\\begin{matrix}0&0\\0&0\\\end{matrix}&\begin{matrix}0&1\\1&0\\\end{matrix}\\\end{matrix}\right]\)。假设该量子线路的初始状态为\(\left|\psi\right\rangle=\left|\psi_1\right\rangle\left|\psi_2\right\rangle\),则输出状态为\(T\otimes H\cdot CX\cdot T\otimes H\cdot\left|\psi\right\rangle\)。

在量子计算机上可以执行各种算法和计算任务,如量子搜索\citep{Grover_1996}、量子因子分解\citep{shor}和量子模拟\citep{Feynman}等。量子计算的潜力在于其能够在某些特定问题上比经典计算机更高效地进行计算,尤其在处理大规模数据和解决复杂问题方面具有潜在优势。需要对这部分深入了解的读者,可以自行阅读\citep{nielsen2010quantum}。
\subsection{跃迁系统}
跃迁系统广泛应用于模型检测中待检测系统的建模,其定义为\citep{baier2008principles}:
\begin{equation}
\mathcal{M}=\{S,Act,\rightarrow,I\}
\end{equation}
其中\(S\)为系统状态集合,\(I\)为系统初态集合,因此满足\(I\subseteq S\)。\(Act\)为系统行为集合。\(\rightarrow\)为系统状态转移关系,即\(\rightarrow\subset S\times Act\times S\)。此外还有\(AP\)为描述系统原子命题。L是标记函数,将状态映射为状态满足的原子命题集合。需要验证的属性\(\varphi\)将表述为命题。


系统的有限路径片段\(\pi\)是一个有限状态序列\(s_0,s_1\ldots s_n\)。\(s_i\)满足\(s_{i-1}\overset{a}{\rightarrow}s_i,a\in Act\),对于所有\(0<i\leq n\),其中\(n\geq 0 \)。无限路径片段\(\pi\)是一个无限状态序列\(s_0,s_1\ldots\),使得对于所有\(i>0\),\(s_{i-1} \overset{a}{\rightarrow}  s_i,a\in Act\)。在路径中\(\pi\left[i\right]=s_i,\pi\left[i\right)=s_i\ldots\)。所有以\(s_0\)为开始的路径,构成了路径集合\(Path\left(s_0\right)\)。

\begin{figure}[!htbp]
    \centering
    \includegraphics[width=.6\textwidth]{Img/map.pdf}
    \caption{一种简化版的售货机跃迁系统}
    \label{fig:transition-system}
\end{figure}

图\ref{fig:transition-system} 所示的跃迁系统展示了一个简化版的售货机模型。在该模型中,用户投入硬币,进行选择后就可以得到苏打水或者啤酒。在该例子中,系统状态\(S=\{pay,select,soda,beer\}\),系统初态\(I=pay\)。
系统行为\(Act=\{insert\_coin,\tau,get\_soda,get\_beer\}\),其中\(\tau\)表示立即行动符号。转移关系图中已经展示。原子命题可取\(AP=\{paid,drink\}\)。因此\(L\left( pay \right)=\{\varnothing\}\),\(L\left(soda\right)=L\left(beer\right)=\{paid,drink\}\),\(L\left(select\right)=\{paid\}\)。系统的一个路径是\(\pi=pay\ select\ soda\ pay\ selsect\ \ldots\)。此时\(\pi\left[1\right]=slect,\pi\left[1\right)=select\quad soda\quad pay\quad selsect\ldots\)。同时该路径满足\(\pi\in Path\left(pay\right)\)。
量子模型检测的跃迁系统类似。区别在于状态空间用\(\mathcal{H}\),转移关系用酉矩阵。一个量子自动机定义如下:
\begin{align}
    \mathcal{M}=\{\mathcal{H},Act,\{U_\alpha,\alpha\in Act\},\mathcal{H}_0\}
\end{align}
下面介绍模型检测中的使用的时序逻辑。

\subsection{时序逻辑的验证}
在量子模型检测中,与经典模型检测一样使用时序逻辑指定待验证的属性\(\varphi\)。时序逻辑命题的运算符有两类\citep{goranko_2023}。状态命题公式(State formulas):\(\varphi ::=a\left|\exists\varphi\right|\forall \varphi\left|\lnot\varphi\right|\varphi\land\psi\),其中\(a\in AP\)。以及路径命题公式(Path formulas):\(\varphi\Colon=O\varphi|\varphi U\psi\)。给定模型的一个状态为\(s\),路径为\(\pi\),则具体满足条件分别如下:
\begin{itemize}
    \item \(s\models a,iff \L\left(s\right)\models a\)
    \item \(s\models\exists\varphi,iff\ \pi\models\varphi\)对一些\(\pi\in Path\left(s\right)\)
    \item \(s\models\forall\varphi,iff\ \pi\models\varphi\)对所有\(π\in Paths\)
    \item \(s\models\lnot\varphi,iff\ s\nvDash\varphi\)
    \item \(s\models\varphi\land\psi,iff\ s\models\varphi\ and\ s\models\psi\)
    \item \(\pi\models O\varphi,iff\ \pi\left[1\right]\models\varphi\)
    \item \(\pi\models\varphi U\psi,iff\ \exists j\geq0\).\(\pi\left[j\right)\models\psi\) 同时对所有\(0\le i<j\)有\(\pi\left[i\right)\models\varphi\)
\end{itemize}


图\ref{fig:path-formula-basic} 展示了两种路径命题公式的直观示意图。


\begin{figure}[!htbp]
    \centering
    \begin{subfigure}[b]{0.8\textwidth}
        \centering
        \includegraphics[width=\textwidth]{Img/path_for_Oa.pdf}
    \end{subfigure}
    \\
    \begin{subfigure}{0.8\textwidth}
        \centering
        \includegraphics[width=\textwidth]{Img/path_for_aUb.pdf}
    \end{subfigure}
    \caption{$\pi\models O a $与 $\pi\models a U b$的图示}
    \label{fig:path-formula-basic}
\end{figure}
在模型检测中,有三类比较重要的可达性问题,分别是可达性、持续可达性以及重复可达性。过程中主要涉及以下路径命题公式:\(\lozenge\) 表示最终(eventually),\(\square\)表示总是(always),\(\lozenge\square\)表示总是最终(always eventually),\(\square\lozenge\)表示最终总是(eventually always)。其中\(\lozenge\)和\(\square\)具体定义为:
\begin{itemize}
    \item \(\lozenge\varphi\overset{\text{def} }{=} \text{True}U\varphi\)
    \item \(\square\varphi\overset{\text{def} }{=} \neg\lozenge\neg\varphi\)
\end{itemize}
图\ref{fig:path-formula}展示了这两种基本路径命题公式的直观示意图。
\begin{figure}[!htbp]
    \centering
    \begin{subfigure}[b]{0.8\textwidth}
        \centering
        \includegraphics[width=\textwidth]{Img/path_for_Dia.pdf}
    \end{subfigure}
    \\
    \begin{subfigure}{0.8\textwidth}
        \centering
        \includegraphics[width=\textwidth]{Img/path_for_SQa.pdf}
    \end{subfigure}
    \caption{$\pi\models\lozenge a$与 $\pi\models\square a$的图示}
    \label{fig:path-formula}
\end{figure}

具体的可满足条件为:
\begin{itemize}
    \item \(\pi\models\lozenge\varphi,iff\exists j\ge0.\pi[j)\models\varphi\)
    \item \(\pi\models\square\varphi,iff\forall j\ge 0.\pi[j)\models\varphi\)
    \item \(\pi\models\lozenge\square\varphi,iff\exists i\ge 0.\forall j\ge i,\pi[j)\models\varphi\)
    \item \(\pi\models\square\lozenge\varphi,iff\forall i\ge 0.\exists j\ge i,\pi[j)\models\varphi\)
\end{itemize}
基于此三种可达性问题定义分别如下:
\begin{itemize}
    \item 可达性:\( Pr^{\mathcal{M}}(s \models \lozenge G) = Pr^M(\pi \models \lozenge G : \pi \in \text{Paths}(s))\)
    \item 持续可达性:\( Pr^{\mathcal{M}}(s \models \lozenge \square G) = Pr^M(\pi \models \lozenge \square G : \pi \in \text{Paths}(s))\)
    \item 重复可达性:\( Pr^{\mathcal{M}}(s \models\square \lozenge G) = Pr^M(\pi \models \square\lozenge G : \pi \in \text{Paths}(s))\)
\end{itemize}

\subsection{量子模型检测}
目前量子的模型检测,主要使用Birkhoff-von Neumann Quantum Logic来描述量子系统的性质\citep{birkhoff1987logic}。Birkhoff-von Neumann量子逻辑是一种非经典逻辑,用于描述量子力学中事件的逻辑结构。它由 Birkhoff 和 von Neumann 在 1936 年首次提出。在量子逻辑中,命题的集合不再形成布尔代数,而是形成一个投影算子的正交完备格,这与传统的逻辑系统不同。

在 Birkhoff-von Neumann 量子逻辑中,量子系统的状态可以由希尔伯特空间(Hilbert space)来描述,每个量子命题对应希尔伯特空间的一个闭子空间。对于系统的状态 \(|\psi\rangle\),如果它属于某个特定的闭子空间 \( \mathcal{X} \),我们可以说这个命题是真的。

例如,考虑以下量子逻辑命题:

\begin{itemize}
\item 命题 \( \mathcal{X} \):在时间 \( t \) 时,量子粒子的位置 \( x \) 坐标在区间 \( [a, b] \) 内。
\item 命题 \( \mathcal{Y} \):在时间 \( t \) 时,量子粒子的动量 \( y \) 坐标在区间 \( [a, b] \) 内。
\end{itemize}

这些命题 \( \mathcal{X} \) 和 \( \mathcal{Y} \) 可以通过粒子的状态希尔伯特空间的特定子空间来表示。

在数学上,这种逻辑结构可以使用格理论(lattice theory)来描述,其中格中的元素对应于量子事件,格的操作则对应于逻辑运算。在确定了原子命题后,需要引入连接词,这些连接词可以用来构建更复杂的命题,以描述量子系统的复杂属性。在语义上,这些可以被视为在希尔伯特空间$\mathcal{H}$的一个子空间 \(S(\mathcal{H})\) 中的代数操作。
具体如下:
\begin{itemize}
    \item 子空间之间的包含关系 \( \subseteq \) 在 \(S(\mathcal{H})\) 中是一个偏序关系,它可以理解为量子逻辑的蕴含(元逻辑)。
    \item 一个子空间 \( \mathcal{X} \) 的正交补 \( \mathcal{X}^\perp \) 在量子逻辑中用作否定的解释。
    \item \(S(\mathcal{H})\) 对交集是封闭的,即对于 \(S(\mathcal{H})\) 中的任何元素族 \( \{\mathcal{X}_i\} \),都有$\bigcap_{i} \mathcal{X}_{i} \in \mathcal{S}(\mathcal{H})$。在量子逻辑中用于表示合取。
    \item 对于一组子空间 \(\{\mathcal{X}_i\}\),它们的并集定义为
    \(
    \bigvee_i \mathcal{X}_i = \text{span} \left( \bigcup_i \mathcal{X}_i \right).
    \)
    。在量子逻辑中,析取被解释为并集。
\end{itemize}

\( (S(\mathcal{H}), \cap, \vee, \perp) \) 构成一个正交模糊格,\( \subseteq \) 是其排序,这是 Birkhoff–von Neumann 量子逻辑的代数模型。

在实际应用中,通常只选择 \( S(\mathcal{H}) \) 的一个子集 AP 作为原子命题的集合。AP 中的元素可以被认为是真正关心的那些命题,而其他的可能是不相关的。出于算法目的,通常假设 AP 是可数的甚至是有限的 \( S(\mathcal{H}) \) 的子集,而不是 \( S(\mathcal{H}) \) 本身,因为 \( S(\mathcal{H}) \) 是不可数无限的。

因此给定一组原子命题集 AP,对于 \( \mathcal{X} \in S(\mathcal{H}) \),如果状态 \(|\psi\rangle\) 满足集合中所有命题的交集,我们说 \(|\psi\rangle\) 满足 \(\mathcal{X}\)。
所以在量子模型检测中,计算系统状态是一件非常重要的事情。而量子计算中状态空间\(\mathcal{H}\)维度\(dim\left(\mathcal{H}\right)=2^n\),其中n为比特数量。即状态空间维数随比特个数指数级增长。这为计算量子系统状态带来了困难。

借助更好的数据结构,可以用更少的资源表示量子状态以及量子线路,并计算最终的结果。比如TDD给出了量子电路的紧凑表示,提供了一种方便的实现张量网络各种操作的方式,这些操作对于模拟量子物理系统非常重要。图\ref{fig:P}展示了一个矩阵和TDD形式,其中TDD中的实线表示高边,虚线表示低边。可以明显看到TDD的结构更紧凑。
 	 
\begin{figure}[!htbp]
    \begin{subfigure}[c]{0.4\textwidth}
        \centering
        \includegraphics[width=1.2\textwidth]{Img/matrix_of_tdd.pdf}
        \caption{矩阵$P$的矩阵形式}
        \label{fig:mat_P}
    \end{subfigure}
    \begin{subfigure}[c]{0.4\textwidth}
        \centering
        \includegraphics[height=7cm]{Img/tdd_ex.pdf}
        \caption{矩阵$P$的TDD形式}
        \label{fig:tdd_P}
    \end{subfigure}
    \caption{应用TDD可以减少存储特殊矩阵的资源}
    \label{fig:P}
\end{figure}

TDD特别适用于实现可达性分析和模型检查算法。这是因为基于BDD的模型检查算法中使用的许多优化技术可以推广到收缩量子电路张量网络上\citep{Chaki_2018}。这些为应用TDD解决量子模型检测问题提供了可能的方案。
% //TODO: modify structure. here is the backgroud

\section{研究目的和重要性}
\begin{itemize}
  \item 研究动机:研究课题的原因,包括现有研究中存在的问题或空白。
  \item 研究目标:列出研究实现的具体目标。
  \item 预期影响:讨论研究对于学术界和实际应用可能产生的影响。
\end{itemize}
% // TODO: add importance

\section{论文结构概述}
\begin{itemize}
  \item 章节安排:简单概述论文的结构和每一章的主要内容。
  \item 研究方法概述:预览将采用的研究方法和技术路线。
  \item 论文贡献:简述研究对现有知识体系的补充或扩展。
\end{itemize}
% //TODO: add structure