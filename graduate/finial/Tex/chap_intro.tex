\chapter{绪论}
由于量子计算提供了相比经典计算,对特定计算问题的指数级加速能力,而成为最近几年广受关注的新兴领域。而随着量子计算的快速发展,需要严格验证的方法来确保量子系统和量子算法的正确性和可靠性。
同时物理量子计算机中快速增长的量子比特数量,对验证方法提出了更自动化的需求。

模型检测(Model Checking)是一种自动化形式方法,用于验证有限状态系统的性质。模型检测最初由 E. M. Clarke 和 E. A. Emerson 提出\citep{Emerson_1980,Clarke,Clarke_1986},如今已广泛应用于经典计算机的软件和硬件设计。例如,在嵌入式系统中,可以使用 UML (Unified Model Language)活动图来验证硬件是否符合规范\citep{Grobelna_2015}。
类似地,在量子计算领域,量子模型检测也可以应用于验证过程,提供根据指定属性分析量子系统的行为。 本研究论文重点关注量子模型检测领域内的可达性分析方面,采用张量决策图(TDD,Tensor Decision Diagrams)作为解决量子系统固有复杂性的新方法。


\section{研究背景与研究问题}
\subsection{研究背景}
“量子”概念最早由Max Planck在1900年提出,开启了量子物理学的先河\citep{planck1901law}。20世纪20年代,量子理论的发展为波粒二象性提供了理论基础,奠定了现代物理学的基石
\citep{bhatta2020plurality}。
经历两次世界大战期间,不仅促进了计算理论的飞速发展,如图灵机的提出\citep{hodges2014alan},也加速了量子理论在核物理学等领域的应用\citep{maartensson2006manhattan}。

随着物理学家将量子力学模型应用于计算问题并将经典比特替换为量子比特,量子力学和计算机科学领域开始融合。 1980年,Paul Benioff基于量子理论提出了量子图灵机的概念,为量子计算奠定了理论基础\citep{benioff1980computer}。随着经典计算机在模拟量子动力学过程中遇到的计算限制,Yuri Manin和Richard Feynman分别提出,利用量子现象的计算模型可能更为高效[10][11] [12]。此外, Charles Bennett 和 Gilles Brassard在1984年提出量子密钥分发,进一步展示了量子理论在加强信息安全性方面的潜力 [13][14]。


进入90年代,量子计算的发展迎来了新的突破。Peter Shor的算法证明了量子计算机在破解传统加密方面的潜力 [] 之后Grover的算法为解决非结构化搜索问题提供了量子加速的可能[] 。同时,Seth Lloyd 证明量子计算机可以模拟量子系统,而无需经典模拟中存在的指数开销,[23] 验证了费曼 1982 年的猜想。 [24]

多年来,\textcolor{red}{实验学家利用捕获离子和超导体构建了小型量子计算机。} [25] 1998年,两个量子比特的量子计算机证明了该技术的可行性,[26][27]以及随后的实验增加了量子比特的数量并降低了错误率。[25] 2019 年,Google AI 和 NASA 宣布他们已经通过 54 量子位机器实现了量子霸权,执行了任何经典计算机都无法完成的计算。[28][29][30] 然而,这一说法的有效性仍在积极研究中。[31][32]但这样的说法极大增强了人们对量子计算机的热情,促进了各国政府与私人机构对该领域的投资。

近几年,物理量子计算机。2021 年,IBM 公司宣称已经开发出了能运行127个量子比特的超导
量子处理器“鹰”。英特尔公司于2017年研制出了49个量子比特的量子计算测
试芯片“TangleLake”。Google 公司于2019年制造了53个量子比特的超导量子
计算原型机“悬铃木”[15]。IonQ公司于2021年打造了拥有20个量子比特的
离子阱量子计算机“IonQAria”。微软公司的量子团队目前已经证实拓扑量子比
特的物理机理,并正在构建一个基于拓扑量子比特的工业规模量子计算机。2020
年12月4日,中国科学技术大学潘建伟团队宣布成功构建76个光子的量子计
算原型机“九章”。2021年5月7日,潘建伟团队宣布研制了62个量子比特可
编程的超导量子计算原型机“祖冲之”。2021年10月,“九章二号”操控的光子
数从76个增加到113个,祖冲之二号”操控的超导比特数从62个上升至66个。

量子计算领域的快速发展,带来了对应的验证需求。经典的形式化验证技术,如模型检测(Model Checking)和定理证明(Theorem Proving),虽然成熟但面临新挑战。模型检测技术通过逻辑公式检测状态迁移系统的行为,虽然自动化程度高,但存在状态空间爆炸问题[]。定理证明通过数学逻辑来验证系统模型,尽管在处理复杂问题时具有优势,但仍需人工干预[]。
\subsection{研究的问题}
\begin{itemize}
  \item 可达性分析
  \item 模型检验的挑战
\end{itemize}
% //TODO: add key problem here

在量子模型检测中,存在状态空间爆炸的问题。量子计算中状态空间\(\mathcal{H}\)维度\(dim\left(\mathcal{H}\right)=2^n\),其中n为比特数量。因此状态空间维数随比特个数指数级增长。这为计算量子系统状态带来了困难。

而借助更好的数据结构,可以用更少的资源表示量子状态以及量子线路,并计算最终的结果。比如TDD给出了量子电路的紧凑表示,提供了一种方便的实现张量网络各种操作的方式,这些操作对于模拟量子物理系统非常重要。图\ref{fig:P}展示了一个矩阵和TDD形式,其中TDD中的实线表示高边,虚线表示低边。可以明显看到TDD的结构更紧凑。
 	 
\begin{figure}[!htbp]
    \begin{subfigure}[c]{0.4\textwidth}
        \centering
        \includegraphics[width=1.2\textwidth]{Img/matrix_of_tdd.pdf}
        \caption{矩阵$P$的矩阵形式}
        \label{fig:mat_P}
    \end{subfigure}
    \begin{subfigure}[c]{0.4\textwidth}
        \centering
        \includegraphics[height=7cm]{Img/tdd_ex.pdf}
        \caption{矩阵$P$的TDD形式}
        \label{fig:tdd_P}
    \end{subfigure}
    \caption{应用TDD可以减少存储特殊矩阵的资源}
    \label{fig:P}
\end{figure}

TDD特别适用于实现可达性分析和模型检查算法。这是因为基于BDD的模型检查算法中使用的许多优化技术可以推广到收缩量子电路张量网络上\citep{Chaki_2018}。这些为应用TDD解决量子模型检测问题提供了可能的方案。


而在模型检测中,有三类比较重要的可达性问题,分别是可达性、持续可达性以及重复可达性。因此本次研究的主要目的是借助TDD数据结构,构建能快速计算量子模型检测中可达问题的方案。

\section{研究现状与相关工作}

随着量子计算硬件规模的快速增长,量子电路的验证成为一个重要问题。开始的研究主要集中在BDD在量子计算下的推广算法,如量子信息决策图(Quantum Information Decision Diagram,或QuIDD)\citep{Viamontes_2003},量子多值决策图(Quantum multiple-valued Decision Diagram,或QMDD)\citep{Seiter_2013}等,从而对组合式量子电路进行等效性检测。显然,随着越来越复杂的物理可实现化的硬件出现,将会出现更加复杂的,更加针对于的,新的验证问题。比如量子存储\citep{Kerckhoff_2010},量子反馈网络\citep{Gough_2008},重复至成功(Repeated Until Success,或RUS)的量子电路\citep{Bocharov_2015}。量子模型检测可以为量子电路的验证提供了更多思路。

量子系统模型检测的早期工作旨在验证量子通信协议\citep{Gay,BALTAZAR_2008,davidson2012model}。后来还有针对分析和验证量子程序的应用\citep{ying2016foundations},比如量子自动机\citep{ying2014model}、量子马尔可夫链\citep{Ying_2013}和超算符值马尔可夫链\citep{feng2013model}的模型检测技术。然而,在这些量子模型检测技术与它们在验证量子电路方面实际应用之间存在巨大差距仍需填补。TDD作为新的数据结构,极大加快了计算过程,有可能深化二者的联系,加快实际应用的出现。
量子系统的模型检测尚处于起步阶段,尤其是在量子硬件验证方面。
未来量子模型检测的最重要目标是寻找一类更简单易于检测的属性。过去的研究追求的是普遍性,即只针对检测量子系统的一般可达性和时间逻辑属性。然而,为了实现这一目标,模型检测的效率非常低,且仅适用于非常小规模和深度较小的量子电路。因此,需要确定一类更简单易于检测的属性,以便当前的量子模型检测工具可以高效地进行检测\citep{ying2021model}。
这需要更多在时序逻辑上的研究工作。但同时更实用的模型检测工具也能在研究相关属性起到一定帮助。
\subsection{国外研究进展}
\subsection{国内研究进展}
% //TODO: add new paper

% \begin{itemize}
%   \item 量子计算的兴起与发展:简要介绍量子计算的历史和发展趋势。
%   \item 量子模型检测的重要性:介绍量子模型检测在量子计算领域中的重要性。(todo: 重要性)
%   \item TDD技术的引入:描述Tensor Decision Diagrams(TDD)技术的引入背景及其在量子计算中的潜在应用。
% \end{itemize}
% //TODO:add some


\section{研究目的和重要性}
% \begin{itemize}
%   \item 研究动机:研究课题的原因,包括现有研究中存在的问题或空白。
%   \item 研究目标:列出研究实现的具体目标。
%   \item 预期影响:讨论研究对于学术界和实际应用可能产生的影响。
% \end{itemize}
% // TODO: add importance
随着量子计算硬件规模的快速增长,量子电路的验证成为一个重要问题。开始的研究主要集中在BDD在量子计算下的推广算法,如量子信息决策图(Quantum Information Decision Diagram,或QuIDD)\citep{Viamontes_2003},量子多值决策图(Quantum multiple-valued Decision Diagram,或QMDD)\citep{Seiter_2013}等,从而对组合式量子电路进行等效性检测。显然,随着越来越复杂的物理可实现化的硬件出现,将会出现更加复杂的,更加针对于的,新的验证问题。比如量子存储\citep{Kerckhoff_2010},量子反馈网络\citep{Gough_2008},RUS量子电路\citep{Bocharov_2015}。量子模型检测可以为量子电路的验证提供了更多思路。

量子系统模型检测的早期工作旨在验证量子通信协议\citep{Gay,BALTAZAR_2008,davidson2012model}。后来还有针对分析和验证量子程序的应用\citep{ying2016foundations},比如量子自动机\citep{ying2014model}、量子马尔可夫链\citep{Ying_2013}和超算符值马尔可夫链\citep{feng2013model}的模型检测技术。然而,在这些量子模型检测技术与它们在验证量子电路方面实际应用之间存在巨大差距仍需填补。TDD作为新的数据结构,极大加快了计算过程,有可能深化二者的联系,加快实际应用的出现。

% \begin{itemize}
%   \item QDA方面的验证
% \end{itemize}
% //TODO: add usage here
而在量子计算中的自动设计领域(Quantum Design Automation, QDA)中有大量验证问题,比如对同一算法的不同仿真(synthesis)方案是否等价。比如一类RUS(Repeat Until Success)电路等价性问题。
图\ref{fig:rus-equal}展示了两个RUS量子电路。当电路中测量门结果为$00$时,第三个比特的输出态均为$\left(I+2iZ\right)/ \sqrt 5 |\psi\rangle$。这样的等价性检验任务,可以转换为模型检测中的可达空间是否一致。
\begin{figure}[!htbp]
	\centering
	\begin{subfigure}[b]{0.4\textwidth}
        \centering
        \includegraphics[height=3cm]{Img/rus_s.pdf}
	\end{subfigure}
	\qquad
	\begin{subfigure}[b]{0.4\textwidth}
        \centering
        \includegraphics[height=3cm]{Img/rus_T.pdf}
	\end{subfigure}
	\caption{两个等价的RUS电路\citep{Bocharov_2015}}
	\label{fig:rus-equal}
\end{figure}
未来能够将本研究成果应用于不同的 synthesis 算法的验证等价性中,
从而加速量子算法在实际量子计算机上的实现,从而在量子计算的实际应用中发挥更大的作用。
\section{论文结构概述}
\begin{itemize}
  \item 章节安排:简单概述论文的结构和每一章的主要内容。
  \item 研究方法概述:预览将采用的研究方法和技术路线。
  \item 论文贡献:简述研究对现有知识体系的补充或扩展。
\end{itemize}
% //TODO: add structure