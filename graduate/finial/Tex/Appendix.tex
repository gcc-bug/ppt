\thispagestyle{appendixheader}
\stepcounter{app}
\setcounter{app_fig}{1}
\setcounter{app_tab}{1}
\setcounter{equation}{0}
\renewcommand\theequation{附\arabic{app}-\arabic{equation}}
% \renewcommand\theequation{\Alph{app}.\arabic{equation}}
\renewcommand\chaptername{附录}
\renewcommand\chaptername{Appendix} 
\renewcommand\thechapter{附录\zhnum{app}} 

\setcounter{chapter}{0}
\setcounter{section}{0}
\chapter{部分优化方案的python实现}\label{chap:app1}{
第\ref{sec-optimize}节介绍了针对基于TDD的量子模型检测方法的一些优化策略。第\ref{sec-ex}章节证实了量子拆分技术能够有效降低计算资源消耗,是一种非常实用的优化方案。本附录将以Python为例,介绍这些优化方法的具体实现。

在代码实现中,变量Ts一般表示一个量子迁移系统,其中包含以下变量:
\begin{itemize}
    \item Ts.initial\_states是一个列表,包含所有量子迁移系统的初态。
    \item Ts.operators是一个列表,按顺序包含所有量子线路的操作。
\end{itemize}
基于这样 的量子迁移系统设计,可以设计代码段1。
此代码段核心使用张量收缩技术,以计算量子迁移系统的一次状态转移。具体来说,就是基于系统的初始状态来预测其下一状态空间。这里,simulation函数负责执行量子操作的TDD与量子态的收缩操作,而join函数则用于将得到的新量子状态并入当前状态子空间中。
因此,代码逻辑首先从系统的初始态出发,构建一个初始的状态子空间。接着,它遍历所有量子操作,对每一个初始态执行操作,通过simulation函数计算操作后的结果,并利用join函数将这些新状态并入到状态子空间中,以完成系统下一步状态空间的构建。
\begin{lstlisting}[language=Python, caption={利用TDD收缩直接计算一步迁移\label{code-image}}]
def image(Ts):
    sp= generate_initial_space(Ts.initial_states)
    for op in Ts.operators:
        for phi in Ts.initial_states:
            res=simulation(Ts.operators[op],phi)
            sp=join(sp,res)
    return sp
\end{lstlisting}

类似地,可以设计代码段2中的addition优化方案。其中tn为表示量子线路的量子张量网络。不同在于其中的Ts的operators将会在运行中生成。
其他运行逻辑与代码段1类似,码段2的函数首先生成系统的初始状态子空间。然后,对系统中的每个量子操作,通过函数addition\_partition应用分区优化tn的索引,从而得到Ts的量子操作。这一步调整了量子操作,以便提高执行效率。
在应用了优化后的量子操作进行状态转移的计算过程中,函数对每个初始态进行遍历。对于每个量子操作,它初始化一个表示恒等操作的TDD,作为累加的基础。接着,通过遍历量子操作的所有部分,使用simulation函数计算每部分的效果,并利用add函数将这些TDD累加到res中。由于应用索引划分了线路,相当于对张量网络进行了切片。因此这里需要使用TDD的加法,即add函数。这样最后的res就是对应量子初态在执行所有量子操作后的量子状态。最后通过join函数将该量子状态的结果并入到状态子空间中。
其中的函数addition\_partition如代码段3所示。
\begin{lstlisting}[language=Python, caption=对量子线路应用addition优化方案计算一步迁移]
def image_add_par(tn,Ts,k=0):
    sp= generate_initial_space(Ts.initial_states)

    Ts.operators[op]=addition_partition(tn,k)

    for op in Ts.operators:
        for phi in Ts.initial_states:
            res=get_identity_tdd()
            for p in Ts.operators[op]:
                temp=simulation(p,phi)
                res=add(res,temp)
            sp=join(sp,res)
    return sp
\end{lstlisting}
\begin{lstlisting}[language=Python, caption={addition方案中的电路划分函数}]
def addition_partition(tn,k=0):
    """partition a tensor network tn according to k nodes with biggest degree"""

    lin_graph.add_nodes_from(tn.index_set)

    deg=[lin_graph.degree(key) for key in lin_graph.nodes]

    big_degres=[]
    node_list=list(lin_graph.nodes)

    t = k
    while t:
        biggest_degree=max(deg)
        if biggest_degree<=0:
            break
        idx=deg.index(biggest_degree)
        big_degres.append(node_list[idx])
        deg[idx]=0
        t-=1

    dd_list=[]
    for t in range(2**k):
        b=list(bin(t)[2:])
        b.reverse()
        temp_tn=copy.copy(tn)
        temp_tn.tensors=[]
        for tensor in tn.tensors:
            temp_ts=copy.copy(tensor)
            for idx in tensor.index_set:
                if idx.key in big_degres:
                    if big_degres.index(idx.key)< len(b) and b[big_degres.index(idx.key)]=='1':
                        temp_ts.dd=cont(temp_ts.tdd(),Tensor(ket1,[idx]).tdd())
                    else:
                        temp_ts.dd=cont(temp_ts.tdd(),Tensor(ket0,[idx]).tdd())
            temp_tn.tensors.append(temp_ts)
        dd_list.append(temp_tn.cont())

    return dd_list
\end{lstlisting}


同样也可以设计对量子线路应用contraction的优化方案,具体如代码段4。该函数利用两个参数k1和k2来调节收缩分区的策略,以达到更优的计算性能。函数的输入的tn和Ts的情况和addition方案相同。tn也是表示量子线路的量子张量网络。Ts的operators同样会在运行中生成。
初始步骤与之前类似,函数从生成系统初始状态的子空间开始。不同在于对于量子系统中定义的每个量子操作,根据参数k1和k2,使用contraction\_partition函数对其进行分区。在这一步中Ts中的量子操作将会按块划分,因此之后的Ts中的operators是一个列表,包含所有在该块的量子操作。
这一步骤的目的是重新组织量子操作的结构,使得后续的计算过程更加高效。
在经过优化的量子操作上执行状态转移计算时,该函数遍历所有初始状态。对于每一个量子操作,它从当前的量子状态开始,逐步应用经过收缩优化的量子操作的各个部分。这里的simulation函数用于计算经过每一步操作后的结果。每次操作后得到的新状态通过join函数并入到状态子空间中。其中的函数contraction\_partition如代码段5所示。
\begin{lstlisting}[language=Python, caption=对量子线路应用contraction优化方案计算一步迁移]
def image_cont_par(tn,Ts,k1=0,k2=0):
    sp= generate_initial_space(Ts.initial_states)

    Ts.operators=contraction_partition(tn,k1,k2)

    for op in Ts.operators:
        for phi in Ts.initial_states:
            res=phi
            for p in Ts.operators[op]:
                res=simulation(p,res)
            sp=join(sp,res)
    return sp
\end{lstlisting}

\begin{lstlisting}[language=Python, caption={contraction方案中的电路划分函数}]
def contraction_partition(tn,k1=0,k2=0):
    dd_list=[]

    cir_par=circuit_partition(tn,k1,k2)
    for ver in cir_par:
        for hor in ver:
            tdd,node = TensorNetwork(hor).cont()
            dd_list.append(tdd)

    return dd_list

def circuit_partition(tn,k1=0,k2=0):
    """The first partition scheme;
    cx_max is the number of CNOTs allowed to be cut
    """

    num_qubit=tn.qubits_num

    if k1==0:
        blocks_num=2
        k1=np.ceil(num_qubit/2)
    else:
        blocks_num=int(np.ceil(num_qubit/k1))
    res=[[[] for _ in range(blocks_num)]]

    if k2==0:
        cx_max=num_qubit//2
    else:
        cx_max=k2

    cx_num=0
    level=0

    for tensor in tn.tensors:
        qubit_list = tensor.qubits
        mi=int(min(qubit_list)//k1)
        ma=int(max(qubit_list)//k1)
        if mi==ma:
            res[level][mi].append(tensor)
        else:
            cx_num+=1
            if cx_num<=cx_max:
                res[level][int(qubit_list[-1]//k1)].append(tensor)
            else:
                level+=1
                res.append([])
                for k in range(blocks_num):
                    res[level].append([])
                res[level][int(qubit_list[-1]//k1)].append(tensor)
                cx_num=1

    return res
\end{lstlisting}
% \begin{lstlisting}[language=Python, caption=对TDD运用窗函数划分]
% def tdd_partition(tdd: TDD, windows: list[dict]) -> list[TDD]:
%     partitions = [partition(tdd.node, window) for window in windows]
%     for partitioned_tdd in partitions:
%         G.tdd_initial(partitioned_tdd,tdd)
%         if partitioned_tdd.weight:
%             partitioned_tdd.weight = tdd.weight

%     return partitions
% \end{lstlisting}
% \begin{lstlisting}[language=Python, caption=对TDD进行分割]
% def tdd_split(tdd:TDD, split_point:int)-> list[TDD,TDD]:
%     B= split_before(tdd.node, split_point)
%     A= split_after(tdd.node, split_point)
%     return generate_tdd_index(B, A, tdd, split_point)
% \end{lstlisting}
\thispagestyle{appendixheader}
}