%---------------------------------------------------------------------------%
%-                                                                         -%
%-                           LaTeX Template                                -%
%-                                                                         -%
%---------------------------------------------------------------------------%
%- Copyright (C) Huangrui Mo <huangrui.mo@gmail.com> 
%- This is free software: you can redistribute it and/or modify it
%- under the terms of the GNU General Public License as published by
%- the Free Software Foundation, either version 3 of the License, or
%- (at your option) any later version.
%---------------------------------------------------------------------------%
%->> Document class declaration
%---------------------------------------------------------------------------%
\documentclass[twoside]{Style/ucasthesis}%
%- Multiple optional arguments:
%- [<oneside|twoside|print>]% oneside eprint, twoside eprint, or paper print
%- [fontset=<adobe|none|...>]% specify font set instead of automatic detection
%- [scheme=plain]% thesis writing of international students
%- [draftversion]% show draft version information
%- [standard options for ctex book class: draft|paper size|font size|...]%
%---------------------------------------------------------------------------%
%->> Document settings
%---------------------------------------------------------------------------%
\usepackage{caption}
\usepackage[super,numbers, list,table,math]{Style/artratex}
%- usage: \usepackage[option1,option2,...,optionN]{artratex}
%- Multiple optional arguments:
%- [bibtex|biber]% set bibliography processor and package
%- [<numbers|super|authoryear|alpha>]% set citation and reference style
%- <numbers>: textual: Jones [1]; parenthetical: [1]
%- <super>: textual: Jones superscript [1]; parenthetical: superscript [1]
%- <authoryear>: textual: Jones (1995); parenthetical: (Jones, 1995)
%- <alpha>: textual: not available; parenthetical: [Jon95]
%- [geometry]% reconfigure page layout via geometry package
%- [lscape]% provide landscape layout environment
%- [xhf]% disable header and footer via fancyhdr package
%- [color]% provide color support via xcolor package
%- [background]% enable page background
%- [tikz]% provide complex diagrams via tikz package
%- [table]% provide complex tables via ctable package
%- [list]% provide enhanced list environments for algorithm and coding
%- [math]% enable some extra math packages
%- [xlink]% disable link colors
\usepackage{Style/artracom}% user defined commands
%---------------------------------------------------------------------------%
%->> Document inclusion
%---------------------------------------------------------------------------%
%\includeonly{Tex/Chap_1,...,Tex/Chap_N}% selected files compilation
%---------------------------------------------------------------------------%
%->> Document content
%---------------------------------------------------------------------------%
%-
%-> Titlepage information
%-

%---------------------------------------------------------------------------%
%->> Titlepage information
%---------------------------------------------------------------------------%
%-
%-> 中文封面信息
%-
\confidential{}% 密级:涉密论文或延迟公开论文填写
\schoollogo[scale=0.095]{ucas_logo}% 校徽
\title{基于张量决策图的量子模型检测中的可达性分析}% 题目
% //TODO:Bind

% \author{XX}% 作者姓名
\author{高丁超}% 作者姓名
\advisor{应圣钢~副研究员\\中国科学院软件研究所}% 指导教师:姓名 专业技术职务 工作单位
% \advisor{指导教师一\\指导教师二\\指导教师三}% 多行指导教师示例
\degree{硕士}% 学位:学士、硕士、博士
\degreetype{工学}% 学位类别:理学、工学、工程、医学等
\major{计算机科学与技术}% 一级/二级学科专业名称,领域名称需要与学籍信息一致
% //TODO:bind
\institute{中国科学院软件研究所}% 院系名称
\date{2024~年~6~月}% 毕业日期:夏季为6月、冬季为12月
%-
%-> 英文封面信息
%-
\TITLE{Reachability Analysis in Quantum Model Checking Based on Tensor Decision Diagrams}% 论文英文题目
\AUTHOR{Dingchao Gao}% 论文作者
\ADVISOR{Supervisor: Shenggang Ying}% 指导教师
% //TODO:Bind
% \AUTHOR{XX}% 论文作者
\DEGREE{Master}% 学位:Bachelor, Master, Doctor, Postdoctor。封面据英文学位名称自动切换,需确保拼写准确
\DEGREETYPE{Engineering}% 学位类别:Philosophy, Natural Science, Engineering, Economics, Agriculture 等
\MAJOR{Computer Science and Technology}% 二级学科专业名称
% //TODO:bind
\INSTITUTE{Institute of Software, Chinese Academy of Sciences}% 院系名称
\DATE{June, 2024}% 毕业日期:夏季为June、冬季为December
%---------------------------------------------------------------------------%
%
\usepackage{enumitem}
\usepackage{booktabs}
\newlist{myen}{enumerate}{1} % Defines a new list style named mystyle
\setlist[myen,1]{label=\alph*), leftmargin=*}
\usepackage{multirow}
\usepackage{braket} % Provides \ket and \bra commands
\renewcommand{\algorithmicrequire}{\textbf{输入:}}
\renewcommand{\algorithmicensure}{\textbf{输出:}}
\renewcommand{\algorithmicif}{\textbf{如果}}
\renewcommand{\algorithmicthen}{\textbf{那么}}
\renewcommand{\algorithmicend}{\textbf{结束}}
\renewcommand{\algorithmicelse}{\textbf{否则}}
\renewcommand{\algorithmicreturn}{\textbf{返回}}
\renewcommand{\algorithmicfor}{\textbf{遍历}}
\renewcommand{\algorithmicdo}{\textbf{执行}}
\renewcommand{\algorithmicwhile}{\textbf{当}}

\providecommand{\algorithmname}{算法}
\renewcommand{\listalgorithmname}{算法目录}

\usepackage{tikz}
\usepackage{array}

\newcommand\DiagonalCell[2]{%
  \begin{tikzpicture}[baseline=(current bounding box.center),anchor=center]
    \node[minimum width=0.8cm,minimum height=2em] (box) {};
    \draw (box.north west) -- (box.south east);
    \node[anchor=south west,inner sep=1pt,outer sep=0pt] at (box.south west) {#1};
    \node[anchor=north east,inner sep=1pt,outer sep=0pt] at (box.north east) {#2};
  \end{tikzpicture}%
}
\usepackage{listings}
\usepackage{xcolor}

\definecolor{codegreen}{rgb}{0,0.6,0}
\definecolor{codegray}{rgb}{0.5,0.5,0.5}
\definecolor{codepurple}{rgb}{0.58,0,0.82}
\definecolor{backcolour}{rgb}{0.95,0.95,0.92}

\lstdefinestyle{mystyle}{
    backgroundcolor=\color{backcolour},   
    commentstyle=\color{codegreen},
    keywordstyle=\color{magenta},
    numberstyle=\tiny\color{codegray},
    stringstyle=\color{codepurple},
    basicstyle=\ttfamily\footnotesize,
    breakatwhitespace=false,         
    breaklines=true,                 
    captionpos=b,                    
    keepspaces=true,                 
    numbers=left,                    
    numbersep=5pt,                  
    showspaces=false,                
    showstringspaces=false,
    showtabs=false,                  
    tabsize=2
}
\renewcommand{\lstlistingname}{代码段}
\lstset{style=mystyle}
\begin{document}
%-
%-> Frontmatter: title page, abstract, content list, symbol list, preface
%-

\frontmatter% initialize the environment

%---------------------------------------------------------------------------%
%->> Frontmatter
%---------------------------------------------------------------------------%
%-
%-> 生成封面
%-
% //TODO: bind
\maketitle% 生成中文封面
\MAKETITLE% 生成英文封面
%-
%-> 作者声明
%-
\makedeclaration% 生成声明页
%-
%-> 中文摘要
%-
\intobmk\chapter*{摘\quad 要}% 显示在书签但不显示在目录
\setcounter{page}{1}% 开始页码
\pagenumbering{Roman}% 页码符号

% 随着量子计算领域的快速发展,模型检测作为一种验证技术,在确保量子系统可靠性方面显得尤为关键。本次研究着重于解决量子模型检测中的一个主要挑战:当面对大规模和复杂的量子线路时,计算复杂性会随着系统规模的增长而指数级增加。为应对这一挑战,本研究提出利用张量决策图(Tensor Decision Diagram,简称TDD)来设计一种更高效的量子模型检测方法。

% 研究的核心在于开发能够有效利用TDD进行操作和收缩张量网络的算法,以此降低计算的复杂性和提高处理效率。在具体实现上,包括了开发新技术以分割张量网络、优化TDD结构,以及改进现有算法以适应复杂的量子系统。此外,研究还涉及设计实现一种基于TDD的量子线路状态转移方法,通过逐步收缩表示量子线路的TDD和当前系统状态的TDD,推导出系统下一状态的TDD。这种方法能够有效地验证系统所处的子空间,从而评估量子系统是否达到了需要验证的属性。

% 在本次中期报告前,本研究已在Python和C++两种编程语言环境下,开发出不同版本的TDD,并成功实现了系统的一步状态计算,即image computation。此外,对这些实现进行了一系列的实验测试,以验证其有效性和性能。

% 本研究的最终目标是通过详细的分析和实验,撰写一篇硕士学位论文。该论文将总结本研究的主要成果,为量子系统的设计、验证和优化提供有力的理论和实践支持。通过这项工作,希望为量子计算技术的进一步发展奠定坚实的理论和应用基础。
随着近年来量子计算机在规模和可靠性方面的快速发展,针对量子系统进行自动化验证已成为亟待解决的重要问题。传统的模型检测技术在面对量子系统时往往遇到资源消耗过多的瓶颈。为此,本文提出了一种基于张量决策图(TDD)的量子模型检测新方法,旨在降低资源需求,扩展量子模型检测的适用范围。

本文首先,对量子计算、模型检测及其相关数学基础进行了概述。然后介绍了TDD数据结构,它本质上是一种带权重的决策树,用于紧凑高效地表示张量网络。接着阐述了如何将量子线路转化为TDD表示,并介绍了TDD的规范化、化简等操作。

在此基础上,本文提出了基于TDD的量子模型检测算法流程。首先将量子系统建模为量子迁移系统,然后给出了计算TDD表示的子空间的基、子空间并运算以及系统一步迁移的具体算法。为加速计算,设计了多种优化策略,包括线路拆分、基于TDD的划分,以及近似TDD表示等。

通过在多种量子算法实例上的实验对比,结果表明这些优化方法能够有效降低模型检测中一步迁移过程的时间和空间资源消耗,从而提高算法的可扩展性。其中,基于线路拆分的contraction技术优化效果最为显著,能显著提升处理线路层数较深的量子算法的能力。

最后,本文总结了该研究工作对量子模型检测的贡献,并讨论了未来的工作方向,如结合更复杂量子系统建模、改进TDD的表示能力等。本文的研究为量子计算机软硬件可靠性验证提供了有价值的新工具支持。

\keywords{模型检测,量子计算,张量决策图}% 中文关键词
%-
%-> 英文摘要
%-
\intobmk\chapter*{Abstract}% 显示在书签但不显示在目录

% With the rapid development of the quantum computing field, model checking, as a verification technology, becomes particularly critical in ensuring the reliability of quantum systems. This research focuses on addressing a major challenge in quantum model checking: the computational complexity exponentially increases with the system size when facing large-scale and complex quantum circuits. To tackle this challenge, the study proposes using Tensor Decision Diagrams (TDD) to design a more efficient quantum model checking method.

% The core of the research lies in developing algorithms that can effectively utilize TDD for operations and contraction of tensor networks, thereby reducing computational complexity and improving processing efficiency. Specifically, it includes developing new techniques for partitioning tensor networks, optimizing TDD structures, and improving existing algorithms to adapt to complex quantum systems. Moreover, the study involves designing and implementing a TDD-based quantum circuit state transition method. This method deduces the TDD of the system's next state by progressively contracting the TDD representing the quantum circuit and the TDD of the current system state. This method can effectively verify the subspaces in which the system resides, thereby assessing whether the quantum system has achieved the properties that need to be verified.

% By the mid-term report, this research had developed different versions of TDD in both Python and C++ programming environments and successfully implemented the system's one-step state computation, i.e., image computation. In addition, a series of experimental tests were conducted to verify their effectiveness and performance.

% The ultimate goal of this research is to write a master's thesis through detailed analysis and experiments. This thesis will summarize the main findings of the research, providing strong theoretical and practical support for the design, verification, and optimization of quantum systems. Through this work, it aims to lay a solid theoretical and application foundation for the further development of quantum computing technology.
With the rapid development of quantum computers in terms of scale and reliability in recent years, automated verification for quantum systems has become an important issue that needs urgent resolution. Traditional model checking techniques often encounter the bottleneck of excessive resource consumption when facing quantum systems. To this end, this paper proposes a new quantum model checking method based on Tensor Decision Diagrams (TDD), aimed at reducing resource requirements and expanding the applicability of quantum model checking.

First, this paper provides an overview of quantum computing, model checking, and their related mathematical foundations. Then, it introduces the TDD data structure, which is essentially a weighted decision tree used for compact and efficient representation of tensor networks. Following this, it explains how to convert quantum circuits into TDD representations and introduces operations such as normalization and simplification of TDDs.

Based on this foundation, a quantum model checking algorithm process based on TDD is proposed. It models quantum systems as quantum transition systems, and then presents specific algorithms for computing the basis of the subspaces represented by TDD, subspace union operations, and one-step transitions of the system. To accelerate computation, various optimization strategies are designed, including circuit splitting, partitioning based on TDD, and approximate TDD representation.

Experimental comparisons on various quantum algorithm instances show that these optimization methods can effectively reduce the time and space resource consumption in the model checking one-step transition process, thereby improving the scalability of the algorithm. Among them, the contraction technique optimization based on circuit splitting shows the most significant effect, significantly enhancing the ability to handle quantum algorithms with deeper circuit layers.

Finally, the paper summarizes the contributions of this research work to quantum model checking and discusses future work directions, such as integrating more complex quantum system modeling and improving the representational capability of TDD. The research in this paper provides valuable new tool support for the reliability verification of quantum computer hardware and software.

\KEYWORDS{Model Checking, Quantum Computation, Tensor Decision Diagrams}% 英文关键词

\pagestyle{enfrontmatterstyle}%
\cleardoublepage\pagestyle{frontmatterstyle}%

%---------------------------------------------------------------------------%
% title page, abstract

{% content list region
\linespread{1.2}% local line space
\intobmk*{\cleardoublepage}{\contentsname}% add link to bookmark
\newpage

\tableofcontents% content catalog
\intobmk*{\cleardoublepage}{图表目录}
\pagestyle{figureheader}
{
\cleardoublepage
{
\let\clearpage\relax  % Do nothing when a \clearpage command appears 
\let\cleardoublepage\relax

\renewcommand*{\addvspace}[1]{}
\let\oldnumberline\numberline%
\renewcommand{\numberline}{\figurename~\oldnumberline}%

\listoffigures
\let\clearpage\relax  % Do nothing when a \clearpage command appears 
\let\cleardoublepage\relax
\renewcommand{\numberline}{\appfigname~\oldnumberline}%
\vspace{-30pt}
\listofappfigs
}
{

\let\clearpage\relax  % Do nothing when a \clearpage command appears 
\let\cleardoublepage\relax
\renewcommand*{\addvspace}[1]{}
\let\oldnumberline\numberline%
\renewcommand{\numberline}{\tablename~\oldnumberline}%
\listoftables

\renewcommand{\numberline}{\apptabname~\oldnumberline}%
\vspace{-30pt}
\listofapptabs
\hspace{10cm}
\\
}
{
\let\clearpage\relax  % Do nothing when a \clearpage command appears 
\let\cleardoublepage\relax
\renewcommand*{\addvspace}[4]{}
\let\oldnumberline\numberline%
% \renewcommand{\numberline}{\algorithmname~\oldnumberline}%
\renewcommand{\numberline}[1]{\algorithmname~\oldnumberline{#1}\hspace{-1.5em}}
% \vspace{-30pt}
\listofalgorithms
}
}

}
\thispagestyle{figureheader}
\input{Tex/Prematter}% symbol list, preface content
%-
%-> Mainmatter
%-
\mainmatter% initialize the environment

%---------------------------------------------------------------------------%
%->> Main content
%---------------------------------------------------------------------------%
\renewcommand{\thefigure}{\thechapter-\arabic{figure}}
\renewcommand{\thetable}{\thechapter-\arabic{table}}
\renewcommand{\theequation}{\thechapter-\arabic{equation}}

\section{学位论文进展情况}

本章节将概述学位论文的整体进展情况。
首先,将介绍研究的背景,这部分内容将简要阐述量子计算机,模型检测等相关背景。
接着,会详细描述研究的主要内容,包括研究目标、方法论以及预期的贡献。
然后,会探讨在研究过程中遇到的一系列关键问题。
最后,本章节将展示目前已取得的阶段性成果,包括研究内容进展与学术论文撰写进度。

\subsection{研究背景}

模型检测(Model Checking)是一种自动化形式方法,用于验证有限状态系统的性质。模型检测最初由 E. M. Clarke 和 E. A. Emerson 提出\citep{Emerson_1980,Clarke,Clarke_1986},如今已广泛应用于软件和硬件设计。例如,在嵌入式系统中,可以使用 UML 活动图来验证硬件是否符合规范\citep{Grobelna_2015}。

模型检测将待检测的系统建模为一个跃迁系统(transition system),在时序逻辑(temporal logic)中指定待验证的属性。给定模型\(M\) 和属性\(\varphi\),模型检测将验证是否\(M\)满足\(\varphi\)。在不同的模型检测方法中,高级符号模型检查(Advanced Symbolic Model Checking)\citep{Grobelna_2015}使用简化的有序二叉决策图(Reduced Ordered Binary Decision Diagrams,ROBDDs 或 BDDs)\citep{Bryant_1986}来表示状态集合和转移关系。通过迭代调用图像计算算法来计算所有可达状态,判断一个模型是否满足时间属性,直到达到不动点为止。

最近,随着量子计算的发展,关于量子线路的验证技术也在不断发展\citep{viamontes2007checking,burgholzer2020advanced}。其中,利用模型检测方法对线路进行自动化验证也有了一些应用。由于量子线路运算空间随着量子比特的线性增加而指数级膨胀,传统的计算方法并不能很好应对。因此本次研究希望应用基于张量网络(tensor network)的张量决策图(tensor decision diagrams)进行量子模型检测。 

\subsubsection{量子计算简介}

量子计算机(quantum computer)是一种利用量子比特特性进行计算的一种设备。在量子计算中,量子比特的特殊性质允许其同时处于多种状态,这与经典比特的二进制状态不同。量子计算机的状态空间可以用希尔伯特空间(Hilbert space)\(\mathcal{H}\)表示\citep{nielsen2010quantum},即可以进行内积运算(inner product)的复向量空间。比特状态可以用\(\mathcal{H}\)的向量表示,量子门由\(\mathcal{H}\)上的酉算子(unitary operator)表示。

量子线路(quantum circuit)是一种描述量子计算的模型。在量子线路中,通过量子比特的初始化、应用量子门、测量以及其他可能的操作的序列来构建和执行量子计算任务。量子线路通常从左向右阅读,每个量子门的作用是将输入的量子比特状态转变为输出状态,该过程可以认为是量子门的酉矩阵与输入的量子状态的乘积。
\begin{figure}[!htbp]
    \centering
    \includegraphics[width=.6\textwidth]{Img/example_cir.pdf}
    \caption{一个量子线路的例子}
    \label{fig:example_cir}
\end{figure}

图\ref{fig:example_cir} 所示的量子线路展示了一个具体的量子线路示例。其中有单比特门\(H=\frac{1}{\sqrt2}\left[\begin{matrix}1&1\\1&-1\\\end{matrix}\right],T=\left[\begin{matrix}1&0\\0&e^{-i\pi/4}\\\end{matrix}\right]\),以及双比特门\(CX=\left[\begin{matrix}\begin{matrix}1&0\\0&1\\\end{matrix}&\begin{matrix}0&0\\0&0\\\end{matrix}\\\begin{matrix}0&0\\0&0\\\end{matrix}&\begin{matrix}0&1\\1&0\\\end{matrix}\\\end{matrix}\right]\)。假设该量子线路的初始状态为\(\left|\psi\right\rangle=\left|\psi_1\right\rangle\left|\psi_2\right\rangle\),则输出状态为\(T\otimes H\cdot CX\cdot T\otimes H\cdot\left|\psi\right\rangle\)。

在量子计算机上可以执行各种算法和计算任务,如量子搜索\citep{Grover_1996}、量子因子分解\citep{shor}和量子模拟\citep{Feynman}等。量子计算的潜力在于其能够在某些特定问题上比经典计算机更高效地进行计算,尤其在处理大规模数据和解决复杂问题方面具有潜在优势。需要对这部分深入了解的读者,可以自行阅读\citep{nielsen2010quantum}。

\subsubsection{跃迁系统}
跃迁系统广泛应用于模型检测中待检测系统的建模,其定义为\citep{baier2008principles}:
\begin{equation}
\mathcal{M}=\{S,Act,\rightarrow,I\}
\end{equation}
其中\(S\)为系统状态集合,\(I\)为系统初态集合,因此满足\(I\subseteq S\)。\(Act\)为系统行为集合。\(\rightarrow\)为系统状态转移关系,即\(\rightarrow\subset S\times Act\times S\)。此外还有\(AP\)为描述系统原子命题。L是标记函数,将状态映射为状态满足的原子命题集合。需要验证的属性\(\varphi\)将表述为命题。


系统的有限路径片段\(\pi\)是一个有限状态序列\(s_0,s_1\ldots s_n\)。\(s_i\)满足\(s_{i-1}\overset{a}{\rightarrow}s_i,a\in Act\),对于所有\(0<i\leq n\),其中\(n\geq 0 \)。无限路径片段\(\pi\)是一个无限状态序列\(s_0,s_1\ldots\),使得对于所有\(i>0\),\(s_{i-1} \overset{a}{\rightarrow}  s_i,a\in Act\)。在路径中\(\pi\left[i\right]=s_i,\pi\left[i\right)=s_i\ldots\)。所有以\(s_0\)为开始的路径,构成了路径集合\(Path\left(s_0\right)\)。

\begin{figure}[!htbp]
    \centering
    \includegraphics[width=.6\textwidth]{Img/map.pdf}
    \caption{一种简化版的售货机跃迁系统}
    \label{fig:transition-system}
\end{figure}

图\ref{fig:transition-system} 所示的跃迁系统展示了一个简化版的售货机模型。在该模型中,用户投入硬币,进行选择后就可以得到苏打水或者啤酒。在该例子中,系统状态\(S=\{pay,select,soda,beer\}\),系统初态\(I=pay\)。
系统行为\(Act=\{insert\_coin,\tau,get\_soda,get\_beer\}\),其中\(\tau\)表示立即行动符号。转移关系图中已经展示。原子命题可取\(AP=\{paid,drink\}\)。因此\(L\left( pay \right)=\{\varnothing\}\),\(L\left(soda\right)=L\left(beer\right)=\{paid,drink\}\),\(L\left(select\right)=\{paid\}\)。系统的一个路径是\(\pi=pay\ select\ soda\ pay\ selsect\ \ldots\)。此时\(\pi\left[1\right]=slect,\pi\left[1\right)=select\quad soda\quad pay\quad selsect\ldots\)。同时该路径满足\(\pi\in Path\left(pay\right)\)。
量子模型检测的跃迁系统类似。区别在于状态空间用\(\mathcal{H}\),转移关系用酉矩阵。一个量子自动机定义如下:
\begin{align}
    \mathcal{M}=\{\mathcal{H},Act,\{U_\alpha,\alpha\in Act\},\mathcal{H}_0\}
\end{align}
由于目前量子模型检测的发展还比较初期,因此需要验证的属性\(\varphi\)会表示为\(\mathcal{H}\)的一个子空间。
下一节将简要介绍量子模型检测。

\subsubsection{量子模型检测}
随着量子计算硬件规模的快速增长,量子电路的验证成为一个重要问题。开始的研究主要集中在BDD在量子计算下的推广算法,如量子信息决策图(Quantum Information Decision Diagram,或QuIDD)\citep{Viamontes_2003},量子多值决策图(Quantum multiple-valued Decision Diagram,或QMDD)\citep{Seiter_2013}等,从而对组合式量子电路进行等效性检查。显然,随着越来越复杂的物理可实现化的硬件出现,将会出现更加复杂的,更加针对于的,新的验证问题。比如量子存储\citep{Kerckhoff_2010},量子反馈网络\citep{Gough_2008},RUS量子电路\citep{Bocharov_2015}。量子模型检测可以为量子电路的验证提供了更多思路。

量子系统模型检测的早期工作旨在验证量子通信协议\citep{Gay,BALTAZAR_2008,davidson2012model}。后来还有针对分析和验证量子程序的应用\citep{ying2016foundations},比如量子自动机\citep{ying2014model}、量子马尔可夫链\citep{Ying_2013}和超算符值马尔可夫链\citep{feng2013model}的模型检测技术。然而,在这些量子模型检测技术与它们在验证量子电路方面实际应用之间存在巨大差距仍需填补。TDD作为新的数据结构,极大加快了计算过程,有可能深化二者的联系,加快实际应用的出现。

对于量子模型检测的可达性问题的解决,一种直接的解决办法是计算路径上每个状态,然后与待检测属性比较。但是量子计算中状态空间\(\mathcal{H}\)维度\(dim\left(\mathcal{H}\right)=2^n\),其中n为比特数量。即状态空间维数随比特个数指数级增长。

因此需要借助更好的数据结构,更方便的表示量子状态以及量子线路,并计算最终的结果。比如TDD给出了量子电路的紧凑表示,提供了一种方便的实现张量网络各种操作的方式,这些操作对于模拟量子物理系统非常重要。图\ref{fig:P}展示了一个矩阵和TDD形式,其中TDD中的实线表示高边,虚线表示低边。可以明显看到TDD的结构更紧凑。
 	 
\begin{figure}[!htbp]
    \begin{subfigure}[c]{0.4\textwidth}
        \centering
        \includegraphics[width=\textwidth]{Img/matrix_of_tdd.pdf}
        \caption{矩阵$P$的矩阵形式}
        \label{fig:mat_P}
    \end{subfigure}
    \begin{subfigure}[c]{0.4\textwidth}
        \centering
        \includegraphics[height=6cm]{Img/tdd_ex.pdf}
        \caption{矩阵$P$的TDD形式}
        \label{fig:tdd_P}
    \end{subfigure}
    \caption{应用TDD可以减少存储特殊矩阵的资源}
    \label{fig:P}
\end{figure}

TDD特别适用于实现可达性分析和模型检查算法。这是因为基于BDD的模型检查算法中使用的许多优化技术可以推广到收缩量子电路张量网络上\citep{Chaki_2018}。这些为应用TDD解决量子模型检测问题提供了可能的方案。
目前量子模型检测仅仅应用于小规模线路的验证,如何扩展到更大规模电路,是一个难题。

\subsection{研究内容}
在本节中,将探讨本研究的核心内容,即“量子模型检测中的时序逻辑的验证”。
这一研究问题在量子计算领域具有重要的理论与实际意义,关乎量子计算机的精确性和效率。为了解决这一问题,本研究采用了方法论为利用TDD(Tensor Decision Diagram)数据结构来快速模拟量子计算过程。这种方法提高了计算效率,为量子计算提供了更加准确的模拟。

进一步地,研究过程中将探讨四种可能的优化途径,这些途径旨在进一步提升模拟过程的效率。
这四种方法既有来源于经典计算机的思路,也有量子计算机特有的思路。每一种方法都将简要说明。

\subsubsection{时序逻辑的验证}
在量子模型检测中,与经典模型检测一样使用时序逻辑指定待验证的属性\(\varphi\)。时序逻辑命题的运算符有两类\citep{goranko_2023}。状态命题公式(State formulas):\(\varphi ::=a\left|\exists\varphi\right|\forall \varphi\left|\lnot\varphi\right|\varphi\land\psi\),其中\(a\in AP\)。以及路径命题公式(Path formulas):\(\varphi\Colon=O\varphi|\varphi U\psi\)。给定模型的一个状态为\(s\),路径为\(\pi\),则具体满足条件分别如下:
\begin{itemize}
    \item \(s\models a,iff \L\left(s\right)\models a\)
    \item \(s\models\exists\varphi,iff\ \pi\models\varphi\)对一些\(\pi\in Path\left(s\right)\)
    \item \(s\models\forall\varphi,iff\ \pi\models\varphi\)对所有\(π\in Paths\)
    \item \(s\models\lnot\varphi,iff\ s\nvDash\varphi\)
    \item \(s\models\varphi\land\psi,iff\ s\models\varphi\ and\ s\models\psi\)
    \item \(\pi\models O\varphi,iff\ \pi\left[1\right]\models\varphi\)
    \item \(\pi\models\varphi U\psi,iff\ \exists j\geq0\).\(\pi\left[j\right)\models\psi\) 同时对所有\(0\le i<j\)有\(\pi\left[i\right)\models\varphi\)
\end{itemize}


图\ref{fig:path-formula-basic} 展示了两种路径命题公式的直观示意图。


\begin{figure}[!htbp]
    \centering
    \begin{subfigure}[b]{0.8\textwidth}
        \centering
        \includegraphics[width=\textwidth]{Img/path_for_Oa.pdf}
    \end{subfigure}
    \\
    \begin{subfigure}{0.8\textwidth}
        \centering
        \includegraphics[width=\textwidth]{Img/path_for_aUb.pdf}
    \end{subfigure}
    \caption{$\pi\models O a $与 $\pi\models a U b$的图示}
    \label{fig:path-formula-basic}
\end{figure}
在模型检测中,有三类比较重要的可达性问题,分别是可达性、持续可达性以及重复可达性。过程中主要涉及以下路径命题公式:\(\lozenge\) 表示最终(eventually),\(\square\)表示总是(always),\(\lozenge\square\)表示总是最终(always eventually),\(\square\lozenge\)表示最终总是(eventually always)。其中\(\lozenge\)和\(\square\)具体定义为:
\begin{itemize}
    \item \(\lozenge\varphi\overset{\text{def} }{=} \text{True}U\varphi\)
    \item \(\square\varphi\overset{\text{def} }{=} \neg\lozenge\neg\varphi\)
\end{itemize}
图\ref{fig:path-formula}展示了这两种基本路径命题公式的直观示意图。
\begin{figure}[!htbp]
    \centering
    \begin{subfigure}[b]{0.8\textwidth}
        \centering
        \includegraphics[width=\textwidth]{Img/path_for_Dia.pdf}
    \end{subfigure}
    \\
    \begin{subfigure}{0.8\textwidth}
        \centering
        \includegraphics[width=\textwidth]{Img/path_for_SQa.pdf}
    \end{subfigure}
    \caption{$\pi\models\lozenge a$与 $\pi\models\square a$的图示}
    \label{fig:path-formula}
\end{figure}

具体的可满足条件为:
\begin{itemize}
    \item \(\pi\models\lozenge\varphi,iff\exists j\ge0.\pi[j)\models\varphi\)
    \item \(\pi\models\square\varphi,iff\forall j\ge 0.\pi[j)\models\varphi\)
    \item \(\pi\models\lozenge\square\varphi,iff\exists i\ge 0.\forall j\ge i,\pi[j)\models\varphi\)
    \item \(\pi\models\square\lozenge\varphi,iff\forall i\ge 0.\exists j\ge i,\pi[j)\models\varphi\)
\end{itemize}
基于此三种可达性问题定义分别如下:
\begin{itemize}
    \item 可达性:\( Pr^{\mathcal{M}}(s \models \lozenge G) = Pr^M(\pi \models \lozenge G : \pi \in \text{Paths}(s))\)
    \item 持续可达性:\( Pr^{\mathcal{M}}(s \models \lozenge \square G) = Pr^M(\pi \models \lozenge \square G : \pi \in \text{Paths}(s))\)
    \item 重复可达性:\( Pr^{\mathcal{M}}(s \models\square \lozenge G) = Pr^M(\pi \models \square\lozenge G : \pi \in \text{Paths}(s))\)
\end{itemize}

\subsubsection{研究方法}
本次研究的主要目的是借助TDD数据结构,构建能快速计算量子模型检测中可达问题的方案。因此主要采用的方法是模拟量子计算。本次研究的主要挑战在于尽可能减少程序的运行时间。为此,需要采用一系列方法来开发更有效的算法,以优化TDD操作和收缩张量网络。其中包括开发新技术来分割张量网络和优化TDD结构。
下面简单介绍以下具体研究方法。

\begin{myen}

\item \label{addition}关于常用的量子线路划分方法,
第一种被称为addition\citep{chen2018classical}。将量子电路视为张量网络,首先将一个量子电路C转换成无向图G。G中的每个节点表示量子电路的一个索引,并且如果它们是相同门的输入或输出索引,则在G中连接两个节点。并且当满足以下两个条件之一时输入和输出索引不变:
\begin{itemize}
	\item 是对角线量子门的输入和输出索引;
	\item 是受控门的控制比特位的输入和输出索引。
\end{itemize}
	
图\ref{fig:addition}展示了Grover\_3电路图的索引链接图。该图描述了量子电路的连通性,通过选择图中连通度最大的索引可以对电路进行分割。因此选择图中连通度较大的$x_1^1,x_1^3x_2^1$可以对电路进行较好的划分。
 
\begin{figure}[!htbp]
	\centering
	\includegraphics[height=5cm]{Img/cir_index_graph.pdf}
	\caption{Grover\_3的索引连接图}
	\label{fig:addition}
\end{figure} 

\item 另一种常用的电路划分方法成为contraction。在这一方法中,将量子电路划分为若干个较小的部分,其收缩等于原始电路。对于两个预设整数参数k1和k2,将电路划分为若干小电路。其中每个小电路涉及最多k1个量子比特,并且与至多跨越不同部件的k2个多比特门相连。图\ref{fig:contraction}展示了对Bit flip电路进行k1=3,k2=2的拆分结果。
\begin{figure}[!htbp]
	\centering
	\includegraphics[height=4.5cm]{Img/cir_contraction.pdf}
	\caption{对Bit flip电路进行contraction的拆分}
	\label{fig:contraction}
\end{figure} 


\item \label{contraction}在BDD中,索引的顺序很重要。因为索引顺序会直接影响BDD的大小。一个良好的变量顺序可以使得BDD比一个糟糕的变量顺序小得多。图\ref{fig:bdd-compare}的了两张图都表示了布尔函数ƒ(x1,...,x8)=x1x2+x3x4+x5x6+x7x8,但图\ref{fig:bdd-good}的结构更简单。其中图\ref{fig:bdd-bad}的索引顺序为\{x1,x3,x5,x7,x2,x4,x6,x8\},图\ref{fig:bdd-good}的索引顺序为\{x1,x2,x3,x4,x5,x6,x7,x8\}。找到一个好的索引顺序是一个NP问题。在工程实现中,目前只能通过小规模电路上寻求规律,然后在更大规模电路中应用较优顺序。

\begin{figure}[!htbp]
	\centering
	\begin{subfigure}[b]{.4\textwidth}
        \centering
        \includegraphics[height=4.5cm]{Img/BDD_Variable_Ordering_Bad.svg.pdf}
		\caption{索引顺序为\{x1,x3,x5,x7,x2,x4,x6,x8\}}
		\label{fig:bdd-bad}
	\end{subfigure}
	\begin{subfigure}[b]{.4\textwidth}
        \centering
        \includegraphics[height=5cm]{Img/BDD_Variable_Ordering_Good.svg.pdf}
		\caption{索引顺序为\{x1,x2,x3,x4,x5,x6,x7,x8\}}
		\label{fig:bdd-good}
	\end{subfigure}
	\caption{同一布尔函数在不同索引顺序下的结构图\citep{wiki:bdd}}
	\label{fig:bdd-compare}
\end{figure}


\item 由于量子状态都在同一希尔伯特空间中。因此作用某些算子后,不同的量子状态可能等价。
当存储算子的资源少于存储状态的资源时,就有可能存储算子表示不同的状态\citep{vinkhuijzen2023limdd}。图\ref{fig:qmdd-example}表示了一个QMDD的例子,应用等价性,可以化简为图\ref{fig:limdd-example}。
TDD也可以应用类似技术,进行进一步化简,从而降低资源要求。
\begin{figure}[!htbp]
    \centering
    \begin{subfigure}[b]{.4\textwidth}
        \centering
        \includegraphics[height=6cm]{Img/limdd.pdf}
        \caption{一个QMDD示例}
        \label{fig:qmdd-example}
    \end{subfigure}
    \begin{subfigure}[b]{.4\textwidth}
        \centering
        \includegraphics[height=6cm]{Img/limdd_reduce.pdf}
        \caption{应用等价性化简图\ref{fig:qmdd-example}}
        \label{fig:limdd-example}
    \end{subfigure}
\end{figure}
\end{myen}
\subsection{存在的问题}
在本研究过程中,主要遇到了两个问题,这些问题对研究的深入发展和实际应用产生了重要影响。

首先,面临的一个关键挑战是如何将所提出的方法扩展应用到更大规模的实例。这不仅涉及到算法的效率问题,还包括数据处理能力的提升。对于实现验证量子计算算法在更广泛领域的应用至关重要。为了解决这一挑战,目前采用电路拆分方法来降低TDD的资源消耗,并挖掘可能的并行计算机会。此外,还计划应用更灵活的索引策略和limdd的思路,以期达到更高效的处理效果。

其次,另一个重要的问题是如何将研究方法应用于更加实用的示例。这是将理论研究转化为实际应用的关键一步。目前考虑的主要方向是将此方法应用到量子线路设计,即QDA(Quantum Design Automation)领域。
计划未来能够将本研究成果应用于不同的synthesis算法的验证等价性中,从而在量子计算的实际应用中发挥更大的作用。


\subsection{阶段性成果}
\subsubsection{研究内容进展}
在模型检测中,image computation指的是在给定当前状态$s_i\in S$和行为$\alpha\in Act$的情况下计算接下来的状态。
目前,关于使用TDD对量子的image computation的计算已经完成。表\ref{table:time}给出了在不同电路拆分技术下Grover算法的计算时间,单位为秒。其中basic表示没有使用优化技术,addition表示使用研究方法\ref{addition}的addition优化技术,contraction表示使用研究方法\ref{contraction}中的contraction优化技术。“-”表示超过一小时的运行上限。
\begin{table}[!htbp]
    \centering
    \begin{tabular}{llllllllll}
        \hline
        \multirow{2}{*}{Benchmark} &  & \multicolumn{2}{c}{basic} &  & \multicolumn{2}{c}{addition} &  & \multicolumn{2}{c}{contraction} \\ \cline{3-4} \cline{6-7} \cline{9-10} 
                                   &  & time        & max \#node       &  & time          & max \#node        &  & time           & max \#node          \\ \hline
        Grover\_15 &   & 19.33  & 15785     &   & 17.35      & 15099  & & 1.61 & 597  \\
        Grover\_18 &   & 76.47  & 61694     &   & 66.02      & 60332  & & 2.41 & 516  \\
        Grover\_20 &   & 294.65 & 243946    &   & 259.87     & 241240 & & 4.39  & 1036 \\ 
        Grover\_40 &   & -      &           &   & -          &        & & 2953.57 & 851973 \\
        \hline
        QFT\_15     &  & 34.64   & 65536   &  & 18.88  & 32770   &  & 0.08 & 63  \\
        QFT\_18     &  & 282.12  & 524288  &  & 148.13 & 262146 &   & 0.10  & 31  \\
        QFT\_20     &  & 1199.21 & 2097152 &  & 655.19 & 1048578 &  & 0.12 & 63  \\
        QFT\_30     &  & -       &         &  & -      &        &  & 0.29 & 31  \\
        QFT\_50     &  & -       &         &  & -      &        &  & 1.02 & 51  \\
        QFT\_100    &  & -       &         &  & -      &        &  & 7.14 & 101 \\
        \hline
        BV\_100     &  & 7.36    & 596     &  & 7.43      & 596     &  & 0.41           & 102 \\
        BV\_200     &  & 31.57   & 1196    &  & 30.03     & 1196    &  & 1.70           & 202 \\
        BV\_300     &  & 75.66   & 1796    &  & 75.56     & 1796    &  & 4.28           & 302 \\
        BV\_400     &  & 146.47  & 2396    &  & 145.40    & 2396    &  & 9.18           & 402 \\
        BV\_500     &  & 244.15  & 2996    &  & 223.90    & 2996    &  & 16.31          & 502 \\
        \hline
        GHZ\_100    &  & 0.38    & 595     &  & 0.13      & 301    &  & 0.18           & 200 \\%& 0.03    
        GHZ\_200    &  & 0.72    & 1195    &  & 0.37      & 601    &  & 0.48           & 400 \\%& 0.12     
        GHZ\_300    &  & 1.29    & 1795    &  & 0.62      & 901    &  & 0.80           & 600 \\%& 0.24     
        GHZ\_400    &  & 2.03    & 2395    &  & 1.00      & 1201    &  & 1.26           & 800 \\%& 0.42     
        GHZ\_500    &  & 2.96    & 2995    &  & 1.45      & 1501    &  & 1.72           & 1000\\%& 0.62     
        \hline
        QRW\_15     &  & 36.86   & 13122     &  & 24.59     & 10882     & & 7.16  & 222 \\
        QRW\_18     &  & 139.76  & 90538     &  & 84.69     & 37064     & & 11.23 & 226 \\
        QRW\_20     &  & 341.05  & 265614    &  & 218.29    & 107714    & & 14.31 & 404 \\
        QRW\_30     &   &-       &          &  &-          &          & & 36.82 & 404 \\
        QRW\_50     &   &-       &          &  &-          &          & & 118.08 & 404 \\
        QRW\_100    &   &-       &          &  &-          &          & & 692.08 & 436 \\
        \hline
    \end{tabular}
    \caption{对不同测试实验应用image computation}
    \label{table:time}
\end{table}

通过对比不同优化技术下的计算时间,可以看到使用优化技术能够显著降低计算时间。例如,在Grover-20的例子中,使用"contraction"优化技术的情况下,计算时间从294秒降低到了4秒。这表明优化技术在提高计算效率方面起到了积极的作用。同时,表\ref{table:addition}展示了对同一线路,即Grover\_15应用不同的addition参数的时间,可以看到合适的参数
选择也是非常重要的。
\begin{table}[!htbp]
    \centering
    \begin{tabular}{c|ccccccccccccccc}
        \rowcolor[HTML]{FFFFFF} 
        \diagbox{k1}{k2}                         & 1                           & 2                           & 3                           & 4                           & 5                           & 6                           & 7                          & 8                           & 9                           & 10                          & 11                          & 12                          & 13                          & 14                          & 15                          \\\hline
            \rowcolor[HTML]{FFFFFF} 
    1                          & 2.8                                              & 2.2                         & 2.1                         & \cellcolor[HTML]{CCC0DA}2.0 & \cellcolor[HTML]{CCC0DA}1.9 & \cellcolor[HTML]{CCC0DA}2.0                      & 2.1                        & \cellcolor[HTML]{CCC0DA}2.0 & 2.1                         & \cellcolor[HTML]{CCC0DA}2.0 & \cellcolor[HTML]{CCC0DA}2.0 & 2.1                         & 2.2                         & 2.1                         & 2.1                         \\ \cline{3-7}
    \rowcolor[HTML]{CCC0DA} 
    
    \rowcolor[HTML]{FFFFFF} 
    3                          & \multicolumn{1}{l|}{\cellcolor[HTML]{FFFFFF}2.2} & \cellcolor[HTML]{CCC0DA}1.9 & \cellcolor[HTML]{CCC0DA}1.8 & \cellcolor[HTML]{CCC0DA}1.6 & \cellcolor[HTML]{CCC0DA}2.0 & \multicolumn{1}{l|}{\cellcolor[HTML]{CCC0DA}1.9} & 2.1                        & 2.1                         & 2.5                         & 2.3                         & 2.7                         & 2.3                         & 3.1                         & 2.8                         & 3.3                         \\
    \rowcolor[HTML]{FFFFFF} 
    4                          & \multicolumn{1}{l|}{\cellcolor[HTML]{FFFFFF}2.3} & \cellcolor[HTML]{CCC0DA}1.8 & \cellcolor[HTML]{CCC0DA}2.0 & \cellcolor[HTML]{CCC0DA}1.7 & \cellcolor[HTML]{CCC0DA}2.0 & \multicolumn{1}{l|}{\cellcolor[HTML]{FFFFFF}2.1} & 2.2                        & 2.1                         & 2.6                         & 2.3                         & 2.8                         & 2.7                         & 3.3                         & 3.0                         & 3.3                         \\
    \rowcolor[HTML]{FFFFFF} 
    5                          & \multicolumn{1}{l|}{\cellcolor[HTML]{FFFFFF}2.2} & \cellcolor[HTML]{CCC0DA}1.7 & \cellcolor[HTML]{CCC0DA}1.9 & \cellcolor[HTML]{CCC0DA}1.6 & \cellcolor[HTML]{CCC0DA}1.9 & \multicolumn{1}{l|}{\cellcolor[HTML]{CCC0DA}2.0} & 2.3                        & \cellcolor[HTML]{CCC0DA}1.9 & 2.5                         & 2.3                         & 2.8                         & 2.7                         & 3.4                         & 3.0                         & 3.6                         \\
    \rowcolor[HTML]{FFFFFF} 
    6                          & \multicolumn{1}{l|}{\cellcolor[HTML]{FFFFFF}2.1} & \cellcolor[HTML]{B1A0C7}1.5 & \cellcolor[HTML]{CCC0DA}1.8 & \cellcolor[HTML]{CCC0DA}1.7 & 2.2                         & \multicolumn{1}{l|}{\cellcolor[HTML]{CCC0DA}1.9} & 2.5                        & 2.2                         & 2.9                         & 2.8                         & 3.1                         & 2.9                         & 3.7                         & 3.7                         & 4.2                         \\
    \rowcolor[HTML]{FFFFFF} 
    7                          & \multicolumn{1}{l|}{\cellcolor[HTML]{FFFFFF}2.1} & \cellcolor[HTML]{CCC0DA}1.5 & \cellcolor[HTML]{CCC0DA}1.9 & \cellcolor[HTML]{CCC0DA}1.6 & 2.2                         & \multicolumn{1}{l|}{\cellcolor[HTML]{CCC0DA}1.9} & 2.5                        & 2.2                         & 2.8                         & 3.0                         & 3.6                         & 3.3                       & 4.2                         & 5.7                         & 5.0                         \\
    \rowcolor[HTML]{FFFFFF} 
    8                          & \multicolumn{1}{l|}{\cellcolor[HTML]{CCC0DA}2.0} & \cellcolor[HTML]{CCC0DA}1.7 & \cellcolor[HTML]{CCC0DA}1.8 & \cellcolor[HTML]{CCC0DA}1.7 & 2.1                         & \multicolumn{1}{l|}{\cellcolor[HTML]{CCC0DA}2.0} & 2.4                        & 2.2                         & 2.8                         & 2.8                         & 3.7                         & 3.4                         & 4.3                         & 4.8                         & 5.2                         \\
    \rowcolor[HTML]{FFFFFF} 
    9                          & \multicolumn{1}{l|}{\cellcolor[HTML]{FFFFFF}2.1} & \cellcolor[HTML]{B1A0C7}1.5 & \cellcolor[HTML]{CCC0DA}2.0 & \cellcolor[HTML]{B1A0C7}1.4 & 2.2                         & \multicolumn{1}{l|}{\cellcolor[HTML]{CCC0DA}2.0} & 2.5                        & \cellcolor[HTML]{CCC0DA}2.0 & 3.3                         & 2.9                         & 3.7                         & 3.5                         & 4.9                         & 4.7                         & 5.8                         \\ \cline{3-7}
    \rowcolor[HTML]{FFFFFF} 
    10                         & 2.3                                              & \cellcolor[HTML]{CCC0DA}1.9 & 2.3                         & \cellcolor[HTML]{CCC0DA}1.6 & 2.6                         & 2.7                                              & 3.1                        & 2.2                         & 4.0                         & 3.6                         & 4.6                         & 3.9                         & 5.6                         & 5.2                         & 7.5                         \\
    \rowcolor[HTML]{FFFFFF} 
    11                         & 3.2                                              & 3.2                         & 3.5                         & 3.1                         & 4.7                         & 4.2                                              & 5.6                        & 4.2                         & 6.8                         & 7.2                         & 7.6                         & 6.3                         & 9.0                         & 8.1                         & \cellcolor[HTML]{8DB4E2}11  \\
    \rowcolor[HTML]{FFFFFF} 
    12                         & 5.6                                              & 6.0                         & 7.2                         & 6.0                         & 8.3                         & 9.0                                              & 8.9                        & 7.8                         & \cellcolor[HTML]{8DB4E2}11  & \cellcolor[HTML]{8DB4E2}11  & \cellcolor[HTML]{8DB4E2}12  & \cellcolor[HTML]{8DB4E2}11  & \cellcolor[HTML]{8DB4E2}12  & \cellcolor[HTML]{8DB4E2}15  & \cellcolor[HTML]{8DB4E2}16  \\
    \rowcolor[HTML]{8DB4E2} 
    \cellcolor[HTML]{FFFFFF}13 & 11                                               & 12                          & 14                          & 12                          & 15                          & 18                                               & 18                         & 15                          & 18                          & 20                          & 18                          & 32                          & 32                          & 30                          & 25                          \\
    \rowcolor[HTML]{8DB4E2} 
    \cellcolor[HTML]{FFFFFF}14 & 20                                               & 21                          & 24                          & 32                          & 31                          & 44                                               & 77                         & 50                          & 86                          & \cellcolor[HTML]{538DD5}109 & 68                          & \cellcolor[HTML]{538DD5}133 & 70                          & \cellcolor[HTML]{538DD5}119 & \cellcolor[HTML]{538DD5}142 \\
    \rowcolor[HTML]{538DD5} 
    \cellcolor[HTML]{FFFFFF}15 & \cellcolor[HTML]{8DB4E2}28                       & \cellcolor[HTML]{8DB4E2}30  & \cellcolor[HTML]{8DB4E2}31  & \cellcolor[HTML]{8DB4E2}53  & \cellcolor[HTML]{8DB4E2}69  & 111                                              & \cellcolor[HTML]{8DB4E2}85 & \cellcolor[HTML]{8DB4E2}81  & 102                         & 153                         & 114                         & 130                         & 166                         & 162                         & 235                        
                     
    \end{tabular}
    \caption{对grover\_15应用不同的addition 参数}%calculating the 
    \label{table:addition}
\end{table}
  
在技术实现上,构建了两个版本的量子线路转化为TDD的工具,分别基于C语言和Python语言。这些工具的开发对于实现我们的研究方法至关重要,提高了实验的灵活性和效率。

\subsubsection{学术论文进展:}
在学术论文撰写方面的进展包括:
\begin{itemize}
    \item 完成了学术论文中关于研究背景的详细调查和综述,这部分内容主要包括了量子模型检测的背景知识与重要性。
    \item 撰写了研究内容的方法论部分,详细描述了主要的研究方法和实验设计。阐述了我们的研究方法,并详细介绍了方法的实施步骤和预期目标。
\end{itemize}

\section{研究现状}
随着量子计算硬件规模的快速增长,量子电路的验证成为一个重要问题。开始的研究主要集中在BDD在量子计算下的推广算法,如量子信息决策图(Quantum Information Decision Diagram,或QuIDD)\citep{Viamontes_2003},量子多值决策图(Quantum multiple-valued Decision Diagram,或QMDD)\citep{Seiter_2013}等,从而对组合式量子电路进行等效性检查。显然,随着越来越复杂的物理可实现化的硬件出现,将会出现更加复杂的,更加针对于的,新的验证问题。比如量子存储\citep{Kerckhoff_2010},量子反馈网络\citep{Gough_2008},RUS量子电路\citep{Bocharov_2015}。量子模型检测可以为量子电路的验证提供了更多思路。

量子系统模型检测的早期工作旨在验证量子通信协议\citep{Gay,BALTAZAR_2008,davidson2012model}。后来还有针对分析和验证量子程序的应用\citep{ying2016foundations},比如量子自动机\citep{ying2014model}、量子马尔可夫链\citep{Ying_2013}和超算符值马尔可夫链\citep{feng2013model}的模型检测技术。然而,在这些量子模型检测技术与它们在验证量子电路方面实际应用之间存在巨大差距仍需填补。TDD作为新的数据结构,极大加快了计算过程,有可能深化二者的联系,加快实际应用的出现。

在本节中,将探讨本研究的核心内容,即“量子模型检测中的时序逻辑的验证”。
这一研究问题在量子计算领域具有重要的理论与实际意义,关乎量子计算机的精确性和效率。为了解决这一问题,本研究采用了方法论为利用TDD(Tensor Decision Diagram)数据结构来快速模拟量子计算过程。这种方法提高了计算效率,为量子计算提供了更加准确的模拟。

进一步地,研究过程中将探讨四种可能的优化途径,这些途径旨在进一步提升模拟过程的效率。
这四种方法既有来源于经典计算机的思路,也有量子计算机特有的思路。每一种方法都将简要说明。


\subsection{研究方法}
本次研究的主要目的是借助TDD数据结构,构建能快速计算量子模型检测中可达问题的方案。因此主要采用的方法是模拟量子计算。本次研究的主要挑战在于尽可能减少程序的运行时间。为此,需要采用一系列方法来开发更有效的算法,以优化TDD操作和收缩张量网络。其中包括开发新技术来分割张量网络和优化TDD结构。
下面简单介绍以下具体研究方法。

\begin{myen}

\item \label{addition}关于常用的量子线路划分方法,
第一种被称为addition\citep{chen2018classical}。将量子电路视为张量网络,首先将一个量子电路C转换成无向图G。G中的每个节点表示量子电路的一个索引,并且如果它们是相同门的输入或输出索引,则在G中连接两个节点。并且当满足以下两个条件之一时输入和输出索引不变:
\begin{itemize}
	\item 是对角线量子门的输入和输出索引;
	\item 是受控门的控制比特位的输入和输出索引。
\end{itemize}
	
图\ref{fig:addition}展示了Grover\_3电路图的索引链接图。该图描述了量子电路的连通性,通过选择图中连通度最大的索引可以对电路进行分割。因此选择图中连通度较大的$x_1^1,x_1^3x_2^1$可以对电路进行较好的划分。
 
\begin{figure}[!htbp]
	\centering
	\includegraphics[height=5cm]{Img/cir_index_graph.pdf}
	\caption{Grover\_3的索引连接图}
	\label{fig:addition}
\end{figure} 

\item 另一种常用的电路划分方法成为contraction。在这一方法中,将量子电路划分为若干个较小的部分,其收缩等于原始电路。对于两个预设整数参数k1和k2,将电路划分为若干小电路。其中每个小电路涉及最多k1个量子比特,并且与至多跨越不同部件的k2个多比特门相连。图\ref{fig:contraction}展示了对Bit flip电路进行k1=3,k2=2的拆分结果。
\begin{figure}[!htbp]
	\centering
	\includegraphics[height=4.5cm]{Img/cir_contraction.pdf}
	\caption{对Bit flip电路进行contraction的拆分}
	\label{fig:contraction}
\end{figure} 


\item \label{contraction}在BDD中,索引的顺序很重要。因为索引顺序会直接影响BDD的大小。一个良好的变量顺序可以使得BDD比一个糟糕的变量顺序小得多。图\ref{fig:bdd-compare}的了两张图都表示了布尔函数ƒ(x1,...,x8)=x1x2+x3x4+x5x6+x7x8,但图\ref{fig:bdd-good}的结构更简单。其中图\ref{fig:bdd-bad}的索引顺序为\{x1,x3,x5,x7,x2,x4,x6,x8\},图\ref{fig:bdd-good}的索引顺序为\{x1,x2,x3,x4,x5,x6,x7,x8\}。找到一个好的索引顺序是一个NP问题。在工程实现中,目前只能通过小规模电路上寻求规律,然后在更大规模电路中应用较优顺序。

\begin{figure}[!htbp]
	\centering
	\begin{subfigure}[b]{.4\textwidth}
        \centering
        \includegraphics[height=4.5cm]{Img/BDD_Variable_Ordering_Bad.svg.pdf}
		\caption{索引顺序为\{x1,x3,x5,x7,x2,x4,x6,x8\}}
		\label{fig:bdd-bad}
	\end{subfigure}
	\begin{subfigure}[b]{.4\textwidth}
        \centering
        \includegraphics[height=5cm]{Img/BDD_Variable_Ordering_Good.svg.pdf}
		\caption{索引顺序为\{x1,x2,x3,x4,x5,x6,x7,x8\}}
		\label{fig:bdd-good}
	\end{subfigure}
	\caption{同一布尔函数在不同索引顺序下的结构图\citep{wiki:bdd}}
	\label{fig:bdd-compare}
\end{figure}


\item 由于量子状态都在同一希尔伯特空间中。因此作用某些算子后,不同的量子状态可能等价。
当存储算子的资源少于存储状态的资源时,就有可能存储算子表示不同的状态\citep{vinkhuijzen2023limdd}。图\ref{fig:qmdd-example}表示了一个QMDD的例子,应用等价性,可以化简为图\ref{fig:limdd-example}。
TDD也可以应用类似技术,进行进一步化简,从而降低资源要求。
\begin{figure}[!htbp]
    \centering
    \begin{subfigure}[b]{.4\textwidth}
        \centering
        \includegraphics[height=6cm]{Img/limdd.pdf}
        \caption{一个QMDD示例}
        \label{fig:qmdd-example}
    \end{subfigure}
    \begin{subfigure}[b]{.4\textwidth}
        \centering
        \includegraphics[height=6cm]{Img/limdd_reduce.pdf}
        \caption{应用等价性化简图\ref{fig:qmdd-example}}
        \label{fig:limdd-example}
    \end{subfigure}
\end{figure}
\end{myen}
\subsection{存在的问题}
在本研究过程中,主要遇到了两个问题,这些问题对研究的深入发展和实际应用产生了重要影响。

首先,面临的一个关键挑战是如何将所提出的方法扩展应用到更大规模的实例。这不仅涉及到算法的效率问题,还包括数据处理能力的提升。对于实现验证量子计算算法在更广泛领域的应用至关重要。为了解决这一挑战,目前采用电路拆分方法来降低TDD的资源消耗,并挖掘可能的并行计算机会。此外,还计划应用更灵活的索引策略和limdd的思路,以期达到更高效的处理效果。

其次,另一个重要的问题是如何将研究方法应用于更加实用的示例。这是将理论研究转化为实际应用的关键一步。目前考虑的主要方向是将此方法应用到量子线路设计,即QDA(Quantum Design Automation)领域。
计划未来能够将本研究成果应用于不同的synthesis算法的验证等价性中,从而在量子计算的实际应用中发挥更大的作用。

\section{系统设计与实现}
\subsection{预期目标}
image computation指的是通过当前系统状态与转移关系,计算下一步系统状态。
在模型检测中,image computation是关键的一步。
过去传统模型检测已发展出多种高效的计算算法,特别是利用二叉决策图(BDDs)符号性表示初始状态和转移关系,以及利用状态空间划分和电路划分加速图像计算过程。然而,量子系统的模型检测尚处于起步阶段。
本次研究的重要目标就是利用TDD,构建量子系统的image computation,从而为未来的工作奠定基础。

考虑到量子计算的特殊性,可以利用一系列的优化策略,主要包括基于张量网络的结构特性和TDD的有效性能。通过这些优化,可以提升量子图像计算的效率,尤其是在处理复杂量子系统时的性能表现。

\subsection{研究方法}
本次研究的主要目的是借助TDD数据结构,构建能快速计算量子模型检测中可达问题的方案。本次研究的主要挑战在于尽可能减少程序的运行时间以及空间资源。为此,需要采用一系列方法来开发更有效的算法,以优化TDD操作和收缩张量网络。其中包括开发新技术来分割张量网络和优化TDD结构。
下面简单介绍以下具体研究方法。

\begin{myen}

\item \label{addition}关于常用的量子线路划分方法,
第一种被称为addition\citep{chen2018classical}。将量子电路视为张量网络,首先将一个量子电路C转换成无向图G。G中的每个节点表示量子电路的一个索引,并且如果它们是相同门的输入或输出索引,则在G中连接两个节点。并且当满足以下两个条件之一时输入和输出索引不变:
\begin{itemize}
	\item 是对角线量子门的输入和输出索引;
	\item 是受控门的控制比特位的输入和输出索引。
\end{itemize}
	
图\ref{fig:addition}展示了Grover\_3电路图的索引链接图。该图描述了量子电路的连通性,通过选择图中连通度最大的索引可以对电路进行分割。因此选择图中连通度较大的$x_1^1,x_1^3x_2^1$可以对电路进行较好的划分。
 
\begin{figure}[!htbp]
	\centering
	\includegraphics[height=6cm]{Img/cir_index_graph.pdf}
	\caption{Grover\_3的索引连接图}
	\label{fig:addition}
\end{figure} 

\item 另一种常用的电路划分方法成为contraction。在这一方法中,将量子电路划分为若干个较小的部分,其收缩等于原始电路。对于两个预设整数参数k1和k2,将电路划分为若干小电路。其中每个小电路涉及最多k1个量子比特,并且与至多跨越不同部件的k2个多比特门相连。图\ref{fig:contraction}展示了对Bit flip电路进行k1=3,k2=2的拆分结果。
\begin{figure}[!htbp]
	\centering
	\includegraphics[height=7cm]{Img/cir_contraction.pdf}
	\caption{对Bit flip电路进行contraction的拆分}
	\label{fig:contraction}
\end{figure} 


\item \label{contraction}在BDD中,索引的顺序很重要。因为索引顺序会直接影响BDD的大小。一个良好的变量顺序可以使得BDD比一个糟糕的变量顺序小得多。图\ref{fig:bdd-compare}的了两张图都表示了布尔函数ƒ(x1,...,x8)=x1x2+x3x4+x5x6+x7x8,但图\ref{fig:bdd-good}的结构更简单。其中图\ref{fig:bdd-bad}的索引顺序为\{x1,x3,x5,x7,x2,x4,x6,x8\},图\ref{fig:bdd-good}的索引顺序为\{x1,x2,x3,x4,x5,x6,x7,x8\}。找到一个好的索引顺序是一个NP问题。在工程实现中,目前只能通过小规模电路上寻求规律,然后在更大规模电路中应用较优顺序。

\begin{figure}[!htbp]
	\centering
	\begin{subfigure}[b]{.4\textwidth}
        \centering
        \includegraphics[height=6cm]{Img/BDD_Variable_Ordering_Bad.svg.pdf}
		\caption{索引顺序为\{x1,x3,x5,x7,x2,x4,x6,x8\}}
		\label{fig:bdd-bad}
	\end{subfigure}
    \qquad
	\begin{subfigure}[b]{.4\textwidth}
        \centering
        \includegraphics[height=6cm]{Img/BDD_Variable_Ordering_Good.svg.pdf}
		\caption{索引顺序为\{x1,x2,x3,x4,x5,x6,x7,x8\}}
		\label{fig:bdd-good}
	\end{subfigure}
	\caption{同一布尔函数在不同索引顺序下的结构图\citep{wiki:bdd}}
	\label{fig:bdd-compare}
\end{figure}


\item 由于量子状态都在同一希尔伯特空间中。因此作用某些算子后,不同的量子状态可能等价。
当存储算子的资源少于存储状态的资源时,就有可能存储算子表示不同的状态\citep{vinkhuijzen2023limdd}。图\ref{fig:qmdd-example}表示了一个QMDD的例子,应用等价性,可以化简为图\ref{fig:limdd-example}。
TDD也可以应用类似技术,进行进一步化简,从而降低资源要求。
\begin{figure}[!htbp]
    \centering
    \begin{subfigure}[b]{.4\textwidth}
        \centering
        \includegraphics[height=8cm]{Img/limdd.pdf}
        \caption{一个QMDD示例}
        \label{fig:qmdd-example}
    \end{subfigure}
    \begin{subfigure}[b]{.4\textwidth}
        \centering
        \includegraphics[height=8cm]{Img/limdd_reduce.pdf}
        \caption{应用等价性化简图\ref{fig:qmdd-example}}
        \label{fig:limdd-example}
    \end{subfigure}
\end{figure}
\end{myen}
\subsection{软件系统实现}
为了实现软件的高效运行,模块化设计至关重要。
每个模块在软件系统中扮演着关键角色,并且具有特定的功能和目的。以下是本次毕业设计中软件必须包含的模块及其重要性的说明:

\begin{itemize}
    \item \textbf{输入处理模块}:该模块的主要职责是处理输入数据,例如接收用OpenQASM格式编写的量子算法代码。其核心功能是将这些代码转换为TDD表示形式。鉴于当前存在多种量子编程语言,此模块的模块化处理能够显著提升系统的灵活性和兼容性。
    \item \textbf{内存管理模块}:本模块负责管理TDD节点的存储和维护。当创建新的TDD节点时,它会运用哈希算法与现有节点进行对比,以避免重复创建相同节点。这种方法不仅减少了内存占用,还提高了处理效率。
    \item \textbf{TDD基础模块}:该模块主要执行TDD节点的压缩操作,或者导出TDD的树状结构图。节点收缩是TDD核心的运算过程,而树状结构图的导出功能则有助于用户更好地理解和分析TDD的结构。
    \item \textbf{TDD算法模块}:此模块为TDD提供更复杂的算法支持。例如,它能够调整节点收缩的顺序,以优化系统运行效率。此外,它还能执行其他高级功能,如检验TDD是否存在于特定子空间中。
\end{itemize}


\input{Tex/chap_yjjz.tex}
\section{研究工作计划与进度安排}
整个研究计划的时间安排如下:
\begin{itemize}
    \item 2023.06至2023.07:收集相关资料,完成初期调研,并撰写开题报告。即在这个阶段,深入研究关于基于TDD的量子模型检测方法、量子线路的可达性、持续可达性和重复可达性的相关文献和资料,并撰写开题报告,明确研究的目标和方法。
    \item 2023.07至2023.10:设计并实现可达空间的计算。在这个阶段,我将根据所掌握的理论知识和之前的调研成果,设计并实现基于TDD的量子模型检测方法中的可达空间计算部分。通过编程和模拟实验,验证该方法的有效性,并进行实验结果的分析和讨论。
    \item 2023.10至2024.01:设计并实现重复可达性和持续可达性的计算。在这个阶段,我将进一步完善基于TDD的量子模型检测方法,设计并实现重复可达性和持续可达性的计算部分。通过实验和对比分析,评估这些计算方法在不同场景下的性能和准确度,并探索其在量子线路验证中的潜在应用。
    \item 2024.01至2024.02:进行工具性能测试、优化和评价。在这个阶段,我将对已实现的基于TDD的量子模型检测工具进行性能测试和评价。根据测试结果,对工具进行必要的优化和改进,以提高其计算效率和准确性。同时,结合实验结果和评估数据,对工具的性能进行客观评价,并提出可能的改进方案。
    \item 2024.03至2024.04:总结研究工作并撰写硕士论文。在这个阶段,我将对整个研究过程进行总结和归纳,梳理研究中的重要发现和创新点。然后,撰写硕士论文,包括引言、相关工作、方法设计、实验结果、分析和讨论等部分,以完整而准确地呈现研究成果。此外,还将对未来可能的研究方向和改进方案进行探讨和展望。
\end{itemize}


% \nocite{*}% 使文献列表显示所有参考文献(包括未引用文献)
%---------------------------------------------------------------------------%
% main content
%-
%-> Appendix
%-

%-
%-> Backmatter: bibliography, glossary, index
%-

\intotoc*{\cleardoublepage}{\bibname}% add link to toc
\artxifstreq{\artxbib}{bibtex}{% enable bibtex
    \bibliography{Biblio/ref}% bibliography
}{%
    \printbibliography% bibliography
}
\cleardoublepage%
\appendix% initialize the environment
\thispagestyle{appendixheader}
\stepcounter{app}
\setcounter{app_fig}{1}
\setcounter{app_tab}{1}
\setcounter{equation}{0}
\renewcommand\theequation{附\arabic{app}-\arabic{equation}}
% \renewcommand\theequation{\Alph{app}.\arabic{equation}}
\renewcommand\chaptername{附录}
\renewcommand\chaptername{Appendix} 
\renewcommand\thechapter{附录\zhnum{app}} 

\setcounter{chapter}{0}
\setcounter{section}{0}
\chapter{部分优化方案的python实现}\label{chap:app1}{
第\ref{sec-optimize}节介绍了针对基于TDD的量子模型检测方法的一些优化策略。第\ref{sec-ex}章节证实了量子拆分技术能够有效降低计算资源消耗,是一种非常实用的优化方案。本附录将以Python为例,介绍这些优化方法的具体实现。

在代码实现中,变量Ts一般表示一个量子迁移系统,其中包含以下变量:
\begin{itemize}
    \item Ts.initial\_states是一个列表,包含所有量子迁移系统的初态。
    \item Ts.operators是一个列表,按顺序包含所有量子线路的操作。
\end{itemize}
基于这样 的量子迁移系统设计,可以设计代码段2。
此代码段核心使用张量收缩技术,以计算量子迁移系统的一次状态转移。具体来说,就是基于系统的初始状态来预测其下一状态空间。这里,simulation函数负责执行量子操作的TDD与量子态的收缩操作,而join函数则用于将得到的新量子状态并入当前状态子空间中。
因此,代码逻辑首先从系统的初始态出发,构建一个初始的状态子空间。接着,它遍历所有量子操作,对每一个初始态执行操作,通过simulation函数计算操作后的结果,并利用join函数将这些新状态并入到状态子空间中,以完成系统下一步状态空间的构建。
\begin{lstlisting}[language=Python, caption={利用TDD收缩直接计算一步迁移\label{code-image}}]
def image(Ts):
    sp= generate_initial_space(Ts.initial_states)
    for op in Ts.operators:
        for phi in Ts.initial_states:
            res=simulation(Ts.operators[op],phi)
            sp=join(sp,res)
    return sp
\end{lstlisting}

类似地,可以设计代码段3中的addition优化方案。其中tn为表示量子线路的量子张量网络。不同在于其中的Ts的operators将会在运行中生成。
其他运行逻辑与代码段2类似,码段2的函数首先生成系统的初始状态子空间。然后,对系统中的每个量子操作,通过函数addition\_partition应用分区优化tn的索引,从而得到Ts的量子操作。这一步调整了量子操作,以便提高执行效率。
在应用了优化后的量子操作进行状态转移的计算过程中,函数对每个初始态进行遍历。对于每个量子操作,它初始化一个表示恒等操作的TDD,作为累加的基础。接着,通过遍历量子操作的所有部分,使用simulation函数计算每部分的效果,并利用add函数将这些TDD累加到res中。由于应用索引划分了线路,相当于对张量网络进行了切片。因此这里需要使用TDD的加法,即add函数。这样最后的res就是对应量子初态在执行所有量子操作后的量子状态。最后通过join函数将该量子状态的结果并入到状态子空间中。
其中的函数addition\_partition如代码段4所示。
\begin{lstlisting}[language=Python, caption=对量子线路应用addition优化方案计算一步迁移]
def image_add_par(tn,Ts,k=0):
    sp= generate_initial_space(Ts.initial_states)

    Ts.operators[op]=addition_partition(tn,k)

    for op in Ts.operators:
        for phi in Ts.initial_states:
            res=get_identity_tdd()
            for p in Ts.operators[op]:
                temp=simulation(p,phi)
                res=add(res,temp)
            sp=join(sp,res)
    return sp
\end{lstlisting}
\begin{lstlisting}[language=Python, caption={addition方案中的电路划分函数}]
def addition_partition(tn,k=0):
    """partition a tensor network tn according to k nodes with biggest degree"""

    lin_graph.add_nodes_from(tn.index_set)

    deg=[lin_graph.degree(key) for key in lin_graph.nodes]

    big_degres=[]
    node_list=list(lin_graph.nodes)

    t = k
    while t:
        biggest_degree=max(deg)
        if biggest_degree<=0:
            break
        idx=deg.index(biggest_degree)
        big_degres.append(node_list[idx])
        deg[idx]=0
        t-=1

    dd_list=[]
    for t in range(2**k):
        b=list(bin(t)[2:])
        b.reverse()
        temp_tn=copy.copy(tn)
        temp_tn.tensors=[]
        for tensor in tn.tensors:
            temp_ts=copy.copy(tensor)
            for idx in tensor.index_set:
                if idx.key in big_degres:
                    if big_degres.index(idx.key)< len(b) and b[big_degres.index(idx.key)]=='1':
                        temp_ts.dd=cont(temp_ts.tdd(),Tensor(ket1,[idx]).tdd())
                    else:
                        temp_ts.dd=cont(temp_ts.tdd(),Tensor(ket0,[idx]).tdd())
            temp_tn.tensors.append(temp_ts)
        dd_list.append(temp_tn.cont())

    return dd_list
\end{lstlisting}


同样也可以设计对量子线路应用contraction的优化方案,具体如代码段5。该函数利用两个参数k1和k2来调节收缩分区的策略,以达到更优的计算性能。函数的输入的tn和Ts的情况和addition方案相同。tn也是表示量子线路的量子张量网络。Ts的operators同样会在运行中生成。
初始步骤与之前类似,函数从生成系统初始状态的子空间开始。不同在于对于量子系统中定义的每个量子操作,根据参数k1和k2,使用contraction\_partition函数对其进行分区。在这一步中Ts中的量子操作将会按块划分,因此之后的Ts中的operators是一个列表,包含所有在该块的量子操作。
这一步骤的目的是重新组织量子操作的结构,使得后续的计算过程更加高效。
在经过优化的量子操作上执行状态转移计算时,该函数遍历所有初始状态。对于每一个量子操作,它从当前的量子状态开始,逐步应用经过收缩优化的量子操作的各个部分。这里的simulation函数用于计算经过每一步操作后的结果。每次操作后得到的新状态通过join函数并入到状态子空间中。其中的函数contraction\_partition如代码段6所示。
\begin{lstlisting}[language=Python, caption=对量子线路应用contraction优化方案计算一步迁移]
def image_cont_par(tn,Ts,k1=0,k2=0):
    sp= generate_initial_space(Ts.initial_states)

    Ts.operators=contraction_partition(tn,k1,k2)

    for op in Ts.operators:
        for phi in Ts.initial_states:
            res=phi
            for p in Ts.operators[op]:
                res=simulation(p,res)
            sp=join(sp,res)
    return sp
\end{lstlisting}

\begin{lstlisting}[language=Python, caption={contraction方案中的电路划分函数}]
def contraction_partition(tn,k1=0,k2=0):
    dd_list=[]

    cir_par=circuit_partition(tn,k1,k2)
    for ver in cir_par:
        for hor in ver:
            tdd,node = TensorNetwork(hor).cont()
            dd_list.append(tdd)

    return dd_list

def circuit_partition(tn,k1=0,k2=0):
    """The first partition scheme;
    cx_max is the number of CNOTs allowed to be cut
    """

    num_qubit=tn.qubits_num

    if k1==0:
        blocks_num=2
        k1=np.ceil(num_qubit/2)
    else:
        blocks_num=int(np.ceil(num_qubit/k1))
    res=[[[] for _ in range(blocks_num)]]

    if k2==0:
        cx_max=num_qubit//2
    else:
        cx_max=k2

    cx_num=0
    level=0

    for tensor in tn.tensors:
        qubit_list = tensor.qubits
        mi=int(min(qubit_list)//k1)
        ma=int(max(qubit_list)//k1)
        if mi==ma:
            res[level][mi].append(tensor)
        else:
            cx_num+=1
            if cx_num<=cx_max:
                res[level][int(qubit_list[-1]//k1)].append(tensor)
            else:
                level+=1
                res.append([])
                for k in range(blocks_num):
                    res[level].append([])
                res[level][int(qubit_list[-1]//k1)].append(tensor)
                cx_num=1

    return res
\end{lstlisting}
% \begin{lstlisting}[language=Python, caption=对TDD运用窗函数划分]
% def tdd_partition(tdd: TDD, windows: list[dict]) -> list[TDD]:
%     partitions = [partition(tdd.node, window) for window in windows]
%     for partitioned_tdd in partitions:
%         G.tdd_initial(partitioned_tdd,tdd)
%         if partitioned_tdd.weight:
%             partitioned_tdd.weight = tdd.weight

%     return partitions
% \end{lstlisting}
% \begin{lstlisting}[language=Python, caption=对TDD进行分割]
% def tdd_split(tdd:TDD, split_point:int)-> list[TDD,TDD]:
%     B= split_before(tdd.node, split_point)
%     A= split_after(tdd.node, split_point)
%     return generate_tdd_index(B, A, tdd, split_point)
% \end{lstlisting}
\thispagestyle{appendixheader}
}% appendix content

\thispagestyle{appendixheader}
\backmatter% initialize the environment
%---------------------------------------------------------------------------%
%->> Backmatter
%---------------------------------------------------------------------------%
% //TODO: bind

\chapter[致谢]{致\quad 谢}\chaptermark{致\quad 谢}% syntax: \chapter[目录]{标题}\chaptermark{页眉}
%\thispagestyle{noheaderstyle}% 如果需要移除当前页的页眉
%\pagestyle{noheaderstyle}% 如果需要移除整章的页眉

% 感谢应圣钢老师的教导,感谢洪鑫的xx。感谢朋友,家人。
首先,我要感谢我的导师应圣钢。在过去的三年里,他对我的悉心指导、宝贵建议和无私支持,是我顺利完成学位论文的关键。应老师严谨的科研作风和耐心的指导建议给予了我很大启迪,将影响我终生。

其次,我要诚挚感谢洪鑫学长在研究工作中给予的大力协助。洪鑫学长理论知识丰富,科研写作技能扎实。我们通力合作取得了理想的成果。同时也要感谢实验室其他同学在学习和生活上的友善帮助。比如我的舍友高敏博。

最后,我由衷感谢我的朋友和家人。是你们一如既往的关爱与支持,使我顺利走过了人生这个重要的阶段。

\rightline{2024年6月}
\chapter{作者简历及攻读学位期间发表的学术论文与其他相关学术成果}

\section*{作者简历:}
2017年9月——2021年6月,在西安电子科技大学大学计算机科学与技术院获得学士学位。


2021年9月——2024年6月,在中国科学院软件研究所攻读硕士学位。


\section*{已完成的学术论文:}

{
\setlist[enumerate]{}% restore default behavior
\begin{enumerate}[nosep]
    \item Xing Hong, Dingchao Gao, Sangjiang Li, Shenggang Ying, Mingsheng Ying. \textbf{Image Computation for Quantum Transition Systems}.
\end{enumerate}
}

% \section*{申请或已获得的专利:}

% (无专利时此项不必列出)

% \section*{参加的研究项目及获奖情况:}


\cleardoublepage[plain]% 让文档总是结束于偶数页,可根据需要设定页眉页脚样式,如 [noheaderstyle]
%---------------------------------------------------------------------------%
% other information
\end{document}
%---------------------------------------------------------------------------%

