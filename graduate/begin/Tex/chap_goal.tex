\section{课题主要研究内容、预期目标}
本次课题主要围绕利用基于TDD的量子模型检测,对量子线路的可达性(reachability)、持续可达性(persistence property)以及重复可达性(repeated property)进行计算与验证。TDD提供的计算便捷性将极大地保证工具的实用性。
\subsection{主要研究内容}
在模型检测中,有三类比较重要的可达性问题,分别是可达性、持续可达性以及重复可达性。本研究主要应用基于TDD的量子模型检测,探究可达性问题,从而解决实际量子线路问题。

比如RUS电路等价性问题。图表 16展示了两个RUS量子电路。当电路中测量门结果为$00$时,第三个比特的输出态均为$\left(I+2iZ\right)/ \sqrt 5 |\psi\rangle$。这样的等价性检验任务,可以转换为模型检测中的可达空间是否一致。
 	 
图表 16 两个等价的RUS电路

\subsection{预期目标}
本研究旨在通过优化计算算法、改进数据结构等方法,显著降低计算时间,使量子模型检测方法更加实用和高效。具体有以下小目标:
	对于给定的量子系统,能够应用模型检测技术,计算系统的可达性、持续可达性和重复可达性。
	提供支持量子系统验证和验证工具开发的基础研究,即利用可达性的成果为量子系统的验证和验证工具的开发提供支持。
通过实现以上目标,希望在量子计算领域的模型检测研究中取得一定的突破。从而进一步为实际量子系统的设计、验证和优化提供有力的支持。
