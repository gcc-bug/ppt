\section{已有科研基础与所需的科研条件}
在模型检测中,从目前已知状态出发,计算下一状态的image computation起到了关键的作用。这一步目前已经完成。而得益于导师和研究组师兄的帮助,后续过程也会比较顺利。
\subsection{研究基础}
在模型检测中,image computation指的是在给定当前状态$s_i\in S$和行为$\alpha\in Act$的情况下计算接下来的状态。

目前,关于使用TDD对量子的image computation的计算已经完成。表格 2给出了在不同电路拆分技术下Grover算法的计算时间,单位为秒。其中basic表示没有使用优化技术,addition表示使用错误!未找到引用源。中的addition优化技术,contraction表示使用错误!未找到引用源。中的contraction优化技术。“-”表示超过一小时的运行上限。

表格 2 不同电路拆分技术下的Grover算法运行时间
% Benchmark	basic	addition	contraction
% Grover_15	19.33	17.35	1.61
% Grover_18	76.47	66.02	2.41
% Grover_20	294.65	259.87	4.39
% Grover_30	-	-	96.02
% Grover_40	-	-	2953.57

通过对比不同优化技术下的计算时间,可以看到使用优化技术能够显著降低计算时间。例如,在Grover-20的例子中,使用"contraction"优化技术的情况下,计算时间从294秒降低到了4秒。这表明优化技术在提高计算效率方面起到了积极的作用。

\subsection{必要研究条件}
在硬件方面,实验室提供的服务器为运行程序提供了便利。对于经典计算机模拟量子计算而言,内存容量一直是主要的瓶颈之一。然而,实验室服务器提供了大量的内存空间,为程序的运行提供了必要的条件。这意味着我可以更轻松地处理复杂的计算任务,探索量子线路的可达性、持续可达性以及重复可达性。

在软件层面,实验室内的老师和师兄们拥有丰富的研究经验,他们的知识和经验将为选题的完成提供必要的支持。举例来说,我的导师是应圣钢老师,他在博士期间主要致力于量子马尔可夫链的可达性分析。他的研究背景与我的选题非常契合,而且他在学术指导方面也表现出耐心和专业性。有他的指导,我可以充分利用他的专业知识和研究经验,解决在研究过程中可能遇到的问题。

实验室提供的硬件和导师的支持为我的研究工作奠定了坚实的基础。我相信在这样的环境下,我将能够深入探索基于TDD的量子模型检测方法,以及对量子线路的可达性、持续可达性和重复可达性进行计算与验证。
