\section{已有科研基础与所需的科研条件}
在模型检测中,从目前已知状态出发,计算下一状态的image computation起到了关键的作用。这一步目前已经完成。而得益于导师和研究组师兄的帮助,后续过程也会比较顺利。
\subsection{研究基础}
在模型检测中,image computation指的是在给定当前状态$s_i\in S$和行为$\alpha\in Act$的情况下计算接下来的状态。

目前,关于使用TDD对量子的image computation的计算已经完成。表\ref{table:time}给出了在不同电路拆分技术下Grover算法的计算时间,单位为秒。其中basic表示没有使用优化技术,addition表示使用\ref{addition}中的addition优化技术,contraction表示使用\ref{contraction}中的contraction优化技术。“-”表示超过一小时的运行上限。

\begin{table}[!htbp]
    \centering
    \begin{tabular}{@{}llll@{}}
        \toprule
        Benchmark  & basic  & addition & contraction \\ \midrule
        Grover\_15 & 19.33  & 17.35    & 1.61        \\
        Grover\_18 & 76.47  & 66.02    & 2.41        \\
        Grover\_20 & 294.65 & 259.87   & 4.39        \\
        Grover\_30 & -      & -        & 96.02       \\
        Grover\_40 & -      & -        & 2953.57     \\ \bottomrule
    \end{tabular}
    \caption{不同电路拆分技术下的Grover算法运行时间}
    \label{table:time}
\end{table}

通过对比不同优化技术下的计算时间,可以看到使用优化技术能够显著降低计算时间。例如,在Grover-20的例子中,使用"contraction"优化技术的情况下,计算时间从294秒降低到了4秒。这表明优化技术在提高计算效率方面起到了积极的作用。

\subsection{必要研究条件}
在硬件方面,实验室提供的服务器为运行程序提供了便利。对于经典计算机模拟量子计算而言,内存容量一直是主要的瓶颈之一。然而,实验室服务器提供了大量的内存空间,为程序的运行提供了必要的条件。这意味着我可以更轻松地处理复杂的计算任务,探索量子线路的可达性、持续可达性以及重复可达性。

在软件层面,实验室内的老师和师兄们拥有丰富的研究经验,他们的知识和经验将为选题的完成提供必要的支持。举例来说,我的导师是应圣钢老师,他在博士期间主要致力于量子马尔可夫链的可达性分析。他的研究背景与我的选题非常契合,而且他在学术指导方面也表现出耐心和专业性。有他的指导,我可以充分利用他的专业知识和研究经验,解决在研究过程中可能遇到的问题。

实验室提供的硬件和导师的支持为我的研究工作奠定了坚实的基础。我相信在这样的环境下,我将能够深入探索基于TDD的量子模型检测方法,以及对量子线路的可达性、持续可达性和重复可达性进行计算与验证。
