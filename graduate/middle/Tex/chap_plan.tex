\section{研究工作计划与进度安排}
% //TODO: 需要调整内容
整个研究计划的时间安排如下:
\begin{itemize}
    \item 2023.06至2023.07:收集相关资料,完成初期调研,并撰写开题报告。即在这个阶段,深入研究关于基于TDD的量子模型检测方法、量子线路的可达性、持续可达性和重复可达性的相关文献和资料,并撰写开题报告,明确研究的目标和方法。
    \item 2023.07至2023.10:设计并实现可达空间的计算。在这个阶段,我将根据所掌握的理论知识和之前的调研成果,设计并实现基于TDD的量子模型检测方法中的可达空间计算部分。通过编程和模拟实验,验证该方法的有效性,并进行实验结果的分析和讨论。
    \item 2023.10至2024.01:设计并实现重复可达性和持续可达性的计算。在这个阶段,我将进一步完善基于TDD的量子模型检测方法,设计并实现重复可达性和持续可达性的计算部分。通过实验和对比分析,评估这些计算方法在不同场景下的性能和准确度,并探索其在量子线路验证中的潜在应用。
    \item 2024.01至2024.02:进行工具性能测试、优化和评价。在这个阶段,我将对已实现的基于TDD的量子模型检测工具进行性能测试和评价。根据测试结果,对工具进行必要的优化和改进,以提高其计算效率和准确性。同时,结合实验结果和评估数据,对工具的性能进行客观评价,并提出可能的改进方案。
    \item 2024.03至2024.04:总结研究工作并撰写硕士论文。在这个阶段,我将对整个研究过程进行总结和归纳,梳理研究中的重要发现和创新点。然后,撰写硕士论文,包括引言、相关工作、方法设计、实验结果、分析和讨论等部分,以完整而准确地呈现研究成果。此外,还将对未来可能的研究方向和改进方案进行探讨和展望。
\end{itemize}

