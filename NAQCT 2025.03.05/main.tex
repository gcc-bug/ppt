\documentclass{beamer}
\usepackage{graphicx}
\usepackage{tikz}
\usepackage{yquant}
\useyquantlanguage{groups}
\usepackage{pgfplots}
\usepackage{amsmath}
\pgfplotsset{compat=1.18}
\title{Qubit Movement-Optimized Program Generation on Zoned Neutral Atom Processors}
\author{Enhyeok Jang, Youngmin Kim, Hyungseok Kim, Seungwoo Choi, \textbf{Yipeng Huang}, and Won Woo Ro}
\institute{Yonsei University, Rutgers University}
\date{CGO ’25, March 01–05, 2025}

\begin{document}
	
	\frame{\titlepage}
	
	% Slide 1: Introduction
	\begin{frame}{Introduction}
		\begin{itemize}
			\item Zoned architectures improve gate fidelity ($>99.9\%$ for 1Q, $>99.5\%$ for 2Q).
%			\item However, inter-zone qubit movements take significantly longer than gate execution, leading to high execution overhead.
			\item Standard quantum program structures are inefficient on zoned architectures, requiring excessive zone-to-zone transfers.
			
		\end{itemize}
		\begin{figure}
			\centering
			\includegraphics[width=.8\textwidth]{figure/zone.png}
			\caption{The entangling zone is dedicated to executing 2-qubit gates, while the storage zone is for single qubit gate.}
		\end{figure}
	\end{frame}
	
	% Slide 2: Motivation
	\begin{frame}{Motivation}
		\begin{itemize}
			\item Naïve program execution causes 78-89\% of runtime to be spent on zone-to-zone qubit transfers.
%			\item Gate structures such as CZ chains and ZZ interactions induce frequent interleavings of single- and two-qubit gates.
%			\item Existing neutral atom compilers do not sufficiently optimize inter-zone movements.
			\item \textbf{Goal:} Optimize program execution by reducing inter-zone qubit movements.
		\end{itemize}
		\begin{figure}
			\centering
			\includegraphics[width=.8\textwidth]{figure/intro.png}
			\caption{Execution time breakdown on zoned architectures.}
		\end{figure}
	\end{frame}
	
	% Slide 3: Mantra - Overview
	\begin{frame}{Mantra: Key Techniques}
		\textbf{Mantra (Minimizing trAp movemeNts for aTom aRray Architectures)} introduces:
		\begin{itemize}
			\item \textbf{Fountain-Shaped CZ Chain:} Reduces single-qubit gate overhead and cancels intermediate gates..
			\item \textbf{Preemptive Gate Scheduling:} Executes independent gates earlier to minimize inter-zone transitions.
			\item \textbf{1Q-Gateless ZZ Interaction:} Replaces CZ-based decompositions with direct Rydberg-mediated ZZ rotations
		\end{itemize}
		\textbf{Key Idea:} Reduces inter-zone movements and improves overall execution efficiency by \textbf{rewriting quantum programs} to mitigate frequent transitions between single-qubit and two-qubit gate execution.
	\end{frame}
	
	% Slide 4: Fountain-Shaped CZ Chain
	\begin{frame}{Fountain-Shaped CZ Chain}
		\begin{itemize}
%			\item Standard chain structures cause excessive inter-zone movements due to alternating 1Q/2Q execution.
			\item Structure CZ chains around a common qubit.
		\end{itemize}
		\centering
		\begin{figure}
			\resizebox{!}{12ex}{
			\begin{tikzpicture}
				\begin{yquantgroup}
					\registers{
						qubit {} q[4];
					}
					\circuit{
						 cnot q[2] | q[3];
						 cnot q[1] | q[2];
						 cnot q[0] | q[1];
						 box {$Rz$} q[0];
						 cnot q[0] | q[1];
						 cnot q[1] | q[2];
						 cnot q[2] | q[3];
					}
					\equals
					\circuit{
						cnot q[0] | q[3];
						cnot q[0] | q[2];
						cnot q[0] | q[1];
						box {$Rz$} q[0];
						cnot q[0] | q[1];
						cnot q[0] | q[2];
						cnot q[0] | q[3];
					}
				\end{yquantgroup}
			\end{tikzpicture}}
		\caption[]{An example Hamiltonian simulation kernel $e^{-i\theta ZZZZ}$}
		\end{figure}
		\begin{figure}
			\resizebox{!}{12ex}{
				\begin{tikzpicture}
					\begin{yquantgroup}
						\registers{
							qubit {} q[4];
						}
						\circuit{
							h q[0];
							cnot q[1] | q[0];
							cnot q[2] | q[1];
							cnot q[3] | q[2];
						}
						\equals
						\circuit{
							h q[0];
							cnot q[1] | q[0];
							cnot q[2] | q[0];
							cnot q[3] | q[0];
						}
					\end{yquantgroup}
			\end{tikzpicture}}
			\caption[]{An example of a 4-qubit GHZ-state circuit.}
		\end{figure}
	\end{frame}
	
	% Slide 6: Preemptive Gate Scheduling
	\begin{frame}{Preemptive Gate Scheduling}
		\begin{itemize}
			\item Identifies independent gates that can be executed earlier in the same zone.
%			\item \textbf{Example:} GHZ state preparation.
			\item Reduces zone-to-zone movements while preserving computation dependencies.
		\end{itemize}
		\centering
		\begin{figure}
			\resizebox{!}{16ex}{
				\begin{tikzpicture}
					\begin{yquantgroup}
						\registers{
							qubit {} q[8];
						}
						\circuit{
							h q[0];
							cnot q[1] | q[0];
							cnot q[2] | q[0];
							cnot q[3] | q[0];
							cnot q[4] | q[0];
							cnot q[5] | q[0];
							cnot q[6] | q[0];
							cnot q[7] | q[0];
						}
						\equals
						\circuit{
							h q[0];
							cnot q[4] | q[0];
							
							cnot q[2] | q[0];
							cnot q[6] | q[4];
							
							cnot q[1] | q[0];
							cnot q[3] | q[2];
							cnot q[5] | q[4];
							cnot q[7] | q[6];
						}
					\end{yquantgroup}
			\end{tikzpicture}}
			\caption[]{An example of a 8-qubit GHZ-state circuit.}
		\end{figure}
	\end{frame}
	
	% Slide 5: 1Q-Gateless ZZ Interaction
	\begin{frame}{1Q-Gateless ZZ Interaction}
		\begin{itemize}
%			\item Standard implementations require interleaving 1Q and 2Q gates, causing frequent zone switching.
			\item \textbf{Mantra’s approach:} Uses a combination of adiabatic and Levine-Pichler gates to achieve direct ZZ rotations.
%			\item Eliminates the need for storage zone transitions during ZZ interactions.
		\end{itemize}
		\centering
		\begin{figure}
			\resizebox{!}{8ex}{
				\begin{tikzpicture}
					\begin{yquantgroup}
						\registers{
							qubit {} q[2];
						}
						\circuit{
							cnot q[1] | q[0];
							box {$RZ(\gamma)$} q[1];
							cnot q[1] | q[0];
						}
						\equals
						\circuit{
							box {$LP(\gamma)=
									\begin{bmatrix}
										1 & 0 & 0 & 0 \\
										0 & e^{-i\gamma} & 0 & 0 \\
										1 & 0 & e^{-i\gamma} & 0 \\
										1 & 0 & 0 & e^{2\gamma + \pi}
									\end{bmatrix}$} (q[0],q[1]);
							box {$Ad(\phi_1,\phi_2)=
								\begin{bmatrix}
									1 & 0 & 0 & 0 \\
									0 & 1 & 0 & 0 \\
									1 & 0 & 1 & 0 \\
									1 & 0 & 0 & e^{\phi_2 -2\phi_1}
								\end{bmatrix}$} (q[0],q[1]);
						}
					\end{yquantgroup}
			\end{tikzpicture}}
			\caption[]{The proposed arbitrary ZZ rotation protocol consisting of a single adiabatic (Ad) and a single Levine-Pichler (LP) gate, where $\phi_1 = (\pi+2\gamma+\phi_2)/2$.}
		\end{figure}
	\end{frame}

	
	% Slide 7: Results
	\begin{frame}{Results: Performance Improvement}
%		\begin{itemize}
%			\item Mantra reduces inter-zone movements by 68\%, leading to significant execution time savings.
%			\item .
%			\item Benchmarks (GHZ, QAOA, UCCSD, QNN) show up to 57-87\% reduction in execution time.
%		\end{itemize}
		\centering
		\begin{figure}
			\centering
			\includegraphics[width=.8\textwidth]{figure/res.png}
			\caption[]{Mantra reduces inter-zone movements by 86.6\%, also reduces physical gate counts by 35\%, improving overall fidelity by 17\%}
		\end{figure}
	\end{frame}
	
	\begin{frame}{Execution Breakdown According to Compiler}
		%		\begin{itemize}
			%			\item Mantra reduces inter-zone movements by 68\%, leading to significant execution time savings.
			%			\item .
			%			\item Benchmarks (GHZ, QAOA, UCCSD, QNN) show up to 57-87\% reduction in execution time.
			%		\end{itemize}
		\centering
		\begin{figure}
			\centering
			\includegraphics[width=.8\textwidth]{figure/different.png}
			\caption[]{Execution time breakdown analysis. Combining Atomique and Mantra could further enhance the performance of zoned architectures.}
		\end{figure}
	\end{frame}
	
%	% Slide 8: Discussion
%	\begin{frame}{Discussion and Future Directions}
%		\begin{itemize}
%			\item Zoned architectures are promising but require tailored compilation techniques.
%			\item Mantra improves execution efficiency by minimizing inter-zone movement overhead.
%			\item Future work: Extend optimization strategies for fault-tolerant quantum applications.
%		\end{itemize}
%	\end{frame}
	
	% Slide 9: Conclusion
	\begin{frame}{Conclusion}
		\begin{itemize}
			\item Zoned neutral atom architectures have high gate fidelities but suffer from slow qubit movement.
			\item Mantra optimizes qubit movement but depends on CZ-chain flexibility. Fixed-structure circuits cannot benefit from it.
			\item Results demonstrate significant reductions in execution time and improved fidelity.
%			\item These optimizations bridge the gap between algorithm design and hardware execution.
		\end{itemize}
	\end{frame}
	
	\begin{frame}{Discussion: Co-desgin between Mapping with Synthesis}
		\begin{figure}
			\centering
			\includegraphics[width=.9\textwidth]{figure/dis.png}
			\caption[]{Illustration of the three steps for platform-dependent compilation.}
		\end{figure}
	\end{frame}
\end{document}
