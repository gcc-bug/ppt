\documentclass[aspectratio=1610]{beamer}
\usetheme{Madrid}

\usepackage{amsmath}
\usepackage{graphicx}
\usepackage{amssymb}
\usepackage{listings}
\usepackage{booktabs}
\usepackage{multirow}
\usepackage{lmodern}
\usepackage{xcolor}
\usepackage{float}
\lstset{
  language=Python,
  rulesepcolor=\color{red!20!green!20!blue!20},
  keywordstyle=\color{blue!90}\bfseries,
  commentstyle=\color{red!10!green!70}\textit,
  basicstyle=\footnotesize,
  showstringspaces=true,
  stringstyle=\rmfamily\slshape\color[RGB]{128,0,0},
  breaklines=true,
  extendedchars=false,
  escapeinside=``,
  texcl=true
}

\lstset{breaklines}
\lstset{extendedchars=false}

\usepackage{textcomp}
\usepackage{pythonhighlight}
\usepackage[backend=bibtex,style=authoryear]{biblatex}
\addbibresource{reference.bib}

\usepackage{algorithm}
\usepackage{algorithmic}
\renewcommand{\algorithmicrequire}{\textbf{Input:}}
\renewcommand{\algorithmicensure}{\textbf{Output:}}


\title[gates \& synthesis]{realization of three qubit-gates and synthesis Algorithm}
\author[Gcc]{Dingchao Gao}
\institute[ISCAS]{Institute of Software Chinese Academy of Sciences}

\begin{document}
\frame{\titlepage}

\begin{frame}
\frametitle{Table of Contents}
\tableofcontents[hideallsubsections]
\end{frame}


\section{Realization of a General Three-Qubit Quantum Gate}
\subsection{Two-Qubit Quantum Gates Recap}
\begin{frame}
\frametitle{Two-Qubit Quantum Gates Recap}
\begin{figure}
  \includegraphics[width=\textwidth]{figure/two-qubit.png}
  \caption{Optimal Quantum Circuits for General Two-Qubit Gates are 15 single qubit gates and 3 CNOT gates}
\end{figure}
\end{frame}

\subsection{Introduction to Three-Qubit Quantum Gates}
\begin{frame}
\frametitle{Realization of Three-Qubit Quantum Gates}
\begin{itemize}
  \item universal gate family: $\{R_z,R_y,CNOT\}$
  \item basic idea: KAK Algorithm\cite{kak}, block-diagonal matrix
  \item results:98 single qubits and 40 CNOT gates
\end{itemize}
% introduce three qubit gates and the tools
\end{frame}

\subsection{Realization Techniques}
\subsubsection{Decomposition}
\begin{frame}
\frametitle{KAK Decomposition}
\begin{align}
  x = a
\end{align}
% AKA results and step followed
\end{frame}

\subsubsection{Diagonalization}
\begin{frame}
\frametitle{Diagonalization}
...
% how and results figure
\end{frame}

\subsubsection{Block-Diagonalization}
\begin{frame}
\frametitle{Block-Diagonalization}
...
% result and way
\end{frame}

\subsubsection{Counter}
\begin{frame}
\frametitle{Counter}
...
%  the whole number
\end{frame}

\subsection{Later Work}
\begin{frame}
\frametitle{Later Work}
...
% the later AKA
\end{frame}

\section{QFAST Algorithm}

\subsection{Top-Down vs Bottom-Up Synthesizers}
\begin{frame}
\frametitle{Top-Down vs Bottom-Up Synthesizers}
...
% figure here
\end{frame}

\subsection{Introduction to QFAST Algorithm}
\begin{frame}
\frametitle{Introduction to QFAST Algorithm}
...
% work flow here
\end{frame}

\subsection{Gate Representation}
\begin{frame}
\frametitle{Gate Representation}
...
%  result formula
\end{frame}

\subsection{Cost Function for Optimization}
\begin{frame}
\frametitle{Cost Function for Optimization}
...
% function here and how to optimize
\end{frame}

\subsection{Comparison with Other Algorithms}
\begin{frame}
\frametitle{Comparison with Other Algorithms}
...
% how to compare and the metrics
\end{frame}
\begin{frame}
\frametitle{Comparison results}
\end{frame}
\subsection{Limitations}
\begin{frame}
\frametitle{Limitations}
...
% limitation here
\end{frame}

\section{Conclusion}
\begin{frame}
\frametitle{Summary}
...
% summary something
\end{frame}

\section{References}
\begin{frame}[noframenumbering,allowframebreaks,t]
	\frametitle{references}
	\printbibliography
\end{frame}
\begin{frame}
  \centering
  \Huge{END\\Thank you}
\end{frame}

\end{document}
