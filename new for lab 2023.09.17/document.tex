\documentclass[18 pt]{article}
\usepackage[slantfont,boldfont]{xeCJK}
\usepackage{fontspec}
\setCJKmainfont{SimSun}
\setmainfont{SimSun}
\setsansfont{SimSun}
\title{迎新报告}
\author{高丁超}
\linespread{1.5}
\begin{document}
	\maketitle
	
	
	各位新同学大家好,我是高丁超。欢迎各位加入国重实验室这个大家庭。很高兴有机会在迎新报告会上与大家分享我的学习和生活经历。首先,我要对学校组织此次报告会表示衷心的感谢。
	
	我的研究方向是量子计算这个令人兴奋的前沿领域。随着量子计算理论和技术的发展,未来量子计算机有望实现远超经典计算机的强大计算能力。我目前的工作是基于张量网络的框架,对量子计算模型进行形式化验证。通过构建量子程序的张量网络表示,并进行等价性检验,可以自动化地验证量子算法的正确性。这为构建大规模可靠的量子计算系统奠定了基础。
	
	研究生的学习任务非常重,长时间的高强度工作很容易导致身心的疲惫。所以我强烈建议大家要重视体育锻炼,保持身心健康。雁栖湖校区为我们提供了许多软件所没有的运动环境和设施。比如标准化的户外足球场、网球场,可以让我们尽情挥洒热情。室外游泳池和怀柔滑雪场等设施,丰富了我们的运动方式。我认为大家应该充分利用这些资源,选修体育课程,组织周末运动,减减压力,增强团队协作能力。
	
	除了运动设施,雁栖湖的其他硬件条件也非常优越。图书馆提供了充足的学习资料,值得我们花时间去挖掘。雁栖湖宿舍舒适宽敞,是一个交流和思考的好场所。怀柔美食和古镇景点也值得大家探索。比如怀柔的炒栗子。
	
	
	最后,我再次感谢大家的聆听。希望我们在实验室这个充满活力的环境里,努力学习、勇于探索,取得优异的成绩。
\end{document}