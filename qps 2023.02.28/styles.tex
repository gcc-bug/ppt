
\usepackage{amsmath}
\usepackage{amssymb}
\usepackage{listings}
\usepackage{booktabs}
\usepackage{multirow}
\usepackage{multirow}
\usepackage{lmodern}
\usepackage{xcolor}
\usepackage{float}
\lstset{
  language=Python,  %代码语言使用的是matlab
  % frame=shadowbox, %把代码用带有阴影的框圈起来
  rulesepcolor=\color{red!20!green!20!blue!20},%代码块边框为淡青色
  keywordstyle=\color{blue!90}\bfseries, %代码关键字的颜色为蓝色,粗体
  commentstyle=\color{red!10!green!70}\textit,    % 设置代码注释的颜色
  basicstyle=\footnotesize,
  showstringspaces=true,%不显示代码字符串中间的空格标记
  % numbers=left, % 显示行号
  % numberstyle=8pt,    % 行号字体
  % numberstyle=\color{green},
  stringstyle=\rmfamily\slshape\color[RGB]{128,0,0}, % 代码字符串的特殊格式
  breaklines=true, %对过长的代码自动换行
  extendedchars=false,  %解决代码跨页时,章节标题,页眉等汉字不显示的问题
  escapeinside=``,%代码中出现中文必须加上,否则报错
  texcl=true}

\lstset{breaklines}%自动将长的代码行换行排版

\lstset{extendedchars=false}%解决代码跨页时,章节标题,页眉等汉字不显示的问题

\usepackage{textcomp}
% \usepackage[margin=1in]{geometry}
\usepackage{pythonhighlight}
% \usepackage{minted}
\usepackage[backend=bibtex]{biblatex}
%\usepackage[style=authortitle,backend=biber]{biblatex}
\addbibresource{ResearchRabbit_Export_2022_10_20.bib}

\usepackage{algorithm}
\usepackage{algorithmic}
\renewcommand{\algorithmicrequire}{\textbf{Input:}}
\renewcommand{\algorithmicensure}{\textbf{Output:}}
