\section{Induction}
\subsection{background}
\begin{frame}
    \frametitle{backgroud}
    \begin{itemize}
        \item a system $S$
        \item property $\phi$
        \item verification: Is there a behavior of $S$ that violated the property?
    \end{itemize}
\end{frame}
\begin{frame}
    \frametitle{transition system}
    \begin{itemize}
        \item transition system $TS=(S,S_{0},R)$
        \item a safe property $P\subseteq S$
        \item a inductive invariants $I$ iff.
        \begin{itemize}
            \item $S_{0}\subseteq I$
            \item $R(I)=I$
            \item $I\subseteq P$
        \end{itemize}
    \end{itemize}
\end{frame}

\begin{frame}
    \frametitle{poor logic VS. rich logic}
    \begin{itemize}
        \item compare logic:
        \item why poor logic:

% 需要二阶逻辑的一个原因是一阶逻辑不能表达一些重要的数学概念和性质,比如实数的完备性、集合论的公理等²⁴。二阶逻辑可以量化谓词,而不仅仅是个体,从而可以描述更多的集合和关系¹³⁵。二阶逻辑也可以定义自然数和算术运算,从而证明一些基本的数学定理⁶。

% 一阶逻辑的不可判定类是指一些由一阶逻辑公式组成的集合,对于这些集合中的任意公式,都不存在一个通用的算法能够判断其是否有效1。一阶逻辑的不可判定类有很多,例如:
% 算术公式类:包含自然数、加法和乘法运算、等号和量词的公式2。
% 预言逻辑类:包含个体常量、谓词符号、等号和量词的公式3。
% 模态逻辑类:包含个体常量、谓词符号、等号、量词和模态运算符(如必然性和可能性)的公式3。

% 好的,我可以给您一些一阶逻辑判定子类的例子。请注意,这些例子都是在前束范式下的公式,即所有量词都在最前面,而且没有函数符号。

% - Bernays-Schonfinkel-Ramsey类的一个例子是:∃x∃y(P(x,y)∧Q(y))。这个公式只有存在量词和两个二元谓词。
% - 二元谓词逻辑的一个例子是:∀x∀y(R(x,y)→S(y,x))。这个公式只有两个变量和两个二元谓词。
% - Horn类的一个例子是:∀x(T(x)→(U(x)∨V(x)))。这个公式只有全称量词和每个合取项至多一个正文字。
    \end{itemize}
\end{frame}

\subsection{EPR logic}
\begin{frame}
    \frametitle{effectively-proposiional fragment of first-prder logic}
    \begin{itemize}
        \item relation, but no function
        \item $\exists^{\star} \forall ^{\star}$, but no $\forall^{\star} \exists ^{\star}$
        \item satisfiablity:
        \begin{align}
            &\exists x, y . \forall z . r(x, z) \leftrightarrow r(z, y) \notag\\
            = _{\text {SAT }}& \forall z \cdot r\left(c_{1}, z\right) \leftrightarrow r\left(z, c_{2}\right) \notag\\
            = { }_{\text {SAT }}&\left(r\left(c_{1}, c_{1}\right) \leftrightarrow r\left(c_{1}, c_{2}\right)\right) \wedge\left(r\left(c_{1}, c_{2}\right) \leftrightarrow r\left(c_{2}, c_{2}\right)\right)
        \end{align}
    \end{itemize}
\end{frame}
\begin{frame}
    \frametitle{checking invariants}
    \begin{itemize}
        \item steps here
    \end{itemize}
\end{frame}
\begin{frame}
    \frametitle{deduction}
    \begin{itemize}
        \item Danfny
        \item lvy
    % cite paper here
    \end{itemize}
\end{frame}