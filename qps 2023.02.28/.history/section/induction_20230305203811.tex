\section{Induction}
\subsection{problem}
\begin{frame}
    \frametitle{backgroud}
    \begin{itemize}
        \item a system $S$
        \item property $\phi$
        \item verification: Is there a behavior of $S$ that violated the property?
    \end{itemize}
\end{frame}

\subsection{induction invariants}
\begin{frame}
    \frametitle{transition system}
    \begin{itemize}
        \item transition system $TS=(S,S_{0},R)$
        \item a safe property $P\subseteq S$
        \item a inductive invariants $I$ iff.
        \begin{itemize}
            \item $S_{0}\subseteq I$
            \item $R(I)=I$
            \item $I\subseteq P$
        \end{itemize}
    \end{itemize}
\end{frame}

\begin{frame}
    \frametitle{poor logic VS. rich logic}
    \begin{itemize}
        \item name of logic:
        \item why poor logic:
        \item what is EPR:
% 好的,我可以给您一些一阶逻辑判定子类的例子。请注意,这些例子都是在前束范式下的公式,即所有量词都在最前面,而且没有函数符号。

% - Bernays-Schonfinkel-Ramsey类的一个例子是:∃x∃y(P(x,y)∧Q(y))。这个公式只有存在量词和两个二元谓词。
% - 二元谓词逻辑的一个例子是:∀x∀y(R(x,y)→S(y,x))。这个公式只有两个变量和两个二元谓词。
% - Horn类的一个例子是:∀x(T(x)→(U(x)∨V(x)))。这个公式只有全称量词和每个合取项至多一个正文字。

% 您对这些例子满意吗?

% 源: 与必应的对话, 2023/3/5(1) 结构和一阶逻辑简介(一) - 知乎. https://zhuanlan.zhihu.com/p/456684658 访问时间 2023/3/5.
% (2) 数理逻辑初步:命题逻辑、一阶逻辑和二阶逻辑_音程的博客-CSDN博客. https://blog.csdn.net/qq_43391414/article/details/117397388 访问时间 2023/3/5.
% (3) 一阶逻辑和高阶逻辑的区别,能不能具象一点说明? - 知乎. https://www.zhihu.com/question/22915503 访问时间 2023/3/5.
    \end{itemize}
\end{frame}
\begin{frame}
    \frametitle{checking invariants}
    \begin{itemize}
        \item ste
    \end{itemize}
\end{frame}
\begin{frame}
    \frametitle{deduction}
    \begin{itemize}
        \item Danfny
        \item lvy
    % cite paper here
    \end{itemize}
\end{frame}

\begin{frame}
    \frametitle{basis denfintion}
    for any language $L\subset 2^{S}$
    \begin{itemize}
        \item $\sqsubseteq_{L}$ on S:$s_{1}\sqsubseteq s_{2}$ iff.$\forall A\in L$,$s_{2}\in A\rightarrow s_{1}\in A$
        \item $\text{Avoid}_{L}(s)=A$:$\forall A^{\prime}\in L, s\notin A^{\prime}\rightarrow A^{\prime}\subseteq A$
        \item L-relaxed transition:$(s,s^{\prime})\in R$ iff.$(s,s^{\prime})\in R$or $s^{prime}\sqsubseteq_{L}s$
    \end{itemize}
\end{frame}