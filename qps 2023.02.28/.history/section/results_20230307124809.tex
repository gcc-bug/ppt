\section{Results of This Paper}
\subsection{basis of this paper}
\begin{frame}
    \frametitle{target}
    \begin{itemize}
        \item the decidability of the problem of inferring inductive invariants in a given language
        \item input:
        \begin{itemize}
            \item program
            \item safe property
            \item a given language
        \end{itemize}
    \end{itemize}
\end{frame}
\begin{frame}
    \frametitle{results}
    \begin{itemize}
        \item decidability: $INV[\mathcal{C}_{n^{\star}},\mathcal{L}_{\forall^{\star}}]$
        \item undecidability: $INV[\mathcal{C}_{n^{\star}},\mathcal{L}_{A-F}]$,$INV[\mathcal{C},\mathcal{L}]$
    \end{itemize}
\end{frame}

\subsection{some proof of results}
\begin{frame}
    \frametitle{basis denfintion}
    for any language $L\subset 2^{S}$
    \begin{itemize}
        \item $\sqsubseteq_{L}$ on S:$s_{1}\sqsubseteq s_{2}$ iff.$\forall A\in L$,$s_{2}\in A\rightarrow s_{1}\in A$
        \item $\text{Avoid}_{L}(s)=A$:$\forall A^{\prime}\in L, s\notin A^{\prime}\rightarrow A^{\prime}\subseteq A$
        \item L-relaxed transition:$(s,s^{\prime})\in R$ iff.$(s,s^{\prime})\in R$ or $s^{\prime}\sqsubseteq_{L}s$
    \end{itemize}
\end{frame}

\begin{frame}
    \frametitle{outline of $INV[\mathcal{C}_{n^{\star}},\mathcal{L}_{n^{\star}}]$}
    \begin{itemize}
        \item L-relaxed Trace reached bad property
        \item Establish WQO using Krusal's Tree theorem
    \end{itemize}
\end{frame}
\begin{frame}
    \frametitle{well quasi order}
    \begin{itemize}
        \item possible case:
        \begin{itemize}
            \item no universal inductive Invariant
            \item no relaxed trace reaches bad
        \end{itemize}
        \item solution: well quasi order
    \end{itemize}
\end{frame}
\begin{frame}
    \frametitle{Krusal's Tree theorem}
    \begin{itemize}
        \item If $(X,\le)$is a wqo, then so is $(\mathcal{T}(X),\le)$.
        \item construct tree
    \end{itemize}
\end{frame}
\begin{frame}
    \frametitle{extend decidability result}
    \begin{corollary}
        Extending the vocabulary $\varSigma $ by adding an arbitrary relation (i.e., with any arity) and extending L by adding to the bodies of L any number $\leqslant $ k of occurrences of the new relation symbol, for some fixed k $\geqslant $ 0, maintains the wqo and computability of AvoidL.
    \end{corollary}
\end{frame}
\begin{frame}
    \frametitle{outline of undecidability results}
    \begin{itemize}
        \item Minsky machine: $M=(Q,c_{1},c_{2})$
        \item Safe problem: $c_3$ no 1
        \item basic idea:
        \begin{itemize}
            \item reduction constructs $(TS,P,L)\in (\mathcal{C},\mathcal{L})$
            \item halts of M and decidability
        \end{itemize}
    \end{itemize}
\end{frame}
\begin{frame}
    \frametitle{reduction from counter Machines to $INV[\mathcal{C},\mathcal{L}]$}
    \begin{itemize}
        \item $inc_i, dec_i, id_i, zero_i, init$
        \item $\varphi_{I}=\bigvee_{\left(q_{i}, \ell_{1}, \ell_{2}, \ell_{3}\right) \in \text { Reach }} q_{i} \wedge \varphi_{\mathcal{E}}\left(\ell_{1}, \ell_{2}, \ell_{3}\right) $
    \end{itemize}
\end{frame}
\begin{frame}
    \frametitle{proof $INV\left[\mathcal{C}_{n^{*}}, \mathcal{L}_{A F}\right]$}
\end{frame}
