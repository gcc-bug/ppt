\documentclass{beamer}
\usepackage{hyperref}
\usepackage{tikz}
\usetikzlibrary{shapes.geometric, positioning, fit, arrows.meta}

\title{Introduction to the Munich Quantum Toolkit (MQT)}
\usepackage{amsmath}
\usepackage{graphicx}
\usepackage{amssymb}
\usepackage{listings}
\usepackage{booktabs}
\usepackage{multirow}
\usepackage{lmodern}
\usepackage{xcolor}
\usepackage{float}
\lstset{
  language=Python,
  rulesepcolor=\color{red!20!green!20!blue!20},
  keywordstyle=\color{blue!90}\bfseries,
  commentstyle=\color{red!10!green!70}\textit,
  basicstyle=\footnotesize,
  showstringspaces=true,
  stringstyle=\rmfamily\slshape\color[RGB]{128,0,0},
  breaklines=true,
  extendedchars=false,
  escapeinside=``,
  texcl=true
}

\lstset{breaklines}
\lstset{extendedchars=false}

\usepackage{textcomp}
\usepackage{pythonhighlight}
\usepackage[backend=bibtex,style=authoryear]{biblatex}
\addbibresource{reference.bib}

\usepackage{algorithm}
\usepackage{algorithmic}
\renewcommand{\algorithmicrequire}{\textbf{Input:}}
\renewcommand{\algorithmicensure}{\textbf{Output:}}

\begin{document}

\frame{\titlepage}

\begin{frame}{Overview of MQT}
    \begin{itemize}
        \item \textbf{What is MQT?}
        \begin{itemize}
            \item developed by the Chair for Design Automation at the Technical University of Munich
            \item provides advanced design automation methods and software tools for quantum computing
            \item their objective is to provide solutions for design tasks across the entire quantum software stack
        \end{itemize}
    \end{itemize}
    \begin{figure}
        \includegraphics[width=\textwidth]{flow}
        \caption{MQT includes high-level support for end users in realizing their applications, efficient methods for the classical simulation, compilation, and verification of quantum circuits, tools for quantum error correction, support for physical design, and more.}
    \end{figure}
\end{frame}

\begin{frame}{Key Features}
    \begin{itemize}
        \item  In all these tools, It try to utilize data structures (such as decision diagrams or the ZX-calculus) and core methods (such as reasoning engines) to
        facilitate the efficient handling of quantum computations
        \item Highlight major features of MQT.
        \begin{itemize}
            \item Quantum Circuit Simulation
            \item Quantum Circuit Compilation
            \item Error Correction and Mitigation
        \end{itemize}
        \item Unique Selling Points
        \begin{itemize}
            \item How MQT stands out from other similar toolkits.
            \item Any unique algorithms or methods used.
        \end{itemize}
    \end{itemize}
\end{frame}

\begin{frame}{Modules and Components}
    \begin{itemize}
        \item Detailed description of each major module or component.
        \begin{itemize}
            \item Simulation Module
            \item Compilation Module
            \item Error Correction Module
        \end{itemize}
        \item How these modules interact with each other.
        \item Examples of use cases for each module.
    \end{itemize}
\end{frame}

\begin{frame}{Installation and Setup}
    \begin{itemize}
        \item Brief guide on how to install MQT.
        \begin{itemize}
            \item Supported platforms (Windows, macOS, Linux).
            \item Dependencies and prerequisites.
        \end{itemize}
        \item Initial setup and configuration.
    \end{itemize}
\end{frame}

\begin{frame}{Usage and Examples}
    \begin{itemize}
        \item Basic usage examples.
        \begin{itemize}
            \item Simple quantum circuit simulation.
            \item Compiling a quantum algorithm.
            \item Applying error correction techniques.
        \end{itemize}
        \item Advanced usage scenarios.
        \begin{itemize}
            \item Integrating MQT into larger quantum computing projects.
            \item Customizing MQT for specific research needs.
        \end{itemize}
    \end{itemize}
\end{frame}

\begin{frame}{Community and Support}
    \begin{itemize}
        \item Community engagement.
        \begin{itemize}
            \item Online forums, discussion groups, and social media presence.
            \item How to contribute to the development of MQT.
        \end{itemize}
        \item Documentation and Tutorials.
        \begin{itemize}
            \item Overview of available documentation and learning resources.
            \item Official tutorials and example projects.
        \end{itemize}
    \end{itemize}
\end{frame}

\begin{frame}{Future Developments}
    \begin{itemize}
        \item Upcoming features and improvements.
        \item Long-term vision and goals for MQT.
        \item Opportunities for collaboration and contribution.
    \end{itemize}
\end{frame}

\begin{frame}{Conclusion}
    \begin{itemize}
        \item Recap of the key points covered in the presentation.
        \item Encourage the audience to explore and use MQT.
        \item Provide contact information or resources for further inquiries.
    \end{itemize}
\end{frame}

\begin{frame}{Q\&A Session}
    \begin{itemize}
        \item Open the floor for questions from the audience.
        \item Prepare to answer common questions about MQT's capabilities, usage, and future plans.
    \end{itemize}
\end{frame}

\end{document}
